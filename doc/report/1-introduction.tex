%------------------------------------------------
\section{Introduction}
%------------------------------------------------

%-----------------------------------------------------------------
\subsection{Contexte pédagogique}
%-----------------------------------------------------------------

% Parcours - ok
Actuellement en deuxième année du master professionnel de Mathématiques : Statistiques, Informatique et Techniques Numériques (SITN) à l’Université Claude Bernard - Lyon 1, je me dois de réaliser un stage dans le but de valider ma formation et d'ainsi obtenir mon diplôme. \newline

%labo -ok
Je suis actuellement accueilli pour une durée de 6 mois au LIRIS - Laboratoire d'InfoRmatique en Image et Systèmes d'information - UMR 5205 CNRS. Le LIRIS est issue de la fusion de plusieurs pôles de recherche de la région Lyonnaise. Aujourd'hui composé d'environ 300 personnes, il participe activement à la recherche et à l'éducation à travers ses deux grands départements thématiques : "Image" et "Données, Connaissances, Services".\newline

%encadrant
Je suis encadré par Tristan Roussillon, Maitre de conférence à l'Insa - Institut National des Sciences Appliquées et membre de l'équipe M2Disco - Modèles Multirésolution, Discrets et Combinatoires. Cette équipe est une composante du département Image du LIRIS qui traite de sujet comme l'analyse d'images, l'optimisation, la programmation par contraintes et la géométrie discrète. C'est d'ailleurs sur cette dernière branche que le contexte de mon stage c'est déroulé.



%-----------------------------------------------------------------
\subsection{Contexte scientifique}
%-----------------------------------------------------------------

J'ai découvert pour l'occassion la géométrie discrète, se nommant également sous le nom plus générique de digital geometry dans la langue de Shakespear. Cette displine de recherche gravite à l'intersection de bien des domaines mathématiques et informatiques.\\

(pics)\\

Il s'agit principalement d'étudier la géométrie et la topologie d'objets portés par des structures régulières. Nous intéressons plus particulièrement au disque discret sur la grille $\mathbb{Z}^{2}$. 
