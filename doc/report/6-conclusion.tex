%------------------------------------------------
\section{Conclusion}
%------------------------------------------------

%%TRI: TODO: réponse à la question posée 
%Notre objectif est de comprendre comment sont organisés spatialement les points du bord d’un disque discret, de comprendre comment cette structure est déterminée par les paramètres du disque (position et taille) par rapport à la grille sous-jacente.

L'algorithme incrémental et output-sensitive de calcul de l'$\alpha$-shape pour $\alpha \leq 0$ peut être vu comme une boite noire qui, à partir d'un certain sommet, donne le suivant dans l'ordre trigonométrique. Ainsi, il fournit une réponse partielle à notre problématique : entre les deux extrémités d'une arête de l'enveloppe convexe, l'organisation spatiale des points est déterminée par les convergents de la pente de l'arête. Néanmoins, les relations de distance entre les sommets de l'enveloppe convexe, données par l'$\alpha$-shape pour $\alpha > 0$, reste encore à étudier, par exemple au moyen d'une approche ``Top-down''.


%-----------------------------------------------------------------
\subsection{Poursuite du projet}
%-----------------------------------------------------------------

Le projet scientifique commencé par l'intermédiaire de ce stage peut être étendu sur bien des points. 

\begin{itemize}
\item Nous avons élaboré un calcul incrémental et output-sensitive des $\alpha$-shape pour $\alpha \leq 0$. Etant donnée la structure récursive des triangulations de Delaunay des motifs de droite discrète, une approche récursive à partir des arêtes de l'enveloppe convexe pourrait être proposée et comparée à celle déjà implémentée. 
\item La preuve de la complexité du calcul de l'enveloppe convexe d'un disque discret en $O(R^{2/3} \log R)$ a été apporté par Har-Peled  \cite{HarPeled98}. Nous conjecturons que la complexité du calcul de l'$\alpha$-shape d'un disque discret est aussi $O(R^{2/3} \log R)$ lorsque $\pm 1 / \alpha$ est en $O(R)$. 
\item Nous avons proposé un calcul ``Bottom-Up'' des $\alpha$-shape pour $\alpha > 0$.  Implémenter l'approche ``Top-down'' pour le cas $\alpha > 0$ permettrait de mieux comprendre les relations de distance dans l'organisation spatiale des points du bord d'un disque discret tout en donnant une complexité plus intéressante. 
\item Nous pourrions aussi traiter d'autres formes convexes, comme l'ellipse.  
\item Enfin, il serait intéressant de mener ce travail en dimension supérieure. En dimension $d$, le nombre de sommets de l'enveloppe convexe de la discrétisation d'une boule est toujours sous-linéaire, plus précisément en $O(R^{d/(d+1)})$ \cite{Balog1991}.
\end{itemize}

%% Enfin, le but de tout algorithme étant d'être utilisé, il serait intéressant une fois le projet achevé de chercher à l'intégrer à la librairie DGtal afin de pouvoir utiliser les $\alpha$-shapes dans un cadre de reconnaissance de forme et d’échantillonnage \cite{BernardiniB97}.


%-----------------------------------------------------------------
\subsection{Compétences acquises}
%-----------------------------------------------------------------

À l'occasion de ce stage, j'ai pu m'améliorer dans des domaines bien distincts. D'un point de vue scientifique, j'ai découvert un domaine de recherche : la géométrie discrète particulièrement intéressant et original dans ses approches de résolutions. À travers l'implémentation des méthodes de résolution couvertes lors de mon stage, j'ai beaucoup appris sur le développement informatique en général. En particulier, j'ai grandement apprécié renouer au C++ à travers la découverte de la programmation générique. Cet apprentissage technique doit être mis en parallèle de l’acquisition de compétences dans le domaine du travail collaboratif et de la gestion de projet. L'utilisation quotidienne de git et de nombreux outils automatisant le projet ont facilités la discussion et le partage autour du programme.

%-----------------------------------------------------------------
\subsection{Ouverture personnelle}
%-----------------------------------------------------------------

Je n'aurai pu souhaiter meilleure fin pour ce manuscrit qui ne signifie pas seulement la fin de ce stage, mais aussi la fin d'un cycle entamé il y a deçà plusieurs années : celui de mes études et de mon enseignement scientifique supérieur. Fort de cette expérience gratifiante et enrichissante, je possède aujourd'hui le sentiment d'être prêt à refermer cette porte pour m'ouvrir vers ce nouvel environnement, qui je l'espère gravitera autour d'un domaine scientifique et continuera à m'apprendre et m'enseigner dans des domaines divers et variés.

