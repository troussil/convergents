%------------------------------------------------
\section{Annexes}
%------------------------------------------------

%-----------------------------------------------------------------
\subsection{Équivalence entre le calcul du pgcd par la méthode Euclidienne et géométrique}
%-----------------------------------------------------------------

%-----------------------------------------------------------------
\subsubsection{Trouver $q_n$}

Soit d la droite d'équation ax + by + c = 0 et soit un rayon issue de $S=(x_s, y_s)$ dans la direction du vecteur $V=(x_v, y_v)$.\\
Soit $M(x,y)$ un point tel que $\overrightarrow{SM} = q \overrightarrow{V}$. Les coordonnées de M vérifient : \\
 
\begin{equation*}
    \left\{
    \begin{split}
      x - x_s &= q x_v \\
      y - y_s &= q y_v 
   \end{split}
   \right. 
\end{equation*}

\begin{equation*}
    \left\{
    \begin{split}
      x &= q x_v + x_s \\
      y &= q y_v + y_s 
   \end{split}
   \right. 
\end{equation*}


On choisit le point M comme l'intersection de la droite d d'équation d : ax + by +c = 0 et du rayon. Les coordonnées de M vérifient : \\

$$(a x_v + b y_v) * q = -(a x_s + b y_s +c)$$
$$q = - \frac{a x_s + b y_s +c}{(a x_v + b y_v)}$$

%-----------------------------------------------------------------
\subsubsection{Rappel sur le calcul du pgcd par l'algorithme d'Euclide}


On rappelle les résultats de l'algorithme d'Euclide pour a,b : \\
\noindent $a       = k_0*b       + r_0$ d'où $k_0 = \lfloor b / a \rfloor$\\
\noindent $b       = k_1*r_0     + r_1$\\
\noindent $r_{n-2} = k_n*r_{n-1} + r_{n}$\\

%-----------------------------------------------------------------
\subsubsection{Preuve par réccurence sur n que $q_n = k_n$}

\paragraph{}

On vérifie que que la $q_0 = k_0$.\\

Soit de la droite passant par O(0,0) et P(b, a) d'équation d: ax + by =0 et soit le rayon issue de $p_{-2} = (1,0)$ dans la direction de $p_{-1} = (0,1)$.\\

On note le convergent $p_{0} = p_{-2} + q_0 p_{-1}$ avec $q_0$ est le plus grand entier tel que $p_0$ et $p_{-2}$ soient du même côté. Sa valeur est la partie entière de $-(a x_s + b y_s +c)/(a x_v + b y_v)$

$$q_0 = - \lfloor\frac{a*1 + b*0 +0}{a*0 + b*1}\rfloor = - \lfloor\frac{a}{b}\rfloor = k_0$$

On a montré que la relation est vraie au rang 0 avec \textbf{$q_0 = k_0$}.

\paragraph{}
On suppose vrai la relation au rang n.

$q_n = k_n$\\

Ce qui implique :  
$r_{n-2} = k_n*r_{n-1} + r_{n}$\\
$a x_{n-2} + b y_{n-2} = r_{n-2}$\\
$a x_{n-1} + b y_{n-1} = r_{n-1}$\\

Soit le rayon issue de $p_{n-1}$ dans la direction de $p_{n}$.\\
On note le convergent $p_{n+1} = p_{n-1} + q_{n+1} p_{n}$. $q_{n+1}$ est le plus grand entier tel que $p_{n+1}$ et $p_{n-1}$ soient du même côté. Sa valeur est la partie entière de $-(a x_{n-1} + b y_{n-1} +c)/(a x_{n} + b y_{n})$


$q_0$ est égale au premier coefficient de la division Euclidienne de b par a.

\begin{align*}
q_{n+1} &= - \lfloor\frac{a x_{n-1} + b y_{n-1}}{ a x_{n} + b y_{n}}\rfloor \\
        &= - \lfloor\frac{a x_{n-1} + b y_{n-1}}{ (a (x_{n-2} + q_n * x_{n-1}) + b (y_{n-2} + q_n * y_{n-1})}\rfloor \\
        &= - \lfloor\frac{a x_{n-1} + b y_{n-1}}{ q_n*(a x_{n-1} + b*y_{n-1} ) + a x_{n-2} + b*y_{n-2}  }\rfloor \\
        &= - \lfloor\frac{r_{n-1}              }{ r_{n-2} + k_n * r_{n-1}}\rfloor \\
        &= - \lfloor\frac{r_{n-1}              }{ r_{n} }\rfloor \\
        &= k_{n+1}
\end{align*}

On a montré que \textbf{$q_{n+1}= k_{n+1}$}. Comme la relation est vraie au rang 0 et au rang (n+1),  \textbf{elle est vraie pour tout n > 0}.



