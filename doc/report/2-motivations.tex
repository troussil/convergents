%------------------------------------------------
\section{Motivations}
%------------------------------------------------

%-----------------------------------------------------------------
\subsection{Le bord du disque discret}
%-----------------------------------------------------------------

%-----------------------------------------------------------------
\subsubsection{Du disque Euclidien au disque discret}

Avant de s'attarder sur la représentation d'un disque discret, il convient de revenir un court instant au disque fermé Euclidien et à l'ensemble reprenant tous les points de son bord, le cercle Euclidien.

\begin{Definition}{Disque fermé Euclidien}
\label{def:disk-euc}
 $$\mathcal{D}_e =  \left\{ (x,y) \in \mathbb{R}^{2} |  (x - u)^2 + (y - v)^2 \leq R^2 \right\}$$
\end{Definition}

\begin{Definition}{Cercle Euclidien}
\label{def:cer-euc}
  $$\mathcal{C}_e =  \left\{ (x,y) \in \mathbb{R}^{2} |  (x - u)^2 + (y - v)^2 = R^2 \right\}$$
  avec $(u,v) \in \mathbb{R}^{2}$ les coordonnées du centre et $R \in \mathbb{R}^{+*}$ le rayon.\\
\end{Definition}



\begin{figure}[H]
  \centering
  \includegraphics[width=6cm]{fig/2-mot/circle/circle-euc-0.pdf}
  \caption{Disque et Cercle Euclidiens}
\end{figure}
%PICS et PICS de disque et de cercle euclidien

Si la définition du disque fermé discret dans $\mathbb{Z}^{2}$ (appelé pour la suite du rapport uniquement disque discret) reste très proche de la définition Euclidienne, l'équivalent discret de son cercle afin de récupérer tous les points du bord change quant à elle considérablement.

\begin{Definition}{Disque discret}
\label{def:disk-dis}
  $$\mathcal{D} =  \left\{ (x,y) \in \mathbb{Z}^{2} |  (x - u)^2 + (y - v)^2 = R^2 \right\}$$
  avec $(u,v) \in \mathbb{R}^{2}$ les coordonnées du centre et $R \in \mathbb{R}^{+*}$ le rayon.\\
\end{Definition}



\begin{figure}[H]
  \centering
  \includegraphics[width=6cm]{fig/2-mot/circle/circle-dis-0.pdf}
  \caption{Disque discret et Ensemble }
\end{figure}

L'ensemble des points représentés en rouge sur le dessin, appartenant à l'ensemble : $\left\{ (x,y) \in \mathbb{Z}^{2} |  (x - u)^2 + (y - v)^2 = R^2 \right\}$, étant donc positionnés exactement sur le cercle Euclidien ne suffisent pas pour représenter les points du bord du disque discret. Il faut choisir une définition plus générale.

\begin{Definition}{Cercle discret}
\label{def:cer-dis}
  $$ \mathcal{C} =  \left\{ (x,y) \in \mathbb{Z}^{2} | \exists (i,j) \in \{ (0,1), (1,0), (0,-1), (-1,0)\}, | (x+i,y+j) \notin \mathcal{D} \right\}$$
\end{Definition}


\begin{figure}[H]
  \centering
  \includegraphics[width=6cm]{fig/2-mot/circle/circle-dis-1.pdf}
  \caption{Cercle discrets}
\end{figure}

%-----------------------------------------------------------------
\subsubsection{L'étude des points du bord}

À partir de la définition précédente du cercle discret, il est possible de formuler plusieurs remarques. De par le fait de travailler sur $\mathbb{Z}^{2}$, on sait que les points sont ordonnées sur une grille régulière. Pour chaque point, il est possible d'atteindre quatre voisins en se dirigeant d'une unité vers les quatre points cardinaux. Chaque point possède donc un voisin, en haut, à gauche, en bas et à droite de lui. Or, on observe que les points strictement à l'intérieur du disque discret et non sur le bord (en bleu clair) possèdent tous leurs quatre voisins à l'intérieur du disque (maillage en rouge). 

Pour autant, les points du bords du disque discret possèdent eux une distribution différente. Chacun des points du bord possède entre un et trois voisins à l'extérieur du disque (en vert clair) et son complémentaire à quatre à l'intérieur.\\

\begin{figure}[H]
  \centering
  \includegraphics[width=6cm]{fig/2-mot/circle/circle-dis-2.pdf}
  \includegraphics[width=6cm]{fig/2-mot/circle/circle-dis-3.pdf}
  \caption{Voisins des points strictement à l'intérieur et sur le bord }
\end{figure}

Pour définir un disque discret, on se retrouve avec deux ensembles. Le premier est bien ordonnée. Il représente les points strictement à l'intérieur du disque. Le deuxième ensemble représente les points du bord du disque et propose une distribution de points non régulière. De part la régulartité du premier, seule l'étude des points du bord nous semble pertinente pour comprendre l'organisation et la structure des disque discret.


%-----------------------------------------------------------------
\subsection{Alpha-Shape}
%-----------------------------------------------------------------

%-----------------------------------------------------------------
\subsubsection{Définition générale}

En s'intéressant principalement à des contours de formes discrètes, un ensemble d'outil nous est apparu comme particulièrement opportun. Il s'agit des $\alpha$-hull et des $\alpha$-shape définit pour la première fois par edel (référence requise) de la manière suivante.

\begin{Definition}{$\alpha$-hull de $\mathcal{S}$}
\label{def:ah}
    $$\left\{ \cap \mathcal{D}_{\alpha} | \forall (x,y)\in \mathcal{S} \subset \mathbb{Z}^{2} \Rightarrow (x,y) \in \mathcal{D}_{\alpha} \right\}$$
    Intersection de tous les disques généralisés de rayon $1/\alpha$ qui contiennent tous les points de l'ensemble.
\end{Definition}

%PICS avec les alpha hull

Les sommets de $\alpha$-hull, les points appartenant à son bord de l'$\alpha$-hull sont appelés points $\alpha$-extreme. S'ils sont situés sur le même bord en étant relié par un arc de cercle de rayon 1/ $\lvert \alpha \rvert$, on dit alors qu'ils sont adjacents.

\begin{Definition}{$\alpha$-shape}\\
\label{def:as}
      Enveloppe reliant tous les $\alpha$-extremes adjacents.
\end{Definition}

%PICS avec les alpha-shape\\

L'utilité des $\alpha$-shapes est d'être une sous-ensemble des points du bord. 
% need more

%-----------------------------------------------------------------
\subsubsection{$\alpha$-shapes de disque discret}

En prenant un panel assez large d'$\alpha$-shape avec $\alpha$ succéssivement négatif et positif variant de -2 à $R_D$ (le rayon du disque) on remarque que les $\alpha$-shapes représentent un large panel d'ensemble représentatif des points du bord d'un disque discret.

%PICS avec alpha shapes de disque.

%-----------------------------------------------------------------
\subsubsection{Plusieurs remarques}

\paragraph{}
Le cas central de $\alpha = 0$ représente une intersection de disque de rayon infini, cela peut être interprété comme une intersection de demi-plan. On retrouve exactement le calcul de l'enveloppe convexe.

\paragraph{}
Les plus petits $\alpha = -2$ et $\alpha = -1$ représenter les cas bien connus de suivi de bord. En effet, en ne pouvant s'éloigner au plus de disques de rayon -2 et -1, on ne peut suivre le bord de nos disques que par l'intermédiaire des voisins 4-connexes et 8-connexes de chaque point du bord.

\paragraph{}
En réalisant l'union de nos $\alpha$-shape pour le cas négatif et le cas positif, il semblerait que l'on s'appuie également sur les triangulations d'ordre 0 et n de Delaunay. Mais nous reviendrons dessus plus en détail dans la section "Existants".\\

% need more 
