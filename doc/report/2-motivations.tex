%------------------------------------------------
\section{Motivations}
%------------------------------------------------

%-----------------------------------------------------------------
\subsection{Le bord du disque discret}
%-----------------------------------------------------------------

%-----------------------------------------------------------------
\subsubsection{Du disque Euclidien au disque discret}


Le disque discret $\mathcal{D}$ est défini par analogie avec le disque Euclidien $\mathcal{D}_e$ qui représente l'ensemble des points situé à une distance inférieur à $R$ de son centre $O$.


\begin{Definition}{Disque fermé Euclidien}
\label{def:disk-euc}
 $$\mathcal{D}_e =  \left\{ p=(x,y) \in \mathbb{R}^{2} |  (x - u)^2 + (y - v)^2 \leq R^2 \right\}$$
\end{Definition}

\begin{Definition}{Disque fermé discret}
\label{def:disk-dis}
  $$\mathcal{D} =  \left\{ p=(x,y) \in \mathbb{Z}^{2} |  ax + by + c( x^2 + y^2 ) + d = 0. \right\}$$
  
  avec $(u,v) = ( -a/(2c) , -b/(2c) )) \in \mathbb{Z}^{2}$ les coordonnées du centre et $R = \sqrt{ (a^2 + b^2 - 4cd) / 4c^2 } \in \mathbb{Z}^{+*}$ le rayon.\\
\end{Definition}

\begin{figure}[H]
  \centering
  \includegraphics[width=6cm]{fig/2-mot/circle/circle-euc-0.pdf}
  \includegraphics[width=6cm]{fig/2-mot/circle/circle-dis-0.pdf}
  \caption{Disque Euclidiens et Discret}
\end{figure}


Si la définition du disque fermé discret dans $\mathbb{Z}^{2}$ (appelé pour la suite du rapport uniquement disque discret) reste très proche de la définition Euclidienne, l'équivalent discret de son cercle afin de récupérer tous les points du bord change quant à elle considérablement. L'ensemble des points représentés en rouge sur le dessin, appartenant à l'ensemble : $\left\{ (x,y) \in \mathbb{Z}^{2} |  (x - u)^2 + (y - v)^2 = R^2 \right\}$, étant donc positionnés exactement sur le cercle Euclidien ne suffisent pas pour représenter le disque discret et son ensemble de points sur le bord. Il faut choisir une définition plus générale.

\begin{Definition}{Cercle Euclidien}
\label{def:cer-euc}
  $$\mathcal{C}_e =  \left\{ (x,y) \in \mathbb{R}^{2} |  (x - u)^2 + (y - v)^2 = R^2 \right\}$$
  avec $(u,v) \in \mathbb{R}^{2}$ les coordonnées du centre et $R \in \mathbb{R}^{+*}$ le rayon.\\
\end{Definition}

\begin{Definition}{Cercle discret}
\label{def:cer-dis}
  $$ \mathcal{C} =  \left\{ p=(x,y) \in \mathbb{D} | \exists q \notin \mathbb{D} | d(p, q) = 1  \right\}$$
\end{Definition}

\begin{figure}[H]
  \centering
  \includegraphics[width=6cm]{fig/2-mot/circle/circle-dis-1.pdf}
  \caption{Cercle Euclidiens}
\end{figure}
%PICS et PICS de disque et de cercle euclidien


%-----------------------------------------------------------------
\subsubsection{L'étude des points du bord}

À partir de la définition précédente du cercle discret, il est possible de formuler plusieurs remarques. De par le fait de travailler sur $\mathbb{Z}^{2}$, on sait que les points sont ordonnées sur une grille régulière. Pour chaque point, il est possible d'atteindre quatre voisins en se dirigeant d'une unité vers les quatre points cardinaux. Chaque point possède donc un voisin, en haut, à gauche, en bas et à droite de lui. Or, on observe que les points strictement à l'intérieur du disque discret et non sur le bord (en bleu clair) possèdent tous leurs quatre voisins à l'intérieur du disque (maillage en rouge).

\begin{Definition}{Ensemble de points strictement à l'intérieur d'un disque}
\label{def:cer-dis}
  $$ \stackrel{\ \circ}{D} =  \left\{ p=(x,y) \in \stackrel{\ \circ}{D} | d(p, q) = 1 \Rightarrow q \in \mathcal{D} \right\}$$
\end{Definition}

Pour autant, les points du bords du disque discret \RefDef{def:cer-dis}, possèdent eux une distribution différente. Chacun des points du bord possède entre un et trois voisins à l'extérieur du disque (en vert clair).

\begin{figure}[H]
  \centering
  \includegraphics[width=6cm]{fig/2-mot/circle/circle-dis-2.pdf}
  \includegraphics[width=6cm]{fig/2-mot/circle/circle-dis-3.pdf}
  \caption{Voisins des points strictement à l'intérieur et sur le bord }
\end{figure}

On se retrouve alors avec deux ensembles disjoints pour définir et représenter un disque discret. Le premier est bien ordonnée. Il représente les points strictement à l'intérieur du disque. Le deuxième ensemble représente les points du bord du disque et propose une distribution de points non régulière. De part la régulartité du premier, seule l'étude des points du bord nous semble pertinente pour comprendre l'organisation et la structure des disque discret.


%-----------------------------------------------------------------
\subsection{Alpha-Shape}
%-----------------------------------------------------------------

%-----------------------------------------------------------------
\subsubsection{Définition générale}

En s'intéressant principalement à des contours de formes discrètes, un ensemble d'outil nous est apparu comme particulièrement opportun. Il s'agit des $\alpha$-hull et des $\alpha$-shape définit pour la première fois dans \cite{EdeKirSei83} de la manière suivante.\\



\begin{Definition}{$\alpha$-hull de $\mathcal{S}$}
\label{def:ah-txt}
    Intersection de tous les disques généralisés de rayon $1/\alpha$ qui contiennent tous les points de l'ensemble.
\end{Definition}

Un disque généralisé permet de définir des disques avec des rayons négatifs en faisant appel au complémentaire.

\begin{Definition}{Disques génralisés de rayon $1/\alpha$}
\label{def:ah-txt}
   Si $\alpha > 0$, $\mathcal{D}_{\alpha}$ est le disque fermé de rayon $1/\alpha$.\\
   Si $\alpha > 0$, $\mathcal{D}_{\alpha}$ est le complémentaire fermé du disque de rayon $1/\alpha$.
\end{Definition}

On en déduit une nouvelle définition pour les $\alpha$-hulls.

\begin{Definition}{$\alpha$-hull de $\mathcal{S}$}
\label{def:ah}
    $$\left\{ \cap \mathcal{D}_{\alpha} | \forall (x,y)\in \mathcal{S} \Rightarrow (x,y) \in \mathcal{D}_{\alpha} \right\}$$
\end{Definition}

\begin{figure}[h!]
  \centering
  \includegraphics[width=0.4\linewidth,page=1]{fig/2-mot/as/mot-alpha-shape.pdf}
  \includegraphics[width=0.4\linewidth,page=3]{fig/2-mot/as/mot-alpha-shape.pdf}
  \caption{$\alpha$-Hull négative et $\alpha$-Hull positive }
\end{figure}

       

Les sommets de $\alpha$-hull sont appelés points $\alpha$-extreme. S'ils sont relié par un arc de cercle de rayon 1/ $\lvert \alpha \rvert$ qui n'exclue pas de points, on dit alors qu'ils sont adjacents.

\begin{Definition}{$\alpha$-shape}\\
\label{def:as}
      Graphe "plongé" reliant tous les $\alpha$-extremes adjacents par des segments de droite discrète.
\end{Definition}

\begin{figure}[h!]
  \centering
  \includegraphics[width=0.4\linewidth,page=2]{fig/2-mot/as/mot-alpha-shape.pdf}
  \includegraphics[width=0.4\linewidth,page=4]{fig/2-mot/as/mot-alpha-shape.pdf}
  \caption{$\alpha$-Shape négative et $\alpha$-Shape positive }
\end{figure}

L'utilité des $\alpha$-shapes est d'être une sous-ensemble des points du bord. 
% need more

%-----------------------------------------------------------------
\subsubsection{$\alpha$-shapes de disque discret}

En prenant un panel assez large d'$\alpha$-shape avec $\alpha$ succéssivement négatif et positif variant de -2 à $R_D$ (le rayon du disque) on remarque que les $\alpha$-shapes représentent un large panel d'ensemble représentatif des points du bord d'un disque discret.

%PICS avec alpha shapes de disque.

%-----------------------------------------------------------------
\subsubsection{Plusieurs remarques}

\paragraph{}
Le cas central de $\alpha = 0$ représente une intersection de disque de rayon infini, cela peut être interprété comme une intersection de demi-plan. On retrouve exactement l'enveloppe convexe.

\paragraph{}
Les plus petits $\alpha = -2$ et $\alpha = -\sqrt{2}$ représenter les bords. En effet, en ne pouvant s'éloigner au plus de disques de rayon $1/2$ et $\sqrt{2}/2$, on ne peut suivre le bord de nos disques que par l'intermédiaire des voisins 4-connexes et 8-connexes de chaque point du bord.

\paragraph{}
En réalisant l'union de nos $\alpha$-shape pour le cas négatif et le cas positif, on s'appuie également sur les triangulations d'ordre 0 et n de Delaunay. 
% need more 
