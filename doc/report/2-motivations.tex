%------------------------------------------------
\section{Motivations}
%------------------------------------------------

%-----------------------------------------------------------------
\subsection{Le bord du disque discret}
%-----------------------------------------------------------------

%-----------------------------------------------------------------
\subsubsection{Du disque Euclidien au disque discret}


Le  disque fermé discret $\mathcal{D}$ dans $\mathbb{Z}^{2}$ (appelé pour la suite du rapport uniquement disque discret) est défini par analogie avec le disque Euclidien $\mathcal{D}_e$. Il représente l'ensemble des points situés à une distance inférieure à $R$ de son centre $O(u,v)$.


\begin{Definition}{Disque fermé Euclidien}
\label{def:disk-euc}
 $$\mathcal{D}_e =  \left\{ (x,y) \in \mathbb{R}^{2} |  (x - u)^2 + (y - v)^2 \leq R^2 \right\}$$
\end{Definition}

\begin{Definition}{Disque fermé discret}
\label{def:disk-dis}
  $$\mathcal{D} =  \left\{ (x,y) \in \mathbb{Z}^{2} |  (x - u)^2 + (y - v)^2 \leq R^2 \right\}$$
  avec $O(u,v) \in \mathbb{Q}^{2}$ les coordonnées du centre et $R^2 \in \mathbb{Q}$ le rayon.\\
\end{Definition}

\begin{figure}[H]
  \centering
  \includegraphics[width=6cm]{fig/2-mot/circle/circle-euc-0.pdf}
  \includegraphics[width=6cm]{fig/2-mot/circle/circle-dis-0.pdf}
  \caption{Disque Euclidien et Discret de centre O et de rayon R}
\label{fig:disk}
\end{figure}

La définition du cercle Euclidien découle directement de la définition du disque Euclidien en changeant l'inégalité par une égalité. Cependant, la définition du cercle discret ne dérive pas explicitement de la définition du disque discret. L'ensemble des points, représentés en rouge sur \textsc{figure} \ref{fig:disk} et appartenant à l'ensemble : $\left\{ (x,y) \in \mathbb{Z}^{2} |  (x - u)^2 + (y - v)^2 = R^2 \right\}$ ne suffit pas à représenter un cercle discret. Nous avons besoin de revenir à une définition plus générale du bord de tout ensemble discret en faisant appel à la notion de voisinage.\\

\begin{Definition}{4-Voisinage d'un point $(u,v)$}
\label{def:vois-4}
  $$\mathcal{V}_4(u,v) =  \left\{ (x,y) \in \mathbb{Z}^{2} |  |x-u|+|y-u| = 1 \right\}$$
\end{Definition}

\begin{Definition}{8-Voisinage d'un point $(u,v)$}
\label{def:vois-8}
  $$\mathcal{V}_8(u,v) =  \left\{ (x,y) \in \mathbb{Z}^{2} |  max(|x-u|,|y-u|) = 1 \right\}$$
\end{Definition}

\begin{figure}[H]
  \centering
  \includegraphics[width=.5\linewidth]{fig/2-mot/connexe/connexite.pdf}
  \caption{4-voisinage et 8-voisinage}
\end{figure}

Le bord d'un ensemble discret $S$ se traduit par l'ensemble des points à l'intérieur de $S$ dont le 4-voisinage (respectivement 8) n'est pas intégralement contenu dans $S$. Le bord récupéré est un ensemble 8-connexes (respectivement 4). Une illustration est donnée \textsc{figure} \ref{fig:bord}. Le cercle discret est défini commme le bord d'un disque discret. Nous utilisons le joker * pour désigner arbitrairement $4$ ou $8$. 

\begin{Definition}{Bord *-connexe d'un ensemble discret $S$}
\label{def:bord-ens}
  $$ \partial S_{*} =  \left\{ (x,y) \in S | \left( \mathcal{V}_{*}(x,y) \cap S \right) \neq \mathcal{V}_{*}(x,y) \right\}$$
\end{Definition}

\begin{Definition}{Cercle discret défini par le *-voisinage}
\label{def:cer-dis}
  $$ \mathcal{C}_{*} =  \left\{ (x,y) \in \mathcal{D} | \left( \mathcal{V}_{*}(x,y) \cap \mathcal{D} \right) \neq \mathcal{V}_{*}(x,y) \right\}$$
\end{Definition}

\begin{figure}[H]
  \centering
  \includegraphics[width=.3\linewidth]{fig/2-mot/circle/circle-dis-1a.pdf}
  \includegraphics[width=.3\linewidth]{fig/2-mot/circle/circle-dis-1b.pdf}
  \caption{Cercle Discret 8-connexes et 4-connexes}
\label{fig:bord}
\end{figure}


%-----------------------------------------------------------------
\subsubsection{Énoncé de la problématique}

Les points discrets sont organisés sur la grille régulière $\mathbb{Z}^{2}$. Chaque point discret possède donc quatre plus proches voisins à une distance d'une unité, en haut, à gauche, en bas et à droite de lui. Or, les points strictement à l'intérieur du disque discret et non sur le bord (en bleu clair sur \textsc{figure} \ref{fig:disque-bord}) possèdent tous leurs quatre plus proches voisins à l'intérieur du disque (maillage en rouge sur \textsc{figure} \ref{fig:disque-bord}).

\begin{Definition}{Ensemble de points strictement à l'intérieur d'un disque}
\label{def:int-ens}
  $$\stackrel{\ \circ}{\mathcal{D}}_{*} = \mathcal{D} / \mathcal{C_{*}} $$
  $$ \stackrel{\ \circ}{\mathcal{D}}_{*} =  \left\{ (x,y) \in \mathcal{D} | \mathcal{N}_{*}(x,y) \cap \mathcal{D} = \mathcal{N}_{*}(x,y) \right\}$$
\end{Definition}

\begin{figure}[H]
  \centering
  \includegraphics[width=.3\linewidth]{fig/2-mot/circle/circle-dis-2.pdf}
  \caption{Réseau de points strictement à l'intérieur d'un disque.}
\label{fig:disque-bord}
\end{figure}

Le disque discret est l'union de deux ensembles disjoints : l'intérieur et le bord. La structure du premier est évidente. C'est pourquoi, seule l'étude du second, le bord, nous intéresse pour comprendre l'organisation des points des disques discrets.\\

Notre objectif est de comprendre comment sont organisés spatialement les points du bord d'un disque discret, de comprendre comment cette structure est déterminée par les paramètres du disque (position et taille) par rapport à la grille sous-jacente.   

%-----------------------------------------------------------------
\subsection{Alpha-Shape}
%-----------------------------------------------------------------

En s'intéressant à des bords d'objets discrets, un panel d'outils nous est apparu comme particulièrement opportun. Il s'agit des $\alpha$-hulls et des $\alpha$-shapes définis pour la première fois par Edelsbrunner \emph{et. al.} \cite{EdeKirSei83} et faisant appel aux disques généralisés.\\

Un disque généralisé permet de définir des disques avec des rayons négatifs en faisant appel au complémentaire.

\begin{Definition}{Disques généralisés de rayon $1/\alpha$}
\label{def:dis-gen}
\begin{itemize}
  \item $\mathcal{D}_{\alpha}$ est le disque fermé de rayon $1/\alpha$ pour $\alpha > 0$.
  \item $\mathcal{D}_{\alpha}$ est le complémentaire fermé du disque de rayon $- 1/\alpha$ pour $\alpha < 0$. 
\end{itemize}
\end{Definition}

%-----------------------------------------------------------------
\subsubsection{Définition}

Soit $\mathcal{S}$ un ensemble fini de points. 

\begin{Definition}{$\alpha$-hull de $\mathcal{S}$}\\
\label{def:ah-txt}
    Intersection de tous les disques généralisés de rayon $1/\alpha$ qui contiennent tous les points de l'ensemble.
    $$ \alpha_h(\mathcal{S}) = \cap \left\{ \mathcal{D}_{\alpha} | \mathcal{S} \subseteq \mathcal{D}_{\alpha} \right\}$$
\end{Definition}

\begin{figure}[H]
  \centering
  \includegraphics[width=0.3\linewidth,page=1]{fig/2-mot/as/mot-alpha-shape.pdf}
  \includegraphics[width=0.3\linewidth,page=3]{fig/2-mot/as/mot-alpha-shape.pdf}
  \caption{$\alpha$-Hull négative et $\alpha$-Hull positive }
\end{figure}
  
Les sommets de $\alpha$-hull sont appelés points $\alpha$-extrêmes. S'ils sont reliés par un arc de cercle de rayon $\pm 1/ \alpha$ qui ne contient aucun autre point que ses extrémités et qui se trouve sur le bord d'un disque généralisé contenant l'ensemble des points, on dit qu'ils sont $\alpha$-adjacents.

\begin{Definition}{$\alpha$-shape}\\
\label{def:as}
      Graphe plongé dans le plan reliant tous les points $\alpha$-extrêmes adjacents par des segments de droite.
\end{Definition}

\begin{figure}[H]
  \centering
  \includegraphics[width=0.3\linewidth,page=2]{fig/2-mot/as/mot-alpha-shape.pdf}
  \includegraphics[width=0.3\linewidth,page=4]{fig/2-mot/as/mot-alpha-shape.pdf}
  \caption{$\alpha$-Shape négative et $\alpha$-Shape positive }
\end{figure}


%-----------------------------------------------------------------
\subsubsection{$\alpha$-shapes de disque discret}

Le paramètre $\alpha$ est défini dans l'intervalle allant de $-2$ à $R_{min}$ (le rayon du plus petit cercle englobant). 
Dans cet intervalle un vaste ensemble d'$\alpha$-shape est possible.  
%Les $\alpha$-shapes représentent un sous-ensemble des points du bord du disque discret.

\begin{figure}[H]
  \centering
  \includegraphics[width=0.4\linewidth]{fig/2-mot/as/mot-as-1.pdf}
  \includegraphics[width=0.4\linewidth]{fig/2-mot/as/mot-as-2.pdf}
  \caption{$\alpha$-shapes disque discret - $\alpha < 0$}
\end{figure}

\begin{figure}[H]
  \centering
  \includegraphics[width=0.4\linewidth]{fig/2-mot/as/mot-as-3.pdf}
  \includegraphics[width=0.4\linewidth]{fig/2-mot/as/mot-as-4.pdf}
  \caption{$\alpha$-shapes de disque discret - $\alpha = 0$ et - $\alpha > 0$}
\end{figure}



%-----------------------------------------------------------------
\subsubsection{Propriétés}


\begin{itemize}
  \item Le cas où $\alpha = 0$ est une intersection de disques généralisés de rayon infini. Ce cas s'interprète comme une intersection de demi-plans, menant à la définition de l'enveloppe convexe.
  \item Les cas où $\alpha = -2$ et $\alpha = -\sqrt{2}$ correspondent aux bords définis au moyen du 8 et 4-voisinage. 
  \item L'union des $\alpha$-shape \cite{EdeKirSei83} pour le cas négatif et le cas positif représente des sous-ensembles des triangulations d'ordre 0 et d'ordre n de Delaunay.   
\end{itemize}
 
%-----------------------------------------------------------------
\subsection{Triangulation de Delaunay}
%-----------------------------------------------------------------

La triangulation est un moyen de relier les points d'un ensemble entre eux. Les triangulations de Delaunay possèdent des propriétés intéressantes.  Elles ont été nommées d'après le mathématicien russe Boris Delone : 1890-1980.

Soit $\mathcal{S}$ un ensemble fini de points. 

\begin{Definition}{Triangulation de Delaunay d'ordre 0 de $\mathcal{S}$}\\
\label{def:tri-del-0}
  La triangulation de Delaunay d'ordre 0 est une triangulation où chaque disque circonscrit au triangle ne contient aucun autre point que les sommets du triangle.
\end{Definition}

\begin{Definition}{Triangulation de Delaunay d'ordre n de $\mathcal{S}$}\\
\label{def:tri-del-n}
  La triangulation de Delaunay d'ordre n est une triangulation de l'enveloppe convexe de $\mathcal{S}$ où chaque disque circonscrit au triangle contient tous les points de l'ensemble.
\end{Definition}

\begin{figure}[H]
  \centering
  \includegraphics[width=0.4\linewidth]{fig/2-mot/tri/mot-tri-a.pdf}
  \includegraphics[width=0.4\linewidth]{fig/2-mot/tri/mot-tri-b.pdf}
  \caption{Triangulations d'ordre 0 et n}
\end{figure}


Les triangulations de Delaunay sont reliées aux $\alpha$-shapes par plusieurs propriétés. \cite{EdeKirSei83}

\begin{Lemma}
  Une $\alpha$-shape de $\mathcal{S}$ est un sous-graphe de la triangulation de Delaunay.
\end{Lemma}

\begin{Lemma}
  Pour tout point $p \in \mathcal{S}$, il existe un réel $\alpha_{max}(p)$ tel que $p$ soit un $\alpha$-extrême de $\mathcal{S}$ si et seulement si $\alpha \leq \alpha_{max}(p)$.
\end{Lemma}

\begin{Lemma}
  Toutes arêtes $e$ de notre triangulation de Delaunay est également une arête de l'$\alpha$-shape s'il existe $\alpha_{min}(e) \leq \alpha_{max}(e)$ tel que $\alpha_{min}(e) \leq \alpha \leq \alpha_{max}(e)$.
\end{Lemma}
