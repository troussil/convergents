%%%%%%%%%%%%%%%%%%%%%%%%%%%%%%%%%%%%%%%%%
% Simple Sectioned Essay Template
% LaTeX Template
%
% This template has been downloaded from:
% http://www.latextemplates.com
%
% Note:
% The \lipsum[#] commands throughout this template generate dummy text
% to fill the template out. These commands should all be removed when 
% writing essay content.
%
%%%%%%%%%%%%%%%%%%%%%%%%%%%%%%%%%%%%%%%%%

%----------------------------------------------------------------------------------------
%	PACKAGES AND OTHER DOCUMENT CONFIGURATIONS
%----------------------------------------------------------------------------------------

\documentclass[12pt]{article} % Default font size is 12pt, it can be changed here
\usepackage{geometry} % Required to change the page size to A4
\geometry{a4paper} % Set the page size to be A4 as opposed to the default US Letter

\usepackage{graphicx} % Required for including pictures
\usepackage{float} % Allows putting an [H] in \begin{figure} to specify the exact location of the figure


\usepackage[T1]{fontenc} %codage
\usepackage[english,francais]{babel}
\usepackage[utf8]{inputenc} %codage
\usepackage{lmodern}

\usepackage{amsmath,amsfonts,amssymb,amsthm}

\usepackage[linesnumbered, ruled, vlined]{algorithm2e}
\SetAlFnt{\small\sffamily}

\usepackage{url}

\usepackage{tikz}         
\usetikzlibrary{mindmap}

\linespread{1.2} % Line spacing

%\setlength\parindent{0pt} % Uncomment to remove all indentation from paragraphs





\begin{document}

%----------------------------------------------------------------------------------------
%	RE-DEFINITION 
%----------------------------------------------------------------------------------------
% MATHS
%-----------

\newtheorem{Definition}{Definition}
\newtheorem{Theorem}{Theorem}
\newtheorem{Proposition}{Proposition}
\newtheorem{Corollary}{Corollary}
\newtheorem{Problem}{Problem}
\newtheorem{Lemma}{Lemma}
 
\newcommand{\RefSec}[1]{Section~\ref{#1}}
\newcommand{\RefFig}[1]{Fig.~\ref{#1}}
\newcommand{\RefTab}[1]{Tab.~\ref{#1}}
\newcommand{\RefDef}[1]{Definition~\ref{#1}}
\newcommand{\RefPro}[1]{Proposition~\ref{#1}}
\newcommand{\RefLem}[1]{Lemma~\ref{#1}}
\newcommand{\RefThe}[1]{Theorem~\ref{#1}}
\newcommand{\Eq}[1]{(\ref{#1})}

% LINES
%-----------
\newcommand{\HRule}{\rule{\linewidth}{0.5mm}} % Defines a new command for the horizontal lines, change thickness here

%----------------------------------------------------------------------------------------
%	TITLE PAGE
%----------------------------------------------------------------------------------------

\begin{titlepage}

\center % Center everything on the page

\textsc{\LARGE Laboratoire d'InfoRmatique en Image et Syst\`{e}mes d'information}\\[1.5cm] % Name of your university/college
%\textsc{\Large }\\[0.5cm] % Major heading such as course name
\textsc{\large Rapport de stage}\\[0.5cm] % Minor heading such as course title

\HRule \\[0.4cm]
{ \huge \bfseries Structure du bord des disques discrets}\\[0.4cm] % Title of your document
\HRule \\[1.5cm]

\begin{minipage}{0.4\textwidth}
\begin{flushleft} \large
\emph{Auteur:}\\
Thomas \textsc{Lafond} % Your name
\end{flushleft}
\end{minipage}
~
\begin{minipage}{0.4\textwidth}
\begin{flushright} \large
\emph{Encadrant:} \\
Tristan \textsc{Roussillon} % Supervisor's Name
\end{flushright}
\end{minipage}\\[4cm]

{\large 3 Juillet 2013}\\[3cm] % Date, change the \today to a set date if you want to be precise

%\includegraphics{Logo}\\[1cm] % Include a department/university logo - this will require the graphicx package

\vfill % Fill the rest of the page with whitespace

\end{titlepage}

%----------------------------------------------------------------------------------------
%	TABLE OF CONTENTS
%----------------------------------------------------------------------------------------

\tableofcontents % Include a table of contents
\newpage % Première section sur une nouvelle page. 

%------------------------------------------------
\section{Introduction}
%------------------------------------------------

%-----------------------------------------------------------------
\subsection{Contexte pédagogique}
%-----------------------------------------------------------------

% Parcours - ok
Actuellement en deuxième année du master professionnel de Mathématiques : Statistiques, Informatique et Techniques Numériques (SITN) à l’Université Claude Bernard - Lyon 1, je me dois de réaliser un stage dans le but de valider ma formation et d'ainsi obtenir mon diplôme. \newline

%labo -ok
Je suis actuellement accueilli pour une durée de 6 mois au LIRIS - Laboratoire d'InfoRmatique en Image et Systèmes d'information - UMR 5205 CNRS. Le LIRIS est issue de la fusion de plusieurs pôles de recherche de la région Lyonnaise. Aujourd'hui composé d'environ 300 personnes, il participe activement à la recherche et à l'éducation à travers deux grands départements thématiques : "Image" et "Données, Connaissances, Services".\newline

%encadrant
Je suis encadré par Tristan Roussillon, Maitre de conférence à l'Insa - Institut National des Sciences Appliquées et membre de l'équipe M2Disco - Modèles Multirésolution, Discrets et Combinatoires. Cette équipe est une composante du département Image du LIRIS qui traite de sujets comme l'analyse d'images, l'optimisation, la programmation par contraintes et la géométrie discrète. C'est d'ailleurs sur cette dernière branche que le contexte de mon stage c'est déroulé.



%-----------------------------------------------------------------
\subsection{Contexte scientifique}
%-----------------------------------------------------------------

J'ai découvert pour l’occasion la géométrie discrète, plus fréquemment appelé digital geometry dans la langue de Shakespeare. Cette discipline de recherche gravite à l'intersection de bien des domaines mathématiques et informatiques.\\

%(pics de digital geometry) % traduire ?

\begin{figure}[h!]
  \centering
  \scalebox{0.7}
  {
    \begin{tikzpicture}[
      root concept/.append style={concept color=blue!20},
      level 1 concept/.append style={sibling angle=45},mindmap]
      \node [concept] { Digital Geometry} [clockwise from=90]
      child { node[concept] { Computational Geometry}} 
      child { node[concept] { Topology}} 
      child { node[concept] { Complexity analysis and Algorithmic}}
      child { node[concept] { Combinatorics}}
      child { node[concept] { Arithmetic}}
      child { node[concept] {Number Theory}} 
      child { node[concept] {Computer Graphics}} 
      child { node[concept] {Image Processing}};
    \end{tikzpicture}

  }
  \caption{Digital Geometry - Crédit : David Coeurjolly}
   
\end{figure}
Il s'agit principalement d'étudier la géométrie et la topologie de formes, d'objets portés sur des structures régulières. Dans le cadre de cet stage, nous intéresserons plus particulièrement au disque discret sur la grille $\mathbb{Z}^{2}$. 


%------------------------------------------------
\section{Motivations}
%------------------------------------------------

%-----------------------------------------------------------------
\subsection{Le bord du disque discret}
%-----------------------------------------------------------------

%-----------------------------------------------------------------
\subsubsection{Du disque Euclidien au disque discret}

Avant de s'attarder sur la représentation d'un disque discret, il convient de revenir un court instant au disque fermé Euclidien et à l'ensemble reprenant tous les points de son bord, le cercle Euclidien.

\begin{proof}[Disque fermé Euclidien]
  $$\mathcal{D}_e =  \left\{ (x,y) \in \mathbb{R}^{2} |  (x - u)^2 + (y - v)^2 \leq R^2. \right\}$$
\end{proof}

\begin{proof}[Cercle Euclidien]
  $$\mathcal{C}_e =  \left\{ (x,y) \in \mathbb{R}^{2} |  (x - u)^2 + (y - v)^2 = R^2. \right\}$$
\end{proof}

avec $(u,v) \in \mathbb{R}^{2}$ les coordonnées du centre et $R \in \mathbb{R}^{+*}$ le rayon.\\

PICS et PICS de disque et de cercle euclidien\\

Si la définition du disque fermé discret dans $\mathbb{R}^{2}$ (appelé pour la suite du rapport uniquement disque discret) reste très proche de la définition Euclidienne, l'équivalent discret pour récupérer tous les points du bord quant à elle change considérablement.

\begin{proof}[Disque discret]
  $$\mathcal{D} =  \left\{ (x,y) \in \mathbb{Z}^{2} |  ax + by + c(x^2 + y^2) + d \leq 0. \right\}$$
\end{proof}

avec $(u,v) \in \mathbb{R}^{2}$ les coordonnées du centre et $R \in \mathbb{R}^{+*}$ le rayon.\\

PICS de disque discret\\

L'ensemble des points appartenant à $\mathbb{Z}^{2}$ représentés en rouge sur le dessin et étant positionnés exactement sur le cercle euclidien ne suffisent pas à représenté l'ensemble des points du cercle discret. Il faut donc prendre une définition plus générale.

\begin{proof}[Cercle discret]
  $$ \mathcal{C} =  \left\{ (x,y) \in \mathbb{Z}^{2} | \exists (i,j) \in \{0,1\}^2, (i,j) \ne (0,0) | (x+i,y+j) \notin \mathcal{D} \right\}$$
\end{proof}

PICS de cercle discret\\

%-----------------------------------------------------------------
\subsubsection{L'étude des points du bord}

En continuant l'investigation de la définition précédente du cercle discret, il est possible de formuler plusieurs remarques. De part le fait de travailler sur $\mathbb{Z}^{2}$, on remarque que les points sont ordonnées sur une grille régulière. Pour un point donnée, il est alors possible d'atteindre quatre voisins en suivant les quatre directions cardinales d'une unité. En éffet, chaque point possède un voisin, en haut, à gauche, en bas et à droite de lui. \\

PICS de réseau 4-connexes\\

Or, on observe que les points strictement à l'intérieur du disque discret (et donc pas sur le bord) possède tous leurs quatre voisins également à l'intérieur du disque. Pour autant, les points du bords du disque discret possèdent eux une distribution différente. Chacun des points du bord possède entre un et trois voisins à l'extérieur du disque et le complémentaire à quatre à l'intérieur.\\

PICS de points du bord sans 4 voisins in\\

Pour définir un disque discret, on se retrouve finalement avec deux ensembles. Le premier représente les points strictement à l'intérieur. Ils sont ordonnées avec exactement quatre voisins également à l'intérieur du disque. Le deuxième ensemble égale au cercle discret propose une distribution de points bien moins régulière. Finalement, seule l'étude des points du bord nous semble pertinente pour comprendre l'organisation et la structure des disque discret.

%-----------------------------------------------------------------
\subsection{Alpha-Shape}
%-----------------------------------------------------------------



%------------------------------------------------
\section{Méthodes de calculs existantes}
%------------------------------------------------

%-----------------------------------------------------------------
\subsection{Suivi de Bord}
%-----------------------------------------------------------------

La méthode du suivi du bord d'un disque discret est un processus étudié et connu. Elle est sensiblement la même suivant le cas souhaité : 4-connexes ou 8-connexes. Elle se décompose en deux étapes. Trouver un point sur le bord, puis chercher le point suivant de manière répétée jusqu'à retrouver le premier point et refermer le bord.

%-----------------------------------------------------------------
%\subsubsection{Trouver le sommet de départ}

Pour chercher un point de départ sur le bord du disque discret à partir de la seule connaissance de ses paramètres, nous avons choisi de récupérer le point d'ordonnée minimale et d'abscisse maximale à l'intérieur du disque. Nous prenons le point de $\mathbb{Z}^2$ avec les coordonnées entières du centre du disque. Puis, nous descendons le long de l'axe vertical d'une longueur entière égale au rayon pour trouver un point d'ordonnée minimale. Ensuite, nous translatons suivant l'axe horizontal afin de récupérer le point d'abscisse maximal. On note \textbf{a} ce point de départ. Cette procédure prend un temps constant. 

\begin{figure}[H]
  \centering
  \includegraphics[width=0.3\linewidth,page=1]{fig/4-exi/suivi/exi-depart-0.pdf}
  \includegraphics[width=0.3\linewidth,page=1]{fig/4-exi/suivi/exi-depart-1.pdf}
  \caption{Recherche de $a$.}
\end{figure}
  

%-----------------------------------------------------------------
%\subsubsection{Trouver le sommet suivant}

La deuxième étape consiste à trouver le sommet suivant appartenant au bord. On répète cette étape jusqu'à retrouver le point a.

A partir d'un point quelconque du bord $p$ ($a$ au début), le point suivant est choisi suivant la position des 4 voisins (resp 8 voisins) de $p$ par rapport au disque et d'un sens arbitraire de rotation. En tournant dans le sens trigonométrique et en partant d'un voisin situé à l'extérieur du disque, le point suivant est le premier voisin situé à l'intérieur du disque. 

\begin{figure}[H]
  \centering
  \includegraphics[width=0.4\linewidth,page=1]{fig/4-exi/suivi/exi-suivi-0.pdf}
  \caption{Suivi de bord 4-connexes et 8-connexes}
\end{figure}
  
Cette étape prend un temps constant, limité par la taille constante du voisinage. Elle est répétée autant de fois qu'il y a de points sur le bord du disque. Rappelons que $R$ est le rayon du disque. Comme il y a $O(R)$ points sur le bord du disque, la complexité en temps du suivi est en $O(R)$. 

%-----------------------------------------------------------------
\subsection{Enveloppe Convexe : Algorithme de Har-Peled}
%-----------------------------------------------------------------

De nombreux algorithmes existent pour calculer l'enveloppe convexe d'un ensemble de points. L'algorithme de Graham \cite{Graham1972} l'implémente en $O(n \log n)$ pour un ensemble quelconque de $n$ points. Quand les $n$ points sont ordonnés, comme le sont les points du bord d'un disque discret, le parcours de Graham est en $O(n)$. Par conséquent, calculer l'enveloppe convexe des points d'un disque discret se calcule par suivi de bord et parcours de Graham en $O(R)$. Cependant un algorithme géométrique introduit par Har-Peled en 1998 \cite{HarPeled98} calcule l'enveloppe convexe des points d'un disque discret de manière incrémentale et ``output-sensitive''.
%% reference pour Graham
%% reference pour Har-Peled

\begin{Definition}{Output sensitive}\\
\label{def:os}
      Un algorithme output sensitive possède un temps d’exécution qui dépend de la taille de sa sortie.
\end{Definition}

La méthode de Har-Peled dépend du nombre de sommets de l'enveloppe convexe. Elle construit successivement les arêtes du polygone à l'aide des convergents qui représente le pendant géométrique du calcul du pgcd de deux nombres entiers (voir l'annexe \ref{annexe-euc-geo}). La complexité en temps de cet algorithme pour un disque de rayon $R$ relève d'une part de la recherche du prochain sommet en $O(\log R)$ et également du nombre de sommets qui est $O(R^{2/3})$. Soit une complexité totale en temps de $O( R^{2/3} \log R)$.

%-----------------------------------------------------------------
\subsubsection{Calcul des convergents}

Cette méthode de calcul est géométrique. Soient l’origine $O=(0,0)$ et $P = (P_x, P_y)$ un point à coordonnées entières. Nous cherchons le premier point de $\mathbb{Z}^{2}$ appartenant au segment de droite [O,P]. Le coefficient trouvé correspond au pgcd de $P_x$ et $P_y$.\\

Soient $p_{-2} = (1,0)$ et $p_{-1} = (0,1)$ les deux premiers convergents. Pour trouver les convergents suivants, nous mettons en place une méthode récursive :

$$p_{k} = p_{k-2} + q_k p_{k-1}$$

où $q_k$ est le plus grand entier tel que $p_{k}$ et $p_{k-2}$ soient du même côté de la droite.\\

L'opération correspond à jeter un rayon de $p_{k-2}$ dans la direction de $p_{k-1}$ pour étudier l’intersection du vecteur et du segment de droite de direction $y_P / x_P$. La méthode s’arrête quand un convergent $p_{k}$ est exactement sur la droite.\\

\begin{figure}[H]
  \centering
  \includegraphics[width=0.4\linewidth]{fig/4-exi/har/exi-har-0.pdf}
  \includegraphics[width=0.4\linewidth]{fig/4-exi/har/exi-har-1.pdf}
  \caption{Calcul des convergents du point (3,8)}
\end{figure}


%-----------------------------------------------------------------
\subsubsection{Passage au disque}


La méthode de calcul de l'enveloppe convexe d'un disque se décompose en deux étapes. La première consiste à trouver un point de départ. Nous appliquons la même procédure que lors du suivi de bord pour récupérer le point de départ d'ordonnée minimale et d'abscisse maximale. Par définition ce point appartient à l'enveloppe convexe.\\

Ensuite, nous cherchons le sommet suivant jusqu'à retrouver le point de départ. Cette étape est réalisée en calculant les convergents les plus proches du bord du disque. Nous allons alternativement être à l'intérieur du disque lorsque $k$ est impair et à l'extérieur du disque lorsque $k$ est pair. Nous repartons d'un convergent s'il se situe exactement sur le bord du disque. Sinon, nous repartons du dernier convergent de degré impair lorsque que le lancer de rayon n'est pas effectif.\\
%% effectif: pas clair 
%% liens avec la figure ?

\begin{figure}[H]
  \centering
  \includegraphics[width=0.4\linewidth]{fig/4-exi/har/exi-har-10.pdf}
  \includegraphics[width=0.4\linewidth]{fig/4-exi/har/exi-har-11.pdf}
  \caption{Calcul de l'enveloppe convexe d'un disque}
\end{figure}

\subsubsection{Résultats}

Les résultats suivants ont été obtenus en calculant sur une moyenne de 100 disques de rayon $2^k$ possédant un centre à coordonnée rationnel compris dans $[0,1]\times[0,1]$. Afin de vérifier la convergence en $O(R^{2/3})$, nous avons également récupéré la moyenne par rayon de la division du nombre de sommets de l'enveloppe convexe sur le rayon à la puissance 2/3. La zone bleue de la figure correspond à l'intervale entre le minimum et le maximum obtenu.

On s'intéresse également à vérifié la complexité en temps de notre algorithme. On le compare notamment à la marche de Graham. 

\begin{figure}[H]
  \centering
  \includegraphics[width=\linewidth]{fig/4-exi/ch/exi-ch-sommet.png}
  \caption{Sommets et bord de l'enveloppe convexe}
\end{figure}

\begin{table}[H]
  \begin{tabular}{|p{0.09\linewidth}|p{0.13\linewidth}||p{0.2\linewidth}|p{0.13\linewidth}||p{0.2\linewidth}|p{0.13\linewidth}|}
    \hline
    \multicolumn{2}{|c||}{Rayon} & \multicolumn{4}{c|}{Enveloppe convexe} \\  \hline 
    $R=2^k$  &  & \multicolumn{2}{c||}{Nombre de Sommets} &  \multicolumn{2}{c|}{Nombre de points sur le bord} \\ \hline 
    k & R &   & $\# / R^{2/3}$  &   & $\# / R^{2/3}$ \\    
    \hline
    5 & 32         & 35,36     & 3,51 & 102,05   &  10,12\\
    6 & 64         & 55,78     & 3,49 & 170,16   &  10,64\\
    7 & 128        & 87,78     & 3,46 & 283,69   &  11,17\\
    8 & 256        & 139,71    & 3,47 & 465,06   &  11,53\\
    9 & 512        & 222,07    & 3,47 & 761,01   &  11,89\\
    10 & 1024      & 351,72    & 3,46 & 1,24E+03 &  12,21\\
    11 & 2048      & 558,18    & 3,46 & 2,01E+03 &  12,45\\
    12 & 4096      & 883,86    & 3,45 & 3,24E+03 &  12,68\\
    13 & 8192      & 1,40E+003 & 3,45 & 5,25E+03 &  12,92\\
    14 & 16384     & 2,23E+003 & 3,45 & 8,41E+03 &  13,03\\
    15 & 32768     & 3,54E+003 & 3,45 & 1,35E+04 &  13,19\\
    16 & 65536     & 5,62E+003 & 3,46 & 2,16E+04 &  13,28\\
    17 & 131072    & 8,91E+003 & 3,45 & 3,47E+04 &  13,45\\
    18 & 262144    & 1,41E+004 & 3,45 & 5,54E+04 &  13,53\\
    19 & 524288    & 2,25E+004 & 3,45 & 8,87E+04 &  13,64\\
    20 & 1048576   & 3,56E+004 & 3,45 & 1,42E+05 &  13,75\\
    21 & 2097152   & 5,66E+004 & 3,45 & 2,26E+05 &  13,81\\
    22 & 4194304   & 8,98E+004 & 3,45 & 3,61E+05 &  13,88\\
    23 & 8388608   & 1,43E+005 & 3,45 & 5,76E+05 &  13,94\\
    24 & 16777216  & 2,26E+005 & 3,45 & 9,19E+05 &  14,02\\
    25 & 33554432  & 3,59E+005 & 3,45 & 1,46E+06 &  14,07\\
    26 & 67108864  & 5,70E+005 & 3,45 & 2,33E+06 &  14,10\\
    27 & 134217728 & 8,98E+005 & 3,42 & 3,74E+06 &  14,27\\
    28 & 268435456 & 1,35E+06  & 3,24 & 6,62E+06 &  15,90\\
    \hline
  \end{tabular} 
  \caption{Sommet et bord de l'enveloppe convexe}
\end{table}

\begin{figure}[H]
  \centering
  \includegraphics[width=\linewidth]{fig/4-exi/ch/exi-ch-temps.png}
  \caption{Temps de calcul de l'enveloppe convexe (échelle log / log)}
\end{figure}

%%TRI: non pertinent
%% \begin{table}[H]
%%   \begin{tabular}{|p{0.09\linewidth}|p{0.13\linewidth}||p{0.23\linewidth}|p{0.23\linewidth}|p{0.23\linewidth}|}
%%     \hline
%%     \multicolumn{2}{|c||}{Rayon} & \multicolumn{3}{c|}{Temps de calcul (ms) } \\  \hline 
%%     $R=2^k$  &  &  \multicolumn{3}{c|}{Enveloppe Convexe}  \\ \hline
%%     k & R & Marche de Grahaam & Har-Peled - Sommets & Har-Peled - Bord \\
%%     \hline
%%     5  & 32        & 0,64     & 1,05     & 1,98\\
%%     6  & 64        & 1,18     & 1,87     & 3,77\\
%%     7  & 128       & 2,33     & 3,3      & 7,08\\
%%     8  & 256       & 4,64     & 5,86     & 13,02\\
%%     9  & 512       & 9,25     & 10,29    & 23,99\\
%%     10 & 1024      & 32,5     & 17,9     & 43,34\\
%%     11 & 2048      & 37,03    & 31,13    & 78,05\\
%%     12 & 4096      & 74,55    & 53,75    & 139,12\\
%%     13 & 8192      & 149,03   & 92,57    & 247,36\\
%%     14 & 16384     & 297,79   & 159,44   & 433,88\\
%%     15 & 32768     & 596,22   & 272,74   & 760,20\\
%%     16 & 65536     & 1,19E+03 & 466,51   & 1,32E+03\\
%%     17 & 131072    & 2,37E+03 & 795,09   & 2,30E+03\\
%%     18 & 262144    & 4,73E+03 & 1,35E+03 & 3,99E+03\\
%%     19 & 524288    & 9,47E+03 & 2,29E+03 & 6,88E+03\\
%%     20 & 1048576   & 1,89E+04 & 3,88E+03 & 1,18E+04\\
%%     21 & 2097152   & 3,79E+04 & 6,58E+03 & 2,02E+04\\
%%     22 & 4194304   & 7,58E+04 & 1,11E+04 & 3,45E+04\\
%%     23 & 8388608   & 1,52E+05 & 1,86E+04 & 5,88E+04\\
%%     24 & 16777216  & 3,03E+05 & 3,13E+04 & 1,00E+05\\
%%     25 & 33554432  & 6.04E+05 & 5,26E+04 & 1,70E+05\\
%%     26 & 67108864  & 1.20E+06 & 8,80E+04 & 2,87E+05\\
%%     27 & 134217728 & 2.42E+06 & 1,46E+05 & 4,82E+05\\
%%     28 & 268435456 &          & 2,33E+05 & 8,85E+05\\
%%     \hline
%%   \end{tabular} 
%%   \caption{Temps de calcul de l'enveloppe convexe}
%% \end{table}
%%

Nos résultats obtenus sont conformes à ceux de la publication \cite{HarPeled98}. On remarque que la moyenne asymptotique de la division du nombre moyen de sommets de l'enveloppe convexe sur le rayon à la puissance 2/3 est 3,45. Des anomalies commencent à apparaître pour des rayons de la taille de $2^{27} = 134217728$. Il convient de chercher à comprendre d'où elles viennent afin de mieux cerner les possibles limitations de notre algorithme.\\

Le graphique représentant les temps est également intéressant. On observe avec l'échelle logarithmique que la complexité en temps est sous-linéaire pour la méthode de Har-Peled. La méthode devient d'ailleurs plus intéressante que la marche de Grahaam en terme de temps de calcul assez rapidement à partir d'un rayon $2^{10} = 1024$ unités. 
% alors que la méthode qui récupère également les points sur les sommets devient plus rapide qu'à partir de $2^{17} = 131072$.
%% TRI: non pertinent, enlevé la courbe de temps avec les points sur le bord





%------------------------------------------------
\section{Contributions}
%------------------------------------------------

% Intro à refaire 
%-----------------------------------------------------------------
\subsection{Relations aux triangulations de Delaunay}
%-----------------------------------------------------------------


L'enveloppe convexe est unique. Les algorithmes implantés fournissent les sommets dans l'ordre trigonométrique. Il est possible de reconstruire le polygone convexe et d'en déduire ses arêtes. Elles se décomposent en motif de droites discrètes. Un motif de droite discrète est inclue dans les segments de droites discrètes. La triangulation de Delaunay de ces motifs est connue. \cite{RoussillonL11}\\

\begin{figure}[H]
  \centering
  \includegraphics[width=0.5\linewidth]{fig/5-con/tri/con-motif-0.pdf}
  \caption{Triangulation de Delaunay d'un motif de droite discrète}
\end{figure}

La triangulation de Delaunay dépend des convergents. Le dernier convergent n'appartenant pas au segment de droite discrète est le sommet du triangle incident de la triangulation de Delaunay. Il est appelé le point de bézout. À l'aide d'un calcul récursif, il est possible de construire l'intégralité de la triangulation de Delaunay.

\begin{figure}[H]
  \centering
  \includegraphics[width=0.5\linewidth]{fig/5-con/tri/con-conv-0.pdf}
  \caption{Calcul de convergents et triangulations de Delaunauy}
\end{figure}

Comme les convergents sont calculés dans l’algorithme de Har-Peled, il est probable de pouvoir construire l'$\alpha$-shape directement. 

%-----------------------------------------------------------------
\subsection{$\alpha$-shape, $\alpha \leq 0$ - Généralisation de Har-Peled}
%-----------------------------------------------------------------

%-----------------------------------------------------------------
\subsubsection{Construction de l'algorithme}

La méthode de calcul de l'$\alpha$-shape pour $\alpha <0$ reproduit le schéma du calcul de l’enveloppe convexe. Il y a cependant une étape supplémentaire pour chaque convergent à l'intérieur du disque afin de contrôler la possibilité d'avoir construit une arête de l'$\alpha$-shape.

L'algorithme commence similairement à l'enveloppe convexe avec la recherche d'un point de départ. La même méthode est utilisée pour trouver le point $a$ d'ordonnée minimale et d'abscisse maximale. Comme il appartient à l'enveloppe convexe, il appartient également à l'$\alpha$-shape.

À partir de ce point, nous lançons une série de convergents pour récupérer les sommets $e$ potentiels. Les convergents se trouvent alternativement à l'intérieur de disque (convergent de degré impaire) et à l'extérieur où exactement sur le bord du disque (convergent de degré pair). À chaque convergent à l'intérieur (de couleur bleue foncé), on contrôle la possibilité d'avoir trouvé un sommet.

Soient  $b = p_{k-2} + (q_k - 1) * p_{k-1}$ et $c = p_k = p_{k-2} + q_k * p_{k-1}$. Pour savoir si c est un sommet, nous utilisons un prédicat qui compare la taille du rayon \textbf{$R_T$} du cercle circonscrit au triangle : $T(a, b, c)$ à la taille du rayon de notre disque généralisé : \textbf{$R_{\alpha}$} $= -1/\alpha$. Il faut distinguer deux cas de figures.\\

\begin{figure}[H]
  \centering
  \includegraphics[width=0.45\linewidth]{fig/5-con/nas/con-nas-0.pdf}
  \includegraphics[width=0.45\linewidth]{fig/5-con/nas/con-nas-1.pdf}
  \caption{Calcul des convergents et du Prédicat}
\end{figure}

Si $\alpha = \alpha_{2}$ (en jaune) alors \textbf{$R_{\alpha_{2}} < R_T$} et le point b appartient au notre disque généralisé de rayon $-1/R_{\alpha_{2}}$ ( b appartient au complémentaire du disque de rayon $1/R_{\alpha_{2}}$). On ne sait pas encore si le convergent c sera un sommet de l'$\alpha$-shape, mais on sait que b n'en sera pas un. Nous pouvons continuer le calcul des convergents.\\ 

Si $\alpha = \alpha_{1}$ (en violet) alors \textbf{$R_{\alpha_{1}} > R_T$} et le point b n'appartient pas à notre disque généralisé de rayon $-1/R_{\alpha_{1}}$. L'$\alpha$-hull ne peut rejoindre c par a sans au moins passé par b. b et c appartiennent à notre $\alpha$-shape. Il faut maintenant vérifier si les points $b_i = p_{k-2} + i*p_{k-1} \forall i \in [0, q_k-2]$ appartiennent également à l'$\alpha$-shape. \\

De part la construction des triangles, la taille des rayons de leur cercle circonscrit $R_{T_{i}}$ est croissante. Il suffit de tester le dernier avec le plus grand cercle pour savoir si nous pouvons continuer notre algorithme et lancer les convergents suivants ou si nous devons déterminer quel sera le point $\alpha$-extrême par une recherche dichotomique. La recherche dichotomique permet de trouver en $log(q_k)$ le triangle adéquate $T_i = (a, b_{i}, b_{i+1}$ tel que $R_{T_i} > R_{\alpha}$ et $R_{T_{i-1}} \leq R_{\alpha}$.

\begin{figure}[H]
  \centering
  \includegraphics[trim = 1.2cm 1.2cm 1.2cm 0.6cm, clip,width=\linewidth]{fig/5-con/nas/con-nas-dicho.pdf}
  \caption{Taille croissante des rayons des cerlces circonscrits au triangle.}
\end{figure}

Le triangle renvoyé par la méthode dichotomique assure que l'ensemble des points $\left\{ b_{i},\ldots, b_{q_k}, c \right\}$ appartiennent à l'$\alpha$-shape. Nous continuons la méthode en repartant du sommet c.
 
\begin{figure}[H]
  \centering
  \includegraphics[trim = 1.2cm 1.2cm 1.2cm 0.6cm, clip,width=\linewidth]{fig/5-con/nas/con-nas-2.pdf}
  \caption{Nouveaux points et sommets de l'$\alpha$-shape.}
\end{figure}


%-----------------------------------------------------------------
\subsubsection{Résultats}

Les processus de création des disques $\mathcal{D}$ est le même que lors du calcul de l'enveloppe convexe. Néanmoins, pour chaque disque il est possible de tester de nombreuses valeurs de $\alpha$ correspondant au différentes $\alpha$-shapes souhaitées. Pour ce tableau, nous avons voulu vérifier si la propriété de complexité du cas particulier de l'enveloppe convexe pour $\alpha = 0$ se généralisait pour certaine valeur de $\alpha$. Nous avons donc décidé de prendre pour l'ensemble des disques un $\alpha$ proportionnel à l'inverse du rayon avec $\alpha = -\frac{k}{R_{\mathcal{D}}}$ de tel sorte que $R_{\alpha} = -\frac{R_{\mathcal{D}}}{k}$.
 

\begin{figure}[H]
  \centering
  \includegraphics[width=\linewidth]{fig/5-con/nas/con-res-nas.png}
  \caption{Nombre de sommets de l'$\alpha$-shape en fonction de la taille des rayons. (Échelle log)}
\end{figure}


\begin{table}[H]
  \begin{tabular}{|p{0.09\linewidth}|p{0.13\linewidth}||p{0.23\linewidth}||p{0.23\linewidth}|p{0.23\linewidth}|}
    \hline
    \multicolumn{2}{|c||}{Rayon} & prédicat               & \multicolumn{2}{|c|}{$\alpha-shape$} \\  \hline 
    $R=2^k$  &                   & $-\alpha = R/33$ & \multicolumn{2}{|c|}{Nombre de sommets} \\ \hline
    k        & R                 &                        & \# & $\# / R^{2/3}$ \\ 
    \hline
    5  & 32        & 3,2        & 179,02   & 17,7610\\
    6  & 64        & 6,4        & 272,92   & 17,0575\\
    7  & 128       & 12,8       & 472,19   & 18,5913\\
    8  & 256       & 25,6       & 774,45   & 19,2088\\
    9  & 512       & 51,2       & 1,30E+03 & 20,3259\\
    10 & 1024      & 102,4      & 2,14E+03 & 21,0878\\
    11 & 2048      & 204,8      & 3,54E+03 & 21,9549\\
    12 & 4096      & 409,6      & 5,68E+03 & 22,1878\\
    13 & 8192      & 819,2      & 9,25E+03 & 22,7644\\
    14 & 16384     & 1638,4     & 1,49E+04 & 23,0413\\
    15 & 32768     & 3276,8     & 2,38E+04 & 23,2816\\
    16 & 65536     & 6553,6     & 3,84E+04 & 23,6175\\
    17 & 131072    & 13107,2    & 6,17E+04 & 23,9124\\
    18 & 262144    & 26214,4    & 9,89E+04 & 24,1500\\
    19 & 524288    & 52428,8    & 1,59E+05 & 24,4137\\
    20 & 1048576   & 104857,6   & 2,54E+05 & 24,5914\\
    21 & 2097152   & 209715,2   & 4,06E+05 & 24,7603\\
    22 & 4194304   & 419430,4   & 6,49E+05 & 24,9402\\
    23 & 8388608   & 838860,8   & 1,04E+06 & 25,0730\\
    24 & 16777216  & 1677721,6  & 1,65E+06 & 25,2002\\
    25 & 33554432  & 3355443,2  & 2,63E+06 & 25,3061\\
    26 & 67108864  & 6710886,4  &    &  \\
    27 & 134217728 & 13421772,8 &    &  \\
    28 & 268435456 & 26843545,6 &    &  \\

    \hline
  \end{tabular} 
  \caption{Nombre de sommets de l'$\alpha$-shape}
\end{table}

%-----------------------------------------------------------------
\subsection{$\alpha$-shape, $\alpha \geq 0$}
%-----------------------------------------------------------------

%-----------------------------------------------------------------
\subsubsection{Construction de l'algorithme}

La méthode de calcul de l'$\alpha$-shape pour $\alpha > 0$ est fondamentalement différentes des méthodes de calculs présentées précédemment. La méthode n'est plus incrémentale mais de type ``Bottom-Up''. L'algorithme part du cas critique de l'enveloppe convexe. L'ensemble des sommets de l'$\alpha$-shape est un sous-ensemble de l'enveloppe convexe. 

L'algorithme commence différemment des autres déjà présentés. Le départ est en deux phases. Il faut d'une part récupérer un point de départ pertinent mais également l'ensemble des sommets de l'enveloppe convexe. Le point d'ordonnée minimale et d'abscisse maximale n'est plus nécessairement dans l'$\alpha$-shape. Un mauvais point de départ engendrerait un calcul potentiellement faux. Pour être certain de notre départ, nous ne le calculons pas mais le choisissons par construction parmi l'un des trois sommets du triangle composant le cercle circonscrit. Ensuite, nous récupérons à partir de ce point l'ensemble des sommets de l'enveloppe convexe du disque discret. On note $\mathcal{S}$ cet ensemble.

\begin{figure}[H]
  \centering
  \includegraphics[width=0.5\linewidth]{fig/5-con/pas/con-pas-0.pdf}
  \caption{Le cercle circonscrit au triangle et l'enveloppe convexe de son disque.}
\end{figure}


Notons a,b et c trois sommets successifs appartenant à S. Supposant que a appartient à l'$\alpha$-shape, nous allons vérifier la présence de b et c dans l'$\alpha$-shape. Nous utilisons un prédicat qui compare la taille du rayon \textbf{$R_T$} du cercle circonscrit au triangle : $T(a, b, c)$ à la taille du rayon de notre disque généralisé : \textbf{$R_{\alpha}$} $= 1/\alpha$. Il faut distinguer deux cas de figures.\\

\begin{figure}[H]
  \centering
  \includegraphics[width=0.6\linewidth]{fig/5-con/nas/con-nas-1.pdf}
  \caption{Calcul du Prédicat}
\end{figure}

Si $\alpha = \alpha_{1}$ alors \textbf{$R_{\alpha_{1}} < R_T$} et le point b appartient au notre disque généralisé de rayon $-1/R_{\alpha_{1}}$. Il est possible de rejoindre c par a à l'aide d'un arc de cercle dont le disque comprend l'ensemble des points de $\mathcal{D}$. b appartient à l'$\alpha$-hull, mais n'appartient pas à l'$\alpha$-shape. On continue la procédure pour étudier les sommets suivants. a reste inchangé, b devient c et c devient le sommet suivant de $\mathcal{S}$.

\begin{figure}[H]
  \centering
  \includegraphics[width=0.4\linewidth]{fig/5-con/pas/con-pas-1.pdf}
  \includegraphics[width=0.4\linewidth]{fig/5-con/pas/con-pas-2.pdf}
  \caption{Calcul du Prédicat}
\end{figure}

Si $\alpha = \alpha_{2}$ alors \textbf{$R_{\alpha_{2}} > R_T$} et le point b n'appartient pas à notre disque généralisé de rayon $1/R_{\alpha_{1}}$. L'$\alpha$-hull ne peut rejoindre c par a sans au moins passé par b. \textbf{b appartient à l'$\alpha$-shape}. Il faut poursuivre la procédure afin de réaliser un tour complet. b devient a, c devient b et c devient le sommet suivant de $\mathcal{S}$.\\

\begin{figure}[H]
  \centering
  \includegraphics[width=0.4\linewidth]{fig/5-con/pas/con-pas-3.pdf}
  \includegraphics[width=0.4\linewidth]{fig/5-con/pas/con-pas-4.pdf}
  \caption{Calcul du Prédicat}
\end{figure}

%-----------------------------------------------------------------
\subsubsection{Compléxité et perspective}

La version présente de l'algorithme s'appuie sur l'enveloppe convexe selon la méthode de Har-Peled qui récupère les sommets en $O(R^{2/3}_{D})$. On parcourt cette liste une seule fois en éliminant des sommets potentiel ou en ajoutant le sommet à l'alpha-shape. 

Une perspective de travail serait d'adopter une approche "top-down" qui partant d'un point, ajouterait incrémentalement les sommets de l'alpha-shape. 


%------------------------------------------------
\section{Développement Informatiques}
%------------------------------------------------

%-----------------------------------------------------------------
\subsection{Programmation}
%-----------------------------------------------------------------

%-----------------------------------------------------------------
\subsubsection{C++}


Le C++ est un langage de programmation née dans les années 1980 dans l'optique d'agrémenter le langage C de nouvelles fonctionnalités. Il fut d'abord nommée par son créateur Bjarne Stroustrup : C with Classes. L’appellation c++, rappelant l'opération d'incrémentation fut adopté peu de temps plus tard à partir de 1983 suite à l'ajout de nouvelles fonctionnalités. (Source \cite{Wiki-cpp})

Aujourd'hui encore, de nombreuses fonctionnalités viennent agrémenter le langage C++ au fil des spécifications. Il est désormais possible de l'utiliser en s'appuyant sur de multiples paradigmes comme la programmation procédurale, la programmation orientée objet et la programmation générique.

Le paradigme choisit dans ce stage est celui de la programmation générique. \cite{troussil-cpp}

%-----------------------------------------------------------------
\subsubsection{Programmation générique}


Le paradigme de la programmation générique s'appuie sur des relations concepts-modèles. 

Pour appartenir à un même concept, les objets doivent posséder les mêmes fonctionnalités et le même comportement. 

\begin{Definition}{Concept}\\
\label{def:cpp-con}
    Un concept définit une certaine interface en terme de méthodes et de types internes. Cela rassemble des contraintes sémantiques et syntaxiques : le nom des méthodes et des types internes est fixé.
\end{Definition}

\begin{Definition}{Modèle}\\
  Un modèle est un type, un objet ou une classe qui satisfait les contraintes d'un concept.
\label{def:cpp-mod}

\end{Definition}

On parle alors de polymorphisme statique dans le sens où un algorithme, une fonction peut s'exécuter avec des types différents.

%-----------------------------------------------------------------
\subsubsection{Exemple concret}

% Points
L'espace de travail $\mathbb{Z}^{2}$ est une grille régulière représentant des points à coordonnées entières. Ce sont les principaux objets que nous allons manipuler. L'un des enjeux de la géométrie discrète est de baser ses calculs principalement sur l'usage d'entiers afin d'éviter tous les problèmes apportés par les incertitudes de précision dû aux flottants. Comme le C++ est un langage fortement typé, la représentation des entiers diffère selon la taille maximal autorisée souhaitée. On parle alors de concept "Entier". Plusieurs modèles de base sont présents : int, long. Nous utiliserons également un modèle BigInteger intégré par l'intermédiaire des librairie DGtal et GMP qui permet de manipuler de très grands entiers.


Notre classe point possède deux variables internes myX et myY de type Entier. Elle est muni de diverses opérations.

\begin{table}[H]
  \begin{tabular}{|p{0.2\linewidth}|p{0.7\linewidth}|}
    \hline
     Opérations & Fonctionalités\\ 
    \hline
    +, -               & Addition et Soutraction\\
    =                  & Affectation par un autre point\\
    +=, -=             & Addition, soustraction et affectation\\
    ==, !=             & Comparaison\\
    $[i], i = 1 \text{ ou } 2$     & Accès aux coordonnées\\
    (,)                & Affectation par les coordonnées\\
    normL1(), normL2() & Diverses normes\\
    std::cout          & Sortie standart\\
    \hline
  \end{tabular} 
  \caption{Nombre de sommets de l'$\alpha$-shape}
\end{table}

Pour mettre à bien nos méthodes de calcul sur des objets dans $\mathbb{Z}^{2}$ , nous avons implémenté un concept Forme. Il permet de répondre à trois questions.

\begin{itemize}
  \item Prédicat de position : Sommes-nous dedans, dehors ou exactement sur le bord de cette forme.
  \item Intersection de rayon : Le rayon émanent de ce point dans cette direction intersecte-t-il cette forme et si oui, quelle est le point le plus proche et du même côté.
  \item Point de départ : Il s'agit de trouver le point à l'intérieur avec l'ordonnée minimal et l'abscisse maximale. 
\end{itemize}

Pour le moment, trois modèles ont été implémentés : RayIntersectableStraightLine : Le segment de droite discrète, ExactRayIntersectableCircle : Un disque discret implémentées avec des calculs sur des entiers, InexactRayIntersectableCircle : Un disque discret implémentées avec des calculs sur des entiers et des flottants.

% graph contexte

%\newpage
%-----------------------------------------------------------------
\subsubsection{Dans la pratique}

\begin{verbatim}
  // ECircle utilise un modèle Forme de disque avec des calculs exactes et de 
  // très grands entiers
  typedef ExactRayIntersectableCircle<DGtal::BigInteger> ECircle;
    
  // On définit des disques de rayon R dont le centre est placé aléatoirement
  // dans le carré [0;1]x[0;1].
  // D : ax + by + c(x^2 + b^b) + d >= 0
    
  DGtal::BigInteger c =  -25;
  DGtal::BigInteger a = - rand() %(2*c);
  DGtal::BigInteger b = - rand() %(2*c);  
  DGtal::BigInteger d = ( a*a + b*b - 4*R*R*c*c)/(4*c);
  
  // Déclaration et affectation du disque
  ECircle exactCircle( a, b, c, d );	 
  
  // Déclaration de l'enveloppe convexe de notre disque.
  OutputSensitiveConvexHull<ECircle> exactCH(exactCircle);
  
  // Variables pour récupérer les sommets succéssifs.
  // C'est un OutputIterator.
  std::back_inserter( std::vector<Point>) exactVertices;
  
  // Récupération de ses sommets par la méthode all 
  // appartenant à la classe OutputSensitiveConvexHull.

  exactCH.all(exactVertices);  
\end{verbatim}
\begin{verbatim}
  // ICircle utilise un modèle Forme de disque avec des calculs flottants et de 
  // grands entiers 
  typedef InexactRayIntersectableCircle<long> ICircle;
    
  // On définit des disques de rayon R dont le centre est placé aléatoirement
  // dans le carré [0;1]x[0;1].
  // D : ax + by + c(x^2 + b^b) + d >= 0
    
  long c =  -25;
  long a = - rand() %(2*c);
  long b = - rand() %(2*c);  
  long d = ( a*a + b*b - 4*R*R*c*c)/(4*c);
  
  // Déclaration et affectation du disque
  ICircle inexactCircle( a, b, c, d );	  
  
  // Déclaration de l'enveloppe convexe de notre disque.
  OutputSensitiveConvexHull<ICircle> inexactCH(inexactCircle);
  
  // Variables pour récupérer les sommets succéssifs.
  // C'est un OutputIterator.
  std::back_inserter( std::vector<Point>) inexactVertices;
  
  // Récupération de ses sommets par la méthode all 
  // appartenant à la classe OutputSensitiveConvexHull.

  inexactCH.all(inexactVertices); 
\end{verbatim}


La programmation générique nous a permis d'implémenter qu'une seule fois la classe OutputSensitiveConvexHull et sa méthode all qui récupère les sommets de l'enveloppe convexe. Nous avons pu ainsi utiliser nos méthodes sur différents modèles en toute transparence. Néanmoins la réalisation d'un projet comportant du développement informatique impose certaines contraintes au niveau du rendu. 

%-----------------------------------------------------------------
\subsection{Génie logiciel}
%-----------------------------------------------------------------

%-----------------------------------------------------------------
\subsubsection{Tests}

Le but de tout algorithme est de répondre juste à la question qui lui est posée. L'une des difficultés rencontrés dans ce stage reste la faible présence d'outils susceptibles de nous répondre à notre question par d'autre chemin. Pour vérifier nos calculs, nous avons implémenté des méthodes de tests s'appuyant sur le suivi de bord pour les méthodes recherchant des bords. En codant les tests unitaires avant nos méthodes de calcul, nous avons mis en place une logique de développement par les tests utilisant le principe : Stop The line. \cite{stoptheline}

Dans le cadre d'un développement informatique, cela consiste à essayer de provoquer l'ensemble des cas critiques et de chercher à les passer. Le but étant de pouvoir utiliser le code en production en minimisant le risque de trouver un résultat incorrecte.\\


Les tests unitaires ont été générés aléatoirement sur un grand nombre de cas et se sont concentré sur certains cas particuliers critiques. Ses tests ont été automatisés avec l'utilisation de make et cmake. L'appel à la commande make test permettant de lancer tous les tests et d'ainsi vérifier l'intégration et la non-régression lors de l'ajout de nouvelles méthodes. 


%-----------------------------------------------------------------
\subsubsection{Espace de collaboration}

Le projet crée à l'occasion de ce stage a été de développer des méthodes en s'appuyant sur un projet de plus grande envergure à travers l'utilisation de la librairie DGtal : \cite{DGtal}. Afin d'être compatible, il faut accepter de suivre un certain formalisme pour homogénéiser l'ensemble. Il faut de suivre les conventions établies au niveau du nommage des variables, de la mise en place des commentaires et de la documentation. Une autre forme de formalisme à suivre se retrouve également à travers l'écriture directe du programme et de la gestion des longues lignes de codes et tout simplement des tabulations/espaces afin de pouvoir tirer au mieux profit de la puissance des gestionnaires de version, de git dans notre cas.\\

La collaboration avec mon encadrant s'est grandement appuyée sur l'utilisation quotidienne du logiciel git et de l’hébergement de notre projet sur la plateforme github : \cite{github-tristan} et \cite{github-thomas} permettant de garder facilement une trace des travaux et modification effectués. Environ 270 commits sont venues répondre à une soixantaine de problèmes soulevés.\\

Un projet informatique n'a de sens et d'utilité que s'il est complet. Il est important dans un souci de pérennité de proposer un travail fini afin que celui-ci puisse être compris et utilisé ultérieurement. Cela a été l’occasion de retravailler le projet dans son ensemble avec du recul pour par exemple nommer correctement et avec des noms cohérents l'ensemble des classes.

%-----------------------------------------------------------------
\subsubsection{Structure du dépôt}


Le dépôt est structuré en plusieurs fichiers et dossiers.

\begin{itemize}
  \item doc contient l'ensemble des présentations, rapports et algorithmes créés à l'occasion de ce stage. Les formats texte et xml ont été privégiés afin de pouvoir utiliser au mieux git.
  \item inc (10 fichiers ) contient les modèles et les méthodes développés en c++. 
  \item stests (8 fichiers ) contient l'ensemble des tests développés afin de vérifier nos modèles et méthodes.
  \item stools (3 fichiers ) contient deux outils qui récupèrent les résultats de nos méthodes.
  \item tools (2 fichiers ) contient des scripts GNUPlot pour tracer les graphiques à partir des résultats.
\end{itemize}

Afin d'utiliser au mieux notre projet il faut maintenant installer la librairie DGtal \cite{DGtal}. Les prérequis sont cmake, boost et GMP (Gnu Multiprecision Arithmetic Library). 

Afin de construire et utiliser les différents programmes créés il faut suivre la procédure suivante :

\begin{itemize}
  \item Créer un répertoire de construction : mkdir build; cd build
  \item Genererer les makefiles : cmake ..
  \item Compiler, créer les executables : make
  \item Éxecuter tous les tests : make test
  \item Éxecuter un test particulier : ./stests/test*
  \item Utiliser un outil : ./stools/tool*
\end{itemize}



%------------------------------------------------
\section{Conclusion}
%------------------------------------------------

%-----------------------------------------------------------------
\subsection{Poursuite du projet}
%-----------------------------------------------------------------

%-----------------------------------------------------------------
\subsection{Compétences acquises}
%-----------------------------------------------------------------

%-----------------------------------------------------------------
\subsection{Ouverture personnelle}
%-----------------------------------------------------------------


\appendix
%------------------------------------------------
\section{Annexes}
%------------------------------------------------

%-----------------------------------------------------------------
\subsection{Équivalence entre le calcul du pgcd par la méthode Euclidienne et géométrique}
\label{annexe-euc-geo}
%-----------------------------------------------------------------

%-----------------------------------------------------------------
\subsubsection{Trouver $q_n$}

Soit d la droite d'équation ax + by + c = 0 et soit un rayon issue de $S=(x_s, y_s)$ dans la direction du vecteur $V=(x_v, y_v)$.\\
Soit $M(x,y)$ un point tel que $\overrightarrow{SM} = q \overrightarrow{V}$. Les coordonnées de M vérifient : \\
 
\begin{equation*}
    \left\{
    \begin{split}
      x - x_s &= q x_v \\
      y - y_s &= q y_v 
   \end{split}
   \right. 
\end{equation*}

\begin{equation*}
    \left\{
    \begin{split}
      x &= q x_v + x_s \\
      y &= q y_v + y_s 
   \end{split}
   \right. 
\end{equation*}


On choisit le point M comme l'intersection de la droite d d'équation d : ax + by +c = 0 et du rayon. Les coordonnées de M vérifient : \\

$$(a x_v + b y_v) * q = -(a x_s + b y_s +c)$$
$$q = - \frac{a x_s + b y_s +c}{(a x_v + b y_v)}$$

%-----------------------------------------------------------------
\subsubsection{Rappel sur le calcul du pgcd par l'algorithme d'Euclide}


On rappelle les résultats de l'algorithme d'Euclide pour a,b : \\
\noindent $a       = k_0*b       + r_0$ d'où $k_0 = \lfloor b / a \rfloor$\\
\noindent $b       = k_1*r_0     + r_1$\\
\noindent $r_{n-2} = k_n*r_{n-1} + r_{n}$\\

%-----------------------------------------------------------------
\subsubsection{Preuve par récurrence sur n que $q_n = k_n$}

\paragraph{}

On vérifie que que la $q_0 = k_0$.\\

Soit de la droite passant par O(0,0) et P(b, a) d'équation d: ax + by =0 et soit le rayon issue de $p_{-2} = (1,0)$ dans la direction de $p_{-1} = (0,1)$.\\

On note le convergent $p_{0} = p_{-2} + q_0 p_{-1}$ avec $q_0$ est le plus grand entier tel que $p_0$ et $p_{-2}$ soient du même côté. Sa valeur est la partie entière de $-(a x_s + b y_s +c)/(a x_v + b y_v)$

$$q_0 = - \lfloor\frac{a*1 + b*0 +0}{a*0 + b*1}\rfloor = - \lfloor\frac{a}{b}\rfloor = k_0$$

On a montré que la relation est vraie au rang 0 avec \textbf{$q_0 = k_0$}.

\paragraph{}
On suppose vrai la relation au rang n.

$q_n = k_n$\\

Ce qui implique :  
$r_{n-2} = k_n*r_{n-1} + r_{n}$\\
$a x_{n-2} + b y_{n-2} = r_{n-2}$\\
$a x_{n-1} + b y_{n-1} = r_{n-1}$\\

Soit le rayon issue de $p_{n-1}$ dans la direction de $p_{n}$.\\
On note le convergent $p_{n+1} = p_{n-1} + q_{n+1} p_{n}$. $q_{n+1}$ est le plus grand entier tel que $p_{n+1}$ et $p_{n-1}$ soient du même côté. Sa valeur est la partie entière de $-(a x_{n-1} + b y_{n-1} +c)/(a x_{n} + b y_{n})$


$q_0$ est égale au premier coefficient de la division Euclidienne de b par a.

\begin{align*}
q_{n+1} &= - \lfloor\frac{a x_{n-1} + b y_{n-1}}{ a x_{n} + b y_{n}}\rfloor \\
        &= - \lfloor\frac{a x_{n-1} + b y_{n-1}}{ (a (x_{n-2} + q_n * x_{n-1}) + b (y_{n-2} + q_n * y_{n-1})}\rfloor \\
        &= - \lfloor\frac{a x_{n-1} + b y_{n-1}}{ q_n*(a x_{n-1} + b*y_{n-1} ) + a x_{n-2} + b*y_{n-2}  }\rfloor \\
        &= - \lfloor\frac{r_{n-1}              }{ r_{n-2} + k_n * r_{n-1}}\rfloor \\
        &= - \lfloor\frac{r_{n-1}              }{ r_{n} }\rfloor \\
        &= k_{n+1}
\end{align*}

On a montré que \textbf{$q_{n+1}= k_{n+1}$}. Comme la relation est vraie au rang 0 et au rang (n+1),  \textbf{elle est vraie pour tout n > 0}.

%-----------------------------------------------------------------
%%THOM
\subsubsection{Exemple d'équivalence pour le calcul du pgcd et des convergents de (3,8)}

%% TRI: je ne vois pas l'intérêt ici: en annexe ?
 \begin{table}[H]
   \centering
   \begin{tabular}{|p{0.04\linewidth}|p{0.04\linewidth}|p{0.04\linewidth}||p{0.04\linewidth}|p{0.04\linewidth}||p{0.2\linewidth}||p{0.35\linewidth}|}
     \hline 
     $k$ & $a$ & $b$ & $q_{k}$ & $r_k$ & (a,b) & $p_k$ \\
     \hline 
     -2 & - & - & - & - & -                    & $p_{-2} = (1,0)$\\
     -1 & - & - & - & - & $3 p_{-2} + 8p_{-1}$ & $p_{-1} = (0,1)$\\
      0 & 8 & 3 & 2 & 2 & $2 p_{-1} + 3p_{0}$  & $p_{0}  = p_{-2} + 2p_{-1} = (1,2)$\\ 
      1 & 3 & 2 & 1 & 1 & $p_{0}  + 2p_{1}$    & $p_{1}  = p_{-1} + p_{0} = (1,3)$\\
      2 & 2 & 1 & 2 & 0 & $p_{2}$              & $p_{2}  = p_{0} + 2p_{1} = (3,8)$\\
     \hline
   \end{tabular} 
   \caption{Résultat du $PGCD(3,8)$ par la méthode d'Euclide et géométrique}
 \end{table}

%-----------------------------------------------------------------
\subsection{Algorithme du calcul de l'$\alpha$-shape pour $\alpha <0$}
%-----------------------------------------------------------------



\begin{algorithm}[h]
  \label{algo:1}
  \KwIn{$\alpha$, $v_{in}$ any $\alpha$-shape vertex  }
  \KwOut{$L$ a list of consecutive $\alpha$-shape vertices starting from $v_{in}$ and $v_{out}$ the last vertex.}
  %
  $p_{-2} \leftarrow (1,0)$, $p_{-1} \leftarrow (0,1)$, $k \leftarrow 0$, stop $\leftarrow$ false \;
  \While{Ray-shooting intersecte the circle and non stop} 
  {
    $p_k = q_k p_{k-1} + q_{k-2}$ \;
    \eIf{$p_k$ is outside the shape} 
    {
      \tcc{k is odd}	
      \If{$k > 0$ et $qk < 0$} 
      {
        $v_{out} \leftarrow p_{k-1}$ \;
        $stop\leftarrow true$ \;
      }
    }
    {
      \tcc{$p_k$ is inside or lie on the shape}	
      \eIf{$k\: is\: even$} 
      {
        \tcc{$R$ returns the circumcircle radius of three points}	
        \If{ $R(v, (q_k - 1)p_{k-1} + p_{k-2}, p_k) > -1/\alpha$ }
        {
           \tcc{Find the greatest integer $0 \leq q \leq q_{k} - 1$ such that}
          $R(v, qp_{k-1} + p_{k-2}, (q+1)p_{k-1} + p_{k-2}) > -1/\alpha$ \;
          
          $L \leftarrow L \cup v$ \; 
          \For{$i$ from $1$ to $q_k-q$ } 
          {
            $L \leftarrow L \cup i p_{k-1} $ \;      
          }
          $v_{out} \leftarrow p_{k}$ \;
          $stop \leftarrow true$ \;
        }          
      }  
      {
        \If{ $R(v, p_k, (q_k - 1)p_{k-1} + p_{k-2}) > -1/\alpha$ }
        {
           \tcc{Find the greatest integer $0 \leq q \leq q_{k} - 1$ such that}
          $R(v, qp_{k-1} + p_{k-2}, (q+1)p_{k-1} + p_{k-2}) > -1/\alpha$ \;
        
          \eIf{ $qkalpha == 0$} 
          {
            $v_{out} \leftarrow p_{k-2}$ \;
            $stop \leftarrow true$ \;
          }
          {
            \For{$i$ from $1$ to $q_k-q$ } 
            {
              $L \leftarrow L \cup p_{k-2} + i p_{k-1} $ \;      
            }
            $v_{out} \leftarrow p_{k}$ \;
            $stop \leftarrow true$ \;
          }  
        }
      } 
    }
    \tcc{Update  $p_{k-1}$ and  $p_{k-2}$}	    
    $k \leftarrow k + 1$ \;
    $p_{k-2} \leftarrow p_{k-1}$ \;
    $p_{k-1} \leftarrow p_{k}$ \;
   }
   \If{$stop == false$ }
   {
    $v_{out} \leftarrow p_{k-1}$ \; 
   } 
\end{algorithm}



%-----------------------
% Biblio
%-------------------------
\bibliographystyle{plain} 
\bibliography{tl_bibliographie} 
%-------------------------

%\part{Développement informatiques}
%%%%%%%%%%%%%%%%%%%%%% chapter.tex %%%%%%%%%%%%%%%%%%%%%%%%%%%%%%%%%
%
% sample chapter
%
% Use this file as a template for your own input.
%
%%%%%%%%%%%%%%%%%%%%%%%% Springer-Verlag %%%%%%%%%%%%%%%%%%%%%%%%%%


\chapter{Environnement de travail}
\label{pt2-ch1-et} % Always give a unique label
% use \chaptermark{}
% to alter or adjust the chapter heading in the running head

\section{Environnement matériel et logiciel}
\label{pt2-ch1-sec:1}

\section{Paradigme de Programmation}
\label{pt2-ch1-sec:2}

\section{Espace de collaboration}
\label{pt2-ch1-sec:3}

\chapter{Structure du code}
\label{pt2-ch2-sd} % Always give a unique label

\section{Principaux concepts}
\label{pt2-ch2-sec:1}

\section{Installation et utilisation}
\label{pt2-ch2-sec:2}

\subsection{Librairies et licence}
\label{pt2-ch2-sec:2:1}

Le code source de ce présent projet est disponible sous les conditions de la licence GPLv3 \cite{GPLv3}. Néanmoins, il convient de remarquer que deux librairies sont utilisées directement : Boost (licence Boost  - \cite{boost-licence})) et DGtal (Licence GPLv3 - \cite{GPLv3}).

\subsubsection{Boost}

Boost \cite{boost} est une grosse librairie C++. \\
On remarque qu'elle est également utilisé par la libraire DGtal.\\

Son utilisation directe dans le cadre de ce projet est relativement parcimonieuse. En effet, son utilisation a été limité à \bsc{Program Options} pour la création d'une interface légère à base de paramètres à ajouter pour l’exécution des programmes commandant les sorties.


\subsubsection{DGtal}

DGtal \cite{DGtal} est une librairie developpé en partie en interne par des membres de l'équipe M2Disco en plus d'autres partenaires. Son objectif principal est de proposer des outils permettant de traiter de géométrie discrète. \\

Son utlisation est intervenu sur plusieurs plans. \\

(?? Utilisation de liste pour alléger un peu la partie ??)

L'avantage indéniable de faire des calculs avec seulement des entiers est de calculer de manière exacte. Néanmoins, une contrainte peut "vite" apparaitre en augmentant la taille des rayons. En effet le type int peut alors être débordé en trouvant un entier plus grand que le plus grand entier alors compréhensible par notre système. L'utilisation de la librairie DGtal a alors permis par l'intermédiaire de gmp d’utiliser de très grands entiers.\\

DGtal a également été utilisé dans le postprocessing pour tirer partie du tableau permettant d'obtenir des traces de nos enveloppes de cercles.

\subsection{Installation}
\label{pt2-ch2-sec:2:2}

\subsection{Utilisation}
\label{pt2-ch2-sec:2:3}






%\part{Conclusion}
%%%%%%%%%%%%%%%%%%%%%% chapter.tex %%%%%%%%%%%%%%%%%%%%%%%%%%%%%%%%%
%
% sample chapter
%
% Use this file as a template for your own input.
%
%%%%%%%%%%%%%%%%%%%%%%%% Springer-Verlag %%%%%%%%%%%%%%%%%%%%%%%%%%

\chapter{Conclusion}
\label{pt5-ch1-con} % Always give a unique label
% use \chaptermark{}
% to alter or adjust the chapter heading in the running head

% En premier lieu il faut faire une transition qui permettra de passer en douceur du développement à la conclusion ; il convient ensuite de passer à un résumé du développement, et enfin de faire une ouverture. La conclusion va du particulier vers le général, c’est-à-dire qu’elle suit le processus inverse de celui adopté dans l’introduction. 

%%%%%%%%%%%%%%%%%%%%%%%%%%%%%%%%%%%%%%%%%%%%%%%%%%%%%%%%%%%%%%%%%%%

\section{Développement}
\label{pt5-ch1-1}

\subsection{Transition}
\label{pt5-ch1-sec:1.1}

\subsection{Objectif}
\label{pt5-ch1-sec:1.2}

\section{Ouverture}
\label{pt5-ch1-sec:2}



%%%%%%%%%%%%%%%%%%%%%%%%%%%%%%%%%%%%%%%%%%%%%%%%%%%%%%%%%%%%%%%%%%%%%%

\end{document}





