%%%%%%%%%%%%%%%%%%%%%%%%%%%%%%%%%%%%%%%%%
% Simple Sectioned Essay Template
% LaTeX Template
%
% This template has been downloaded from:
% http://www.latextemplates.com
%
% Note:
% The \lipsum[#] commands throughout this template generate dummy text
% to fill the template out. These commands should all be removed when 
% writing essay content.
%
%%%%%%%%%%%%%%%%%%%%%%%%%%%%%%%%%%%%%%%%%

%----------------------------------------------------------------------------------------
%	PACKAGES AND OTHER DOCUMENT CONFIGURATIONS
%----------------------------------------------------------------------------------------

\documentclass[12pt]{article} % Default font size is 12pt, it can be changed here
\usepackage{geometry} % Required to change the page size to A4
\geometry{a4paper} % Set the page size to be A4 as opposed to the default US Letter

\usepackage{graphicx} % Required for including pictures
\usepackage{float} % Allows putting an [H] in \begin{figure} to specify the exact location of the figure


\usepackage[T1]{fontenc} %codage
\usepackage[english,francais]{babel}
\usepackage[utf8]{inputenc} %codage
\usepackage{lmodern}
\usepackage{amssymb}
\usepackage{amsmath}

\usepackage{amsthm}

\usepackage{url}

\usepackage{tikz}         
\usetikzlibrary{mindmap}

\linespread{1.2} % Line spacing

%\setlength\parindent{0pt} % Uncomment to remove all indentation from paragraphs





\begin{document}

%----------------------------------------------------------------------------------------
%	RE-DEFINITION 
%----------------------------------------------------------------------------------------
% MATHS
%-----------

\newtheorem{Definition}{Definition}
\newtheorem{Theorem}{Theorem}
\newtheorem{Proposition}{Proposition}
\newtheorem{Corollary}{Corollary}
\newtheorem{Problem}{Problem}
\newtheorem{Lemma}{Lemma}
 
\newcommand{\RefSec}[1]{Section~\ref{#1}}
\newcommand{\RefFig}[1]{Fig.~\ref{#1}}
\newcommand{\RefTab}[1]{Tab.~\ref{#1}}
\newcommand{\RefDef}[1]{Definition~\ref{#1}}
\newcommand{\RefPro}[1]{Proposition~\ref{#1}}
\newcommand{\RefLem}[1]{Lemma~\ref{#1}}
\newcommand{\RefThe}[1]{Theorem~\ref{#1}}
\newcommand{\Eq}[1]{(\ref{#1})}

% LINES
%-----------
\newcommand{\HRule}{\rule{\linewidth}{0.5mm}} % Defines a new command for the horizontal lines, change thickness here

%----------------------------------------------------------------------------------------
%	TITLE PAGE
%----------------------------------------------------------------------------------------

\begin{titlepage}

\center % Center everything on the page

\textsc{\LARGE Laboratoire d'InfoRmatique en Image et Syst\`{e}mes d'information}\\[1.5cm] % Name of your university/college
%\textsc{\Large }\\[0.5cm] % Major heading such as course name
\textsc{\large Rapport de stage}\\[0.5cm] % Minor heading such as course title

\HRule \\[0.4cm]
{ \huge \bfseries Structure du bord des disques discrets}\\[0.4cm] % Title of your document
\HRule \\[1.5cm]

\begin{minipage}{0.4\textwidth}
\begin{flushleft} \large
\emph{Auteur:}\\
Thomas \textsc{Lafond} % Your name
\end{flushleft}
\end{minipage}
~
\begin{minipage}{0.4\textwidth}
\begin{flushright} \large
\emph{Encadrant:} \\
Tristan \textsc{Roussillon} % Supervisor's Name
\end{flushright}
\end{minipage}\\[4cm]

{\large 3 Juillet 2013}\\[3cm] % Date, change the \today to a set date if you want to be precise

%\includegraphics{Logo}\\[1cm] % Include a department/university logo - this will require the graphicx package

\vfill % Fill the rest of the page with whitespace

\end{titlepage}

%----------------------------------------------------------------------------------------
%	TABLE OF CONTENTS
%----------------------------------------------------------------------------------------

\tableofcontents % Include a table of contents

\newpage % Begins the essay on a new page instead of on the same page as the table of contents 

%------------------------------------------------
\section{Introduction}
%------------------------------------------------

%-----------------------------------------------------------------
\subsection{Contexte pédagogique}
%-----------------------------------------------------------------

% Parcours - ok
Actuellement en deuxième année du master professionnel de Mathématiques : Statistiques, Informatique et Techniques Numériques (SITN) à l’Université Claude Bernard - Lyon 1, je me dois de réaliser un stage dans le but de valider ma formation et d'ainsi obtenir mon diplôme. \newline

%labo -ok
Je suis actuellement accueilli pour une durée de 6 mois au LIRIS - Laboratoire d'InfoRmatique en Image et Systèmes d'information - UMR 5205 CNRS. Le LIRIS est issue de la fusion de plusieurs pôles de recherche de la région Lyonnaise. Aujourd'hui composé d'environ 300 personnes, il participe activement à la recherche et à l'éducation à travers deux grands départements thématiques : "Image" et "Données, Connaissances, Services".\newline

%encadrant
Je suis encadré par Tristan Roussillon, Maitre de conférence à l'Insa - Institut National des Sciences Appliquées et membre de l'équipe M2Disco - Modèles Multirésolution, Discrets et Combinatoires. Cette équipe est une composante du département Image du LIRIS qui traite de sujets comme l'analyse d'images, l'optimisation, la programmation par contraintes et la géométrie discrète. C'est d'ailleurs sur cette dernière branche que le contexte de mon stage c'est déroulé.



%-----------------------------------------------------------------
\subsection{Contexte scientifique}
%-----------------------------------------------------------------

J'ai découvert pour l’occasion la géométrie discrète, plus fréquemment appelé digital geometry dans la langue de Shakespeare. Cette discipline de recherche gravite à l'intersection de bien des domaines mathématiques et informatiques.\\

%(pics de digital geometry) % traduire ?

\begin{figure}[h!]
  \centering
  \scalebox{0.7}
  {
    \begin{tikzpicture}[
      root concept/.append style={concept color=blue!20},
      level 1 concept/.append style={sibling angle=45},mindmap]
      \node [concept] { Digital Geometry} [clockwise from=90]
      child { node[concept] { Computational Geometry}} 
      child { node[concept] { Topology}} 
      child { node[concept] { Complexity analysis and Algorithmic}}
      child { node[concept] { Combinatorics}}
      child { node[concept] { Arithmetic}}
      child { node[concept] {Number Theory}} 
      child { node[concept] {Computer Graphics}} 
      child { node[concept] {Image Processing}};
    \end{tikzpicture}

  }
  \caption{Digital Geometry - Crédit : David Coeurjolly}
   
\end{figure}
Il s'agit principalement d'étudier la géométrie et la topologie de formes, d'objets portés sur des structures régulières. Dans le cadre de cet stage, nous intéresserons plus particulièrement au disque discret sur la grille $\mathbb{Z}^{2}$. 


%------------------------------------------------
\section{Motivations}
%------------------------------------------------

%-----------------------------------------------------------------
\subsection{Le bord du disque discret}
%-----------------------------------------------------------------

%-----------------------------------------------------------------
\subsubsection{Du disque Euclidien au disque discret}


Le  disque fermé discret $\mathcal{D}$ dans $\mathbb{Z}^{2}$ (appelé pour la suite du rapport uniquement disque discret) est défini par analogie avec le disque Euclidien $\mathcal{D}_e$. Il représente l'ensemble des points situés à une distance inférieure à $R$ de son centre $O(u,v)$.


\begin{Definition}{Disque fermé Euclidien}
\label{def:disk-euc}
 $$\mathcal{D}_e =  \left\{ (x,y) \in \mathbb{R}^{2} |  (x - u)^2 + (y - v)^2 \leq R^2 \right\}$$
\end{Definition}

\begin{Definition}{Disque fermé discret}
\label{def:disk-dis}
  $$\mathcal{D} =  \left\{ (x,y) \in \mathbb{Z}^{2} |  (x - u)^2 + (y - v)^2 \leq R^2 \right\}$$
  
  avec $O(u,v) \in \mathbb{Q}^{2}$ les coordonnées du centre et $R^2 \in \mathbb{Q}$ le rayon.\\
\end{Definition}

\begin{figure}[H]
  \centering
  \includegraphics[width=6cm]{fig/2-mot/circle/circle-euc-0.pdf}
  \includegraphics[width=6cm]{fig/2-mot/circle/circle-dis-0.pdf}
  \caption{Disque Euclidiens et Discret de centre O et de rayon R}
\label{fig:disk}
\end{figure}

La définition du cercle Euclidien découle directement de la définition du disque Euclidien en changeant l'inégalité par une égalité. Cependant, la définition du cercle discret ne dérive pas explicitement de la définition du disque discret. L'ensemble des points, représentés en rouge sur \textsc{figure} \ref{fig:disk} et appartenant à l'ensemble : $\left\{ (x,y) \in \mathbb{Z}^{2} |  (x - u)^2 + (y - v)^2 = R^2 \right\}$ ne suffit pas à représenter un cercle discret. Nous avons besoin de revenir à une définition plus générale du bord de tout ensemble discret en faisant appel à la notion de voisinage.\\

\begin{Definition}{4-Voisinage d'un point $(u,v)$}
\label{def:vois-4}
  $$\mathcal{V}_4(u,v) =  \left\{ (x,y) \in \mathbb{Z}^{2} |  |x-u|+|y-u| = 1 \right\}$$
\end{Definition}

\begin{Definition}{8-Voisinage d'un point $(u,v)$}
\label{def:vois-8}
  $$\mathcal{V}_8(u,v) =  \left\{ (x,y) \in \mathbb{Z}^{2} |  max(|x-u|,|y-u|) = 1 \right\}$$
\end{Definition}

\begin{figure}[H]
  \centering
  \includegraphics[width=.5\linewidth]{fig/2-mot/connexe/connexite.pdf}
  \caption{4-voisinage et 8-voisinage}
\end{figure}

Le bord d'un ensemble discret $S$ se traduit par l'ensemble des points à l'intérieur de $S$ dont le 4-voisinage (respectivement 8) n'est pas intégralement contenu dans $S$. Le bord récupéré est un ensemble 8-connexes (respectivement 4). Une illustration est donnée \textsc{figure} \ref{fig:bord}. Le cercle discret est défini commme le bord d'un disque discret. Nous utilisons le joker * pour désigner arbitrairement $4$ ou $8$. 

\begin{Definition}{Bord *-connexe d'un ensemble discret $S$}
\label{def:bord-ens}
  $$ \partial S_{*} =  \left\{ (x,y) \in S | \left( \mathcal{V}_{*}(x,y) \cap S \right) \neq \mathcal{V}_{*}(x,y) \right\}$$
\end{Definition}

\begin{Definition}{Cercle discret défini par le *-voisinage}
\label{def:cer-dis}
  $$ \mathcal{C}_{*} =  \left\{ (x,y) \in \mathcal{D} | \left( \mathcal{V}_{*}(x,y) \cap \mathcal{D} \right) \neq \mathcal{V}_{*}(x,y) \right\}$$
\end{Definition}

\begin{figure}[H]
  \centering
  \includegraphics[width=.3\linewidth]{fig/2-mot/circle/circle-dis-1a.pdf}
  \includegraphics[width=.3\linewidth]{fig/2-mot/circle/circle-dis-1b.pdf}
  \caption{Cercle Discret 8-connexes et 4-connexes}
\label{fig:bord}
\end{figure}


%-----------------------------------------------------------------
\subsubsection{Énoncé de la problématique}

Les points discrets sont organisés sur la grille régulière $\mathbb{Z}^{2}$. Chaque point discret possède donc quatre plus proches voisins à une distance d'une unité, en haut, à gauche, en bas et à droite de lui. Or, les points strictement à l'intérieur du disque discret et non sur le bord (en bleu clair sur \textsc{figure} \ref{fig:disque-bord}) possèdent tous leurs quatre plus proches voisins à l'intérieur du disque (maillage en rouge sur \textsc{figure} \ref{fig:disque-bord}).

\begin{Definition}{Ensemble de points strictement à l'intérieur d'un disque}
\label{def:int-ens}
  $$\stackrel{\ \circ}{\mathcal{D}}_{*} = \mathcal{D} / \mathcal{C_{*}} $$
  $$ \stackrel{\ \circ}{\mathcal{D}}_{*} =  \left\{ (x,y) \in \mathcal{D} | \mathcal{N}_{*}(x,y) \cap \mathcal{D} = \mathcal{N}_{*}(x,y) \right\}$$
\end{Definition}

\begin{figure}[H]
  \centering
  \includegraphics[width=.3\linewidth]{fig/2-mot/circle/circle-dis-2.pdf}
  \caption{Réseau de points strictement à l'intérieur d'un disque.}
\label{fig:disque-bord}
\end{figure}

Le disque discret est l'union de deux ensembles disjoints : l'intérieur et le bord. La structure du premier est évidente. C'est pourquoi, seule l'étude du second, le bord, nous intéresse pour comprendre l'organisation des points des disques discrets.\\

Notre objectif est de comprendre comment sont organisés spatialement les points du bord d'un disque discret, de comprendre comment cette structure est déterminée par les paramètres du disque (position et taille) par rapport à la grille sous-jacente.   

%-----------------------------------------------------------------
\subsection{Alpha-Shape}
%-----------------------------------------------------------------

En s'intéressant à des bords d'objets discrets, un panel d'outils nous est apparu comme particulièrement opportun. Il s'agit des $\alpha$-hull et des $\alpha$-shape définis pour la première fois par Edelsbrunner \emph{et. al.} \cite{EdeKirSei83} et faisant appel aux disques généralisés.\\

Un disque généralisé permet de définir des disques avec des rayons négatifs en faisant appel au complémentaire.

\begin{Definition}{Disques généralisés de rayon $1/\alpha$}\\
\label{def:dis-gen}
   \noindent - $\mathcal{D}_{\alpha}$ est le disque fermé de rayon $1/\alpha$ pour $\alpha > 0$.\\
   \noindent - $\mathcal{D}_{\alpha}$ est le complémentaire fermé du disque de rayon $- 1/\alpha$ pour $\alpha < 0$.
\end{Definition}

%-----------------------------------------------------------------
\subsubsection{Définition}

Soit $\mathcal{S}$ un ensemble fini de points. 

\begin{Definition}{$\alpha$-hull de $\mathcal{S}$}\\
\label{def:ah-txt}
    Intersection de tous les disques généralisés de rayon $1/\alpha$ qui contiennent tous les points de l'ensemble.
    $$ \alpha_h(\mathcal{S}) = \cap \left\{ \mathcal{D}_{\alpha} | \mathcal{S} \subseteq \mathcal{D}_{\alpha} \right\}$$
\end{Definition}

\begin{figure}[H]
  \centering
  \includegraphics[width=0.3\linewidth,page=1]{fig/2-mot/as/mot-alpha-shape.pdf}
  \includegraphics[width=0.3\linewidth,page=3]{fig/2-mot/as/mot-alpha-shape.pdf}
  \caption{$\alpha$-Hull négative et $\alpha$-Hull positive }
\end{figure}
  
Les sommets de $\alpha$-hull sont appelés points $\alpha$-extrêmes. S'ils sont reliés par un arc de cercle de rayon $\pm 1/ \alpha$ qui ne contient aucun autre point que ses extrémités et qui se trouve sur le bord d'un disque généralisé contenant l'ensemble des points, on dit qu'ils sont adjacents.

\begin{Definition}{$\alpha$-shape}\\
\label{def:as}
      Graphe plongé dans le plan reliant tous les points $\alpha$-extrêmes adjacents par des segments de droite.
\end{Definition}

\begin{figure}[H]
  \centering
  \includegraphics[width=0.3\linewidth,page=2]{fig/2-mot/as/mot-alpha-shape.pdf}
  \includegraphics[width=0.3\linewidth,page=4]{fig/2-mot/as/mot-alpha-shape.pdf}
  \caption{$\alpha$-Shape négative et $\alpha$-Shape positive }
\end{figure}


%-----------------------------------------------------------------
\subsubsection{$\alpha$-shapes de disque discret}

Le paramètre $\alpha$ est défini dans l'intervalle allant de $-2$ à $R_{min}$ (le rayon du plus petit cercle englobant). 
Dans cet intervalle un vaste ensemble d'$\alpha$-shape est possible.  
%Les $\alpha$-shapes représentent un sous-ensemble des points du bord du disque discret.

\begin{figure}[H]
  \centering
  \includegraphics[width=0.4\linewidth]{fig/2-mot/as/mot-as-1.pdf}
  \includegraphics[width=0.4\linewidth]{fig/2-mot/as/mot-as-2.pdf}
  \caption{$\alpha$-shapes disque discret - $\alpha < 0$}
\end{figure}

\begin{figure}[H]
  \centering
  \includegraphics[width=0.4\linewidth]{fig/2-mot/as/mot-as-3.pdf}
  \includegraphics[width=0.4\linewidth]{fig/2-mot/as/mot-as-4.pdf}
  \caption{$\alpha$-shapes de disque discret - $\alpha = 0$ et - $\alpha > 0$}
\end{figure}



%-----------------------------------------------------------------
\subsubsection{Propriétés}


\begin{itemize}
  \item Le cas où $\alpha = 0$ est une intersection de disques généralisés de rayon infini. Ce cas s'interprète comme une intersection de demi-plans, menant à la définition de l'enveloppe convexe.
  \item Les cas où $\alpha = -2$ et $\alpha = -\sqrt{2}$ correspondent aux bords définis au moyen du 8 et 4-voisinage. 
%En ne pouvant s'éloigner au plus de disques de rayon $1/2$ et $\sqrt{2}/2$, on ne peut suivre le bord de nos disques que par l'intermédiaire des plus proches 4-voisins et 8-voisins.
%  \item Le cas où $\alpha$ < -2 conserve un sens pour l'$\alpha$-hull. Elle est constitué de l'ensemble disjoint des points appartenant au disque. L'$\alpha$-shape n'a pas d'existence propre dans ce cas au vu la non-connexité de l'ensemble.
%  \item Au-delà d'un certain rayon, nous avons $R_{\alpha} < R_D$ et il devient impossible de récupérer l'ensemble des points de notre disque par l'intersection de disques de un rayon plus petit.
  \item L'union des $\alpha$-shape \cite{EdeKirSei83} pour le cas négatif et le cas positif représente des sous-ensembles des triangulations d'ordre 0 et d'ordre n de Delaunay.   
\end{itemize}
 
%-----------------------------------------------------------------
\subsection{Triangulation de Delaunay}
%-----------------------------------------------------------------

La triangulation est un moyen de relier les points d'un ensemble entre eux. Les triangulations de Delaunay possèdent des propriétés intéressantes.  Elles ont été nommées d'après le mathématicien russe Boris Delone : 1890-1980.

\begin{Definition}{Triangulation de Delaunay d'ordre 0 de l'ensemble $\mathcal{S}$}\\
\label{def:tri-del-0}
  La triangulation de Delaunay d'ordre 0 est une triangulation où chaque disque circonscrit au triangle ne contient aucun autre point que les sommets du triangle.
\end{Definition}

\begin{Definition}{Triangulation de Delaunay d'ordre n de l'ensemble $\mathcal{S}$}\\
\label{def:tri-del-n}
  La triangulation de Delaunay d'ordre n est une triangulation de l'enveloppe convexe de $\mathcal{S}$ où chaque disque circonscrit au triangle contient tous les points de l'ensemble.
\end{Definition}

\begin{figure}[H]
  \centering
  \includegraphics[width=0.4\linewidth]{fig/2-mot/tri/mot-tri-a.pdf}
  \includegraphics[width=0.4\linewidth]{fig/2-mot/tri/mot-tri-b.pdf}
  \caption{Triangulations d'ordre 0 et n}
\end{figure}


Les triangulations de Delaunay sont reliées aux $\alpha$-shapes par plusieurs propriétés. \cite{EdeKirSei83}

\begin{Lemma}
  Une $\alpha$-shape de $\mathcal{S}$ est un sous-graphe de la triangulation de Delaunay.
\end{Lemma}

\begin{Lemma}
  Pour tout point $p \in \mathcal{S}$, il existe un réel $\alpha_{max}(p)$ tel que $p$ soit un $\alpha$-extrême de $\mathcal{S}$ si et seulement si $\alpha \leq \alpha_{max}(p)$.
\end{Lemma}

\begin{Lemma}
  Toutes arêtes $e$ de notre triangulation de Delaunay est également une arête de l'$\alpha$-shape s'il existe $\alpha_{min}(e) \leq \alpha_{max}(e)$ tel que $\alpha_{min}(e) \leq \alpha \leq \alpha_{max}(e)$.
\end{Lemma}


%------------------------------------------------
\section{Méthodes de calculs existantes}
%------------------------------------------------

%-----------------------------------------------------------------
\subsection{Suivi de Bord}
%-----------------------------------------------------------------

La méthode du suivi du bord d'un disque discret est un processus étudié et connu. Elle est sensiblement la même suivant le cas souhaité : 4-connexes ou 8-connexes. Elle se décompose en deux étapes. Trouver un point sur le bord, puis chercher le point suivant de manière répétée jusqu'à retrouver le premier point et refermer le bord.

%-----------------------------------------------------------------
%\subsubsection{Trouver le sommet de départ}

Pour chercher un point de départ sur le bord du disque discret à partir de la seule connaissance de ses paramètres, nous avons choisi de récupérer le point d'ordonnée minimale et d'abscisse maximale à l'intérieur du disque. Nous prenons le point de $\mathbb{Z}^2$ avec les coordonnées entières du centre du disque. Puis, nous descendons le long de l'axe vertical d'une longueur entière égale au rayon pour trouver un point d'ordonnée minimale. Ensuite, nous translatons suivant l'axe horizontal afin de récupérer le point d'abscisse maximal. (Illustré \textsc{figure} \ref{fig:depart}.) On note \textbf{a} ce point de départ. Cette procédure prend un temps constant. 

\begin{figure}[H]
  \centering
  \includegraphics[width=0.3\linewidth,page=1]{fig/4-exi/suivi/exi-depart-0.pdf}
  \includegraphics[width=0.3\linewidth,page=1]{fig/4-exi/suivi/exi-depart-1.pdf}
  \caption{Recherche du premire point \textbf{a} à partir du centre et du rayon du disque.}
\label{fig:depart}
\end{figure}
  

%-----------------------------------------------------------------
%\subsubsection{Trouver le sommet suivant}

La deuxième étape consiste à trouver le sommet suivant appartenant au bord. Cette étape est répétée jusqu'à retrouver le point a. A partir d'un point quelconque du bord $p$ ($a$ au début), le point suivant est choisi suivant la position des 4 voisins (resp 8 voisins) de $p$ par rapport au disque et d'un sens arbitraire de rotation. En tournant dans le sens trigonométrique et en partant d'un voisin situé à l'extérieur du disque, le point suivant est le premier voisin situé à l'intérieur du disque. 

\begin{figure}[H]
  \centering
  \includegraphics[width=0.4\linewidth,page=1]{fig/4-exi/suivi/exi-suivi-0.pdf}
  \caption{Suivi de bord 4-connexes et 8-connexes}
\end{figure}
  
Cette étape prend un temps constant, limité par la taille constante du voisinage. Elle est répétée autant de fois qu'il y a de points sur le bord du disque. Rappelons que $R$ est le rayon du disque. Comme il y a $O(R)$ points sur le bord du disque, la complexité en temps du suivi est en $O(R)$. 

%-----------------------------------------------------------------
\subsection{Enveloppe Convexe : Algorithme de Har-Peled}
%-----------------------------------------------------------------

De nombreux algorithmes existent pour calculer l'enveloppe convexe d'un ensemble de points. L'algorithme de Graham \cite{Graham1972} l'implémente en $O(n \log n)$ pour un ensemble quelconque de $n$ points. Quand les $n$ points sont ordonnés, comme le sont les points du bord d'un disque discret, le parcours de Graham est en $O(n)$. Par conséquent, calculer l'enveloppe convexe des points d'un disque discret se calcule par suivi de bord et parcours de Graham en $O(R)$. Cependant un algorithme géométrique introduit par Har-Peled en 1998 \cite{HarPeled98} calcule l'enveloppe convexe des points d'un disque discret de manière incrémentale et ``output-sensitive''.

\begin{Definition}{Output sensitive}\\
\label{def:os}
      Un algorithme output sensitive possède un temps d’exécution qui dépend de la taille de sa sortie.
\end{Definition}

La méthode de Har-Peled dépend du nombre de sommets de l'enveloppe convexe. Elle construit successivement les arêtes du polygone à l'aide des convergents qui représente le pendant géométrique du calcul du pgcd de deux nombres entiers (voir l'annexe \ref{annexe-euc-geo}). La complexité en temps de cet algorithme pour un disque de rayon $R$ relève d'une part de la recherche du prochain sommet en $O(\log R)$ et également du nombre de sommets qui est $O(R^{2/3})$. Soit une complexité totale en temps de $O( R^{2/3} \log R)$.

%-----------------------------------------------------------------
\subsubsection{Calcul des convergents}

Cette méthode de calcul est géométrique. Soient l’origine $O=(0,0)$ et $P = (P_x, P_y)$ un point à coordonnées entières. Nous cherchons le premier point de $\mathbb{Z}^{2}$ appartenant au segment de droite [O,P]. Le coefficient trouvé correspond au pgcd de $P_x$ et $P_y$.\\

Soient $p_{-2} = (1,0)$ et $p_{-1} = (0,1)$ les deux premiers convergents. Pour trouver les convergents suivants, nous mettons en place une méthode récursive :

$$p_{k} = p_{k-2} + q_k p_{k-1}$$

où $q_k$ est le plus grand entier tel que $p_{k}$ et $p_{k-2}$ soient du même côté de la droite.\\

L'opération correspond à jeter un rayon de $p_{k-2}$ dans la direction de $p_{k-1}$ pour étudier l’intersection du vecteur et du segment de droite de direction $y_P / x_P$. La méthode s’arrête quand un convergent $p_{k}$ est exactement sur la droite.\\

\begin{figure}[H]
  \centering
  \includegraphics[width=0.4\linewidth]{fig/4-exi/har/exi-har-0.pdf}
  \includegraphics[width=0.4\linewidth]{fig/4-exi/har/exi-har-1.pdf}
  \caption{Calcul des convergents du point (3,8)}
  \label{fig:conv}
\end{figure}

Le calcul des convergents est illustré en \textsc{figure} \ref{fig:conv} avec le point (3,8). Après avoir positionné $p_{-2} = (1,0)$ et de $p_{-1} = (0,1)$, nous calculons successivement $p_{0} = p_{-1} + 2p_{-2} = (1,2)$ et  $p_{1} = p_{-1} + p_{0} = (1,3)$. Le dernier convergent calculé est $p_{2} = p_{0} + 2p_{1} = (3,8)$ qui est le premier point sur le segment. 3 et 8 sont premiers entre eux.

%-----------------------------------------------------------------
\subsubsection{Passage au disque}


La méthode de calcul de l'enveloppe convexe d'un disque se décompose en deux étapes. La première consiste à trouver un point de départ. Nous appliquons la même procédure que lors du suivi de bord pour récupérer le point à l'intérieur du disque d'ordonnée minimale et d'abscisse maximale. Par définition ce point appartient à l'enveloppe convexe.\\

Ensuite nous cherchons le sommet suivant appartenant au bord et répétons cette étape jusqu'à retrouver le point a, notre point de départ. Cette étape est réalisée en calculant les convergents les plus proches du bord du disque. Nous allons alternativement être à l'intérieur du disque lorsque $k$ est impair et à l'extérieur du disque lorsque $k$ est pair. Nous repartons d'un convergent s'il se situe exactement sur le bord du disque. Sinon, nous repartons du dernier convergent de degré impair lorsque que le lancer de rayon n'intersecte pas notre disque.\\

\begin{figure}[H]
  \centering
  \includegraphics[width=0.4\linewidth]{fig/4-exi/har/exi-har-10.pdf}
  \includegraphics[width=0.4\linewidth]{fig/4-exi/har/exi-har-11.pdf}
  \caption{Calcul de l'enveloppe convexe d'un disque}
\label{fig:disque-conv}  
\end{figure}

L'application de l'algorithme est illustrée \textsc{figure} \ref{fig:disque-conv} en partant d'un sommet sur l'origine. Avec $p_{-2} = (1,0)$ et de $p_{-1} = (0,1)$, on trouve $p_{0} = p_{-1} + 2p_{-2} = (1,2)$ à l'extérieur du cercle puis $p_{1} = p_{-1} + 2p_{0} = (2,5)$ à l'intérieur du cercle. Le rayon $p_{2}$ n'intersectant pas le disque, on arrête l'algorithme et on repart du dernier convergent à l'intérieur du disque : $p_{1}$.

\subsubsection{Résultats}

Les résultats illustrés \textsc{figure} \ref{fig:ch} et écrits \textsc{table} \ref{tab:ch} ont été obtenus en calculant le nombre de sommets sur une moyenne de 100 disques de rayon $2^k$ et avec un centre à coordonnée rationnel compris dans $[0,1]\times[0,1]$. Afin de vérifier la convergence en $O(R^{2/3})$, nous avons également récupéré la moyenne par rayon de la division du nombre de sommets de l'enveloppe convexe sur le rayon à la puissance 2/3. La zone bleue de la figure correspond à l'intervale entre le minimum et le maximum obtenu.

On s'intéresse également à vérifié la complexité en temps de notre algorithme. On compare son temps d'exécution avec celui de la marche de Graham sur la \textsc{figure} \ref{tab:ch-time}. 

\begin{figure}[H]
  \centering
  \includegraphics[width=\linewidth]{fig/4-exi/ch/exi-ch-sommet.png}
  \caption{Sommets et bord de l'enveloppe convexe}
  \label{fig:ch} 
\end{figure}

\begin{table}[H]
  \begin{tabular}{|p{0.09\linewidth}|p{0.13\linewidth}||p{0.2\linewidth}|p{0.13\linewidth}||p{0.2\linewidth}|p{0.13\linewidth}|}
    \hline
    \multicolumn{2}{|c||}{Rayon} & \multicolumn{4}{c|}{Enveloppe convexe} \\  \hline 
    $R=2^k$  &  & \multicolumn{2}{c||}{Nombre de Sommets} &  \multicolumn{2}{c|}{Nombre de points sur le bord} \\ \hline 
    k & R &   & $\# / R^{2/3}$  &   & $\# / R^{2/3}$ \\    
    \hline
    5 & 32         & 35,36     & 3,51 & 102,05   &  10,12\\
    6 & 64         & 55,78     & 3,49 & 170,16   &  10,64\\
    7 & 128        & 87,78     & 3,46 & 283,69   &  11,17\\
    8 & 256        & 139,71    & 3,47 & 465,06   &  11,53\\
    9 & 512        & 222,07    & 3,47 & 761,01   &  11,89\\
    10 & 1024      & 351,72    & 3,46 & 1,24E+03 &  12,21\\
    11 & 2048      & 558,18    & 3,46 & 2,01E+03 &  12,45\\
    12 & 4096      & 883,86    & 3,45 & 3,24E+03 &  12,68\\
    13 & 8192      & 1,40E+003 & 3,45 & 5,25E+03 &  12,92\\
    14 & 16384     & 2,23E+003 & 3,45 & 8,41E+03 &  13,03\\
    15 & 32768     & 3,54E+003 & 3,45 & 1,35E+04 &  13,19\\
    16 & 65536     & 5,62E+003 & 3,46 & 2,16E+04 &  13,28\\
    17 & 131072    & 8,91E+003 & 3,45 & 3,47E+04 &  13,45\\
    18 & 262144    & 1,41E+004 & 3,45 & 5,54E+04 &  13,53\\
    19 & 524288    & 2,25E+004 & 3,45 & 8,87E+04 &  13,64\\
    20 & 1048576   & 3,56E+004 & 3,45 & 1,42E+05 &  13,75\\
    21 & 2097152   & 5,66E+004 & 3,45 & 2,26E+05 &  13,81\\
    22 & 4194304   & 8,98E+004 & 3,45 & 3,61E+05 &  13,88\\
    23 & 8388608   & 1,43E+005 & 3,45 & 5,76E+05 &  13,94\\
    24 & 16777216  & 2,26E+005 & 3,45 & 9,19E+05 &  14,02\\
    25 & 33554432  & 3,59E+005 & 3,45 & 1,46E+06 &  14,07\\
    26 & 67108864  & 5,70E+005 & 3,45 & 2,33E+06 &  14,10\\
    27 & 134217728 & 8,98E+005 & 3,42 & 3,74E+06 &  14,27\\
    28 & 268435456 & 1,35E+06  & 3,24 & 6,62E+06 &  15,90\\
    \hline
  \end{tabular} 
  \caption{Sommet et bord de l'enveloppe convexe}
  \label{tab:ch} 
\end{table}

\begin{figure}[H]
  \centering
  \includegraphics[width=\linewidth]{fig/4-exi/ch/exi-ch-temps.png}
  \caption{Temps de calcul de l'enveloppe convexe (échelle log / log)}
\label{tab:ch-time}   
\end{figure}


Les résultats obtenus sont conformes à ceux de la publication \cite{HarPeled98}. On remarque que la moyenne asymptotique de la division du nombre moyen de sommets de l'enveloppe convexe sur le rayon à la puissance 2/3 est 3,45. Des anomalies commencent à apparaître pour des rayons de la taille de $2^{27} = 134217728$. Il convient de chercher à comprendre d'où elles viennent afin de mieux cerner les possibles limitations de notre algorithme.\\

Le graphique représentant les temps est également intéressant. On observe avec l'échelle logarithmique que la complexité en temps est sous-linéaire pour la méthode de Har-Peled. La méthode devient d'ailleurs plus intéressante que la marche de Grahaam en terme de temps de calcul assez rapidement à partir d'un rayon $2^{10} = 1024$ unités. 






%------------------------------------------------
\section{Contributions}
%------------------------------------------------

%-----------------------------------------------------------------
\subsection{$\alpha$-shape, $\alpha \leq 0$ - Généralisation de Har-Peled}
%-----------------------------------------------------------------

%-----------------------------------------------------------------
\subsubsection{Construction de l'algorithme}

Le principe de l’algorithme est assez simple. On cherche à reproduire le schéma du calcul de l’enveloppe convexe. Néanmoins, à chaque convergents à l'intérieur du disque on va contrôler s'il est possible pour lui d'être atteint par un le rayon d'un disque généralisé.

%-----------------------------------------------------------------
\subsubsection{Résultats}

Les résultats présentés sont : 

\begin{tabular}{|l|c||c|c|}
\hline
Rayons & Prédicat & Nb Sommets & Nb Sommets / $R^{2/3}$\\
\hline

\hline
\end{tabular} 



%-----------------------------------------------------------------
\subsection{$\alpha$-shape, $\alpha \geq 0$}
%-----------------------------------------------------------------

%-----------------------------------------------------------------
\subsubsection{Construction de l'algorithme}


Notre version de l'algorithme s'appuie ici sur les sommets de l'enveloppe convexe. En effet, on va chercher à savoir quel sous-ensemble de point compose notre $\alpha$-shape.

%-----------------------------------------------------------------
\subsubsection{Résultats}

Les résultats présentés sont : 

\begin{tabular}{|l|c||c|c|}
\hline
Rayons & Prédicat & Nb Sommets & Nb Sommets / $R^{2/3}$\\
\hline

\hline
\end{tabular} 




%------------------------------------------------
\section{Développement Informatiques}
%------------------------------------------------

% Je ne sais pas encore où et comment organisé cette section.

%-----------------------------------------------------------------
\subsection{Paradigme de Programmation}
%-----------------------------------------------------------------

Je souhaite parlé ici de ce que j'ai appris en informatiques et plus particulièrement en développemenet. Parler rapidement des paradigmes utilisés : c++ programmation générique : Point + shape Intersectable + algo, compilation automatisé, test unitaire...

%-----------------------------------------------------------------
\subsection{Espace de collaboration}
%-----------------------------------------------------------------

Je souhaite mettre en avant la plateforme utilisé et expliqué rapidement les méthodes de developpement utilisé et appréhender en parallèle. Utilisation de git, branch, de méthode agile.

%-----------------------------------------------------------------
\subsection{Structure du rendu}
%-----------------------------------------------------------------

%-----------------------------------------------------------------
\subsubsection{Principaux concepts}

Traiter un mot sur l'organisation du dépot et sur l'utilisation de celui-ci.
Installation/ utilisation de DGtal.
Les différentes commandes crée : stools, make test, stests...



%------------------------------------------------
\section{Conclusion}
%------------------------------------------------

%%TRI: TODO: réponse à la question posée 
%Notre objectif est de comprendre comment sont organisés spatialement les points du bord d’un disque discret, de comprendre comment cette structure est déterminée par les paramètres du disque (position et taille) par rapport à la grille sous-jacente.

L'algorithme incrémental et output-sensitive de calcul de l'$\alpha$-shape pour $\alpha \leq 0$ peut être vu comme une boite noire qui, à partir d'un certain sommet, donne le suivant dans l'ordre trigonométrique. Ainsi, il fournit une réponse partielle à notre problématique : entre les deux extrémités d'une arête de l'enveloppe convexe, l'organisation spatiale des points est déterminée par les convergents de la pente de l'arête. Néanmoins, les relations de distance entre les sommets de l'enveloppe convexe, données par l'$\alpha$-shape pour $\alpha > 0$, reste encore à étudier, par exemple au moyen d'une approche ``Top-down''.


%-----------------------------------------------------------------
\subsection{Poursuite du projet}
%-----------------------------------------------------------------

Le projet scientifique commencé par l'intermédiaire de ce stage peut être étendu sur bien des points. 

\begin{itemize}
\item Nous avons élaboré un calcul incrémental et output-sensitive des $\alpha$-shape pour $\alpha \leq 0$. Etant donnée la structure récursive des triangulations de Delaunay des motifs de droite discrète, une approche récursive à partir des arêtes de l'enveloppe convexe pourrait être proposée et comparée à celle déjà implémentée. 
\item La preuve de la complexité du calcul de l'enveloppe convexe d'un disque discret en $O(R^{2/3} \log R)$ a été apporté par Har-Peled  \cite{HarPeled98}. Nous conjecturons que la complexité du calcul de l'$\alpha$-shape d'un disque discret est aussi $O(R^{2/3} \log R)$ lorsque $\pm 1 / \alpha$ est en $O(R)$. 
\item Nous avons proposé un calcul ``Bottom-Up'' des $\alpha$-shape pour $\alpha > 0$.  Implémenter l'approche ``Top-down'' pour le cas $\alpha > 0$ permettrait de mieux comprendre les relations de distance dans l'organisation spatiale des points du bord d'un disque discret tout en donnant une complexité plus intéressante. 
\item Nous pourrions aussi traiter d'autres formes convexes, comme l'ellipse.  
\item Enfin, il serait intéressant de mener ce travail en dimension supérieure. En dimension $d$, le nombre de sommets de l'enveloppe convexe de la discrétisation d'une boule est toujours sous-linéaire, plus précisément en $O(R^{d/(d+1)})$ \cite{Balog1991}.
\end{itemize}

%% Enfin, le but de tout algorithme étant d'être utilisé, il serait intéressant une fois le projet achevé de chercher à l'intégrer à la librairie DGtal afin de pouvoir utiliser les $\alpha$-shapes dans un cadre de reconnaissance de forme et d’échantillonnage \cite{BernardiniB97}.


%-----------------------------------------------------------------
\subsection{Compétences acquises}
%-----------------------------------------------------------------

À l'occasion de ce stage, j'ai pu m'améliorer dans des domaines bien distincts. D'un point de vue scientifique, j'ai découvert un domaine de recherche : la géométrie discrète particulièrement intéressant et original dans ses approches de résolutions. À travers l'implémentation des méthodes de résolution couvertes lors de mon stage, j'ai beaucoup appris sur le développement informatique en général. En particulier, j'ai grandement apprécié renouer au C++ à travers la découverte de la programmation générique. Cet apprentissage technique doit être mis en parallèle de l’acquisition de compétences dans le domaine du travail collaboratif et de la gestion de projet. L'utilisation quotidienne de git et de nombreux outils automatisant le projet ont facilités la discussion et le partage autour du programme.

%-----------------------------------------------------------------
\subsection{Ouverture personnelle}
%-----------------------------------------------------------------

Je n'aurai pu souhaiter meilleure fin pour ce manuscrit qui ne signifie pas seulement la fin de ce stage, mais aussi la fin d'un cycle entamé il y a deçà plusieurs années : celui de mes études et de mon enseignement scientifique supérieur. Fort de cette expérience gratifiante et enrichissante, je possède aujourd'hui le sentiment d'être prêt à refermer cette porte pour m'ouvrir vers ce nouvel environnement, qui je l'espère gravitera autour d'un domaine scientifique et continuera à m'apprendre et m'enseigner dans des domaines divers et variés.




%\part{Développement informatiques}
%%%%%%%%%%%%%%%%%%%%%% chapter.tex %%%%%%%%%%%%%%%%%%%%%%%%%%%%%%%%%
%
% sample chapter
%
% Use this file as a template for your own input.
%
%%%%%%%%%%%%%%%%%%%%%%%% Springer-Verlag %%%%%%%%%%%%%%%%%%%%%%%%%%


\chapter{Environnement de travail}
\label{pt2-ch1-et} % Always give a unique label
% use \chaptermark{}
% to alter or adjust the chapter heading in the running head

\section{Environnement matériel et logiciel}
\label{pt2-ch1-sec:1}

\section{Paradigme de Programmation}
\label{pt2-ch1-sec:2}

\section{Espace de collaboration}
\label{pt2-ch1-sec:3}

\chapter{Structure du code}
\label{pt2-ch2-sd} % Always give a unique label

\section{Principaux concepts}
\label{pt2-ch2-sec:1}

\section{Installation et utilisation}
\label{pt2-ch2-sec:2}

\subsection{Librairies et licence}
\label{pt2-ch2-sec:2:1}

Le code source de ce présent projet est disponible sous les conditions de la licence GPLv3 \cite{GPLv3}. Néanmoins, il convient de remarquer que deux librairies sont utilisées directement : Boost (licence Boost  - \cite{boost-licence})) et DGtal (Licence GPLv3 - \cite{GPLv3}).

\subsubsection{Boost}

Boost \cite{boost} est une grosse librairie C++. \\
On remarque qu'elle est également utilisé par la libraire DGtal.\\

Son utilisation directe dans le cadre de ce projet est relativement parcimonieuse. En effet, son utilisation a été limité à \bsc{Program Options} pour la création d'une interface légère à base de paramètres à ajouter pour l’exécution des programmes commandant les sorties.


\subsubsection{DGtal}

DGtal \cite{DGtal} est une librairie developpé en partie en interne par des membres de l'équipe M2Disco en plus d'autres partenaires. Son objectif principal est de proposer des outils permettant de traiter de géométrie discrète. \\

Son utlisation est intervenu sur plusieurs plans. \\

(?? Utilisation de liste pour alléger un peu la partie ??)

L'avantage indéniable de faire des calculs avec seulement des entiers est de calculer de manière exacte. Néanmoins, une contrainte peut "vite" apparaitre en augmentant la taille des rayons. En effet le type int peut alors être débordé en trouvant un entier plus grand que le plus grand entier alors compréhensible par notre système. L'utilisation de la librairie DGtal a alors permis par l'intermédiaire de gmp d’utiliser de très grands entiers.\\

DGtal a également été utilisé dans le postprocessing pour tirer partie du tableau permettant d'obtenir des traces de nos enveloppes de cercles.

\subsection{Installation}
\label{pt2-ch2-sec:2:2}

\subsection{Utilisation}
\label{pt2-ch2-sec:2:3}






%\part{Conclusion}
%%%%%%%%%%%%%%%%%%%%%% chapter.tex %%%%%%%%%%%%%%%%%%%%%%%%%%%%%%%%%
%
% sample chapter
%
% Use this file as a template for your own input.
%
%%%%%%%%%%%%%%%%%%%%%%%% Springer-Verlag %%%%%%%%%%%%%%%%%%%%%%%%%%

\chapter{Conclusion}
\label{pt5-ch1-con} % Always give a unique label
% use \chaptermark{}
% to alter or adjust the chapter heading in the running head

% En premier lieu il faut faire une transition qui permettra de passer en douceur du développement à la conclusion ; il convient ensuite de passer à un résumé du développement, et enfin de faire une ouverture. La conclusion va du particulier vers le général, c’est-à-dire qu’elle suit le processus inverse de celui adopté dans l’introduction. 

%%%%%%%%%%%%%%%%%%%%%%%%%%%%%%%%%%%%%%%%%%%%%%%%%%%%%%%%%%%%%%%%%%%

\section{Développement}
\label{pt5-ch1-1}

\subsection{Transition}
\label{pt5-ch1-sec:1.1}

\subsection{Objectif}
\label{pt5-ch1-sec:1.2}

\section{Ouverture}
\label{pt5-ch1-sec:2}



%%%%%%%%%%%%%%%%%%%%%%%%%%%%%%%%%%%%%%%%%%%%%%%%%%%%%%%%%%%%%%%%%%%%%%

\end{document}





