%------------------------------------------------
\section{Contributions}
%------------------------------------------------


\begin{frame}
\frametitle{Introduction}
\begin{block}{Indices d'existance}
% indiquant la possible existance d'un algorithme incrémental et output sensitive calculant les $\alpha$-shapes de la discrétisation d'un disque (quand $\alpha$ est négatif) : 

\begin{enumerate}
\item Le bord de la discrétisation d'un disque est convexe. Tout bord convexe se décompose de manière unique en motifs de droite discrète \cite{roussillonPR2011}. 
\item Il existe un algorithme incrémental et output sensitive calculant les motifs de la discrétisation d'un disque par les convergents. \cite{HarPeled98}
\item L'union des alpha-shapes lorsque $\alpha$ est négatif est un sous-ensemble de la triangulation de Delaunay \cite{EdeKirSei83}et la triangulation de Delaunay d'un motif est déterminé par les convergents de sa pente \cite{RoussillonL11}. 
\end{enumerate}

\end{block}

%----- Begin biblio -----
\begin{columns}[t]
  \begin{column}{0.5\linewidth}
    \scriptsize  
    \begin{thebibliography}{alpha}
      \bibitem{roussillonPR2011}
      [RS11] T. Roussillon and I. Sivignon
      \newblock Faithful Polygonal Representation of the Convex and Concave Parts of a Digital Curve
      \newblock {\em Pattern Recognition}, 2011.
      
      \bibitem{HarPeled98}
      [HP98] Har-Peled
      \newblock An Output Sensitive Algorithm for Discrete Convex Hulls
      \newblock {\em ICGTA: Computational Geometry: Theory and Applications}, 1998.  
    \end{thebibliography}
    \scriptsize
  \end{column}
  \begin{column}{0.5\linewidth}
    \scriptsize
    \begin{thebibliography}{alpha}
      \bibitem{EKS83}
      [EKS83] Edelsbrunner, H., Kirkpatrick, D., Seidel, R.
      \newblock On the Shape of a Set of Points in the Plane
      \newblock {\em IEEE Transactions on Information Theory}, 29(4):551--559, 1983.
      
      \bibitem{RoussillonL11}
      [RL11] Tristan Roussillon and Jacques-Olivier Lachaud
      \newblock Delaunay Properties of Digital Straight Segments
      \newblock {\em Springer : Discrete Geometry for Computer Imagery - 16th {IAPR}}, 2011.
     \end{thebibliography}
    \scriptsize     
  \end{column}
\end{columns}  
%----- End biblio   -----

\end{frame}



\begin{frame}
\frametitle{Relations aux triangulations de Delaunay}

\end{frame}

%-----------------------------------------------------------------
\subsection{$\alpha$-shape, $\alpha \leq 0$ - Généralisation de Har-Peled}
%-----------------------------------------------------------------

\begin{frame}
\frametitle{$\alpha$-shape, $\alpha \leq 0$ - Généralisation de Har-Peled - Présentation}

\end{frame}

\begin{frame}
\frametitle{Algorithme}

\end{frame}

\begin{frame}
\frametitle{Résultats}

\end{frame}

%-----------------------------------------------------------------
\subsection{$\alpha$-shape, $\alpha \leq 0$}
%-----------------------------------------------------------------

\begin{frame}
\frametitle{$\alpha$-shape, $\alpha \leq 0$ - Présentation}

\end{frame}

\begin{frame}
\frametitle{Algorithme}

\end{frame}

\begin{frame}
\frametitle{Résultats}

\end{frame}


