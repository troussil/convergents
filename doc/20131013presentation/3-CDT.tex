\section{Triangulation de Delaunay d'ordre 0 (CDT)}

  \begin{frame}<beamer>
    \frametitle{Plan}
    \tableofcontents[currentsection]
  \end{frame}

% ----------------------------------------------------------------------
\begin{frame}
  	\frametitle{R�sultat}


%exemple %sch�ma
\includegraphics[width=0.22\linewidth,page=4]{fig/CDT-motif.pdf}\hspace{0.02\linewidth}
\includegraphics[width=0.22\linewidth,page=8]{fig/CDT-motif.pdf}\hspace{0.02\linewidth}
\includegraphics[width=0.45\linewidth]{fig/CDT-tree.pdf}

~

%ref
\scriptsize
\begin{thebibliography}{alpha}

\bibitem{RL11}
[RL11] Roussillon, T., Lachaud, J-O., 
\newblock Delaunay properties of digital straight segments
\newblock {\em Discrete Geometry and Computer Imagery},  308--319, 2011.

\end{thebibliography}


\end{frame}

% ----------------------------------------------------------------------
\begin{frame}
  	\frametitle{Sch�ma de preuve}

%EMST in CDT
\begin{block}{1. L'arbre recouvrant minimal (euclidien) divise la CDT en deux parties} 
\centering
\includegraphics[width=0.22\linewidth,page=2]{fig/CDT-motif.pdf}\hspace{0.02\linewidth}
\includegraphics[width=0.22\linewidth,page=3]{fig/CDT-motif.pdf}\hspace{0.02\linewidth}
\end{block}

%triangulation unique, donc de Delaunay
\begin{block}{2. La partie sup�rieure de l'enveloppe convexe se triangule de mani�re unique} 
\only<1>{
\begin{figure}[htbp]
\centering
\includegraphics[width=0.22\linewidth,page=5]{fig/CDT-motif.pdf}\hspace{0.02\linewidth}
\end{figure}
}
\only<2>{
\begin{figure}[htbp]
\centering
\includegraphics[width=0.22\linewidth,page=6]{fig/CDT-motif.pdf}\hspace{0.02\linewidth}
\end{figure}
}
\only<3>{
\begin{figure}[htbp]
\centering
\includegraphics[width=0.22\linewidth,page=7]{fig/CDT-motif.pdf}\hspace{0.02\linewidth}
\end{figure}
}
\end{block}

\end{frame}

% ----------------------------------------------------------------------
\begin{frame}
  	\frametitle{Application directe: $\alpha$-shape, $\alpha \in [-2;0]$}

\begin{block}{Calcul incr�mental et \emph{output-sensitive} 
de l'enveloppe convexe ($\alpha = 0$)}
%
Complexit� en $O(h \log r)$, o� $h = O(r^{2/3})$ est le nombre de sommets. 
%ref Har-peled
\scriptsize
\begin{thebibliography}{alpha}
\bibitem{H98b}
[H98] Har-Peled, S.
\newblock An output sensitive algorithm for discrete convex hulls
\newblock {\em Computational Geometry}, 10(2):125--138, 1998.
\end{thebibliography}
%
\end{block}


\begin{block}{G�n�ralisation pour $\alpha \leq 0$}
%explications orales ?
 \begin{itemize}
  \item calcul r�cursif � partir des ar�tes de l'enveloppe convexe
  \item calcul incr�mental � partir des convergents des ar�tes
 \end{itemize}
Complexit� en $O(h \log r)$. 

%ref stage
\scriptsize 
Stage de master de Thomas Lafond 
\end{block}

\end{frame}
