\documentclass{beamer}

\usefonttheme{professionalfonts}

\usepackage{times}
\usepackage[latin1]{inputenc} %franc
\usepackage[T1]{fontenc} %ang
\usepackage{xspace}
\usepackage[english,francais]{babel}

\usepackage[]{beamerthemeliris} % paquet theme, options: nogradient,nobackground


\usepackage{graphicx} %paquet graphiques
\usepackage{epsfig}
\usepackage{subfigure}
\graphicspath{{fig/}}

\usepackage{amsmath} %math
\usepackage{amssymb}

\title[GT GEODIS 13/10/2013]
{G�om�trie des motifs de droites discr�tes}
%\subtitle{}

\author[T. Roussillon]
{Tristan Roussillon}


\date{13/10/2013}

%rappel du sommaire 
%\AtBeginSection[]
%{
%  \begin{frame}<beamer>
%    \frametitle{Outline}
%    \tableofcontents[currentsection]
%  \end{frame}
%}


\begin{document}

% ----------------------------------------------------------------------
%slide de titre
\begin{frame}%[plain]
  \titlepage
  \begin{center}
  Journ�e du GT GEODIS
  \end{center}
\end{frame}

% ----------------------------------------------------------------------
%corps de la pr�sentation
\section{Motivation}

% ----------------------------------------------------------------------
\begin{frame}
  	\frametitle{Qu'est-ce que c'est ?}

\centering
\includegraphics[width=0.3\linewidth,page=1]{fig/disque-r.pdf}

\end{frame}

% ----------------------------------------------------------------------
\begin{frame}
  	\frametitle{Disque discret}

\begin{block}{Discr�tisation de Gauss d'un disque euclidien $D(C,r)$}
\centering
\includegraphics[width=0.3\linewidth,page=2]{fig/disque-r.pdf}
\end{block}

\begin{block}{D�finition analytique �quivalente}
\[
\{ (x,y) \in \mathbb{Z}^2 | (x - C_x)^2 + (y - C_y)^2 \leq r^2 \}
\]
\end{block}

\end{frame}

% ----------------------------------------------------------------------
\begin{frame}
  	\frametitle{Trois mod�les de disques discrets}

%% \begin{itemize}
%%   \item $D^e$: centre et rayon entiers
%%   \item $D^{e^2}$: centre et carr� du rayon entiers
%%   \item $D^r$: centre et carr� du rayon rationels
%% \end{itemize}

\begin{figure}[htbp]
\centering
\subfigure[$D^e(C,r)$]{\includegraphics[width=0.3\linewidth]{fig/disque-e.pdf}}\hspace{0.03\linewidth}
\subfigure[$D^{e^2}(C,r)$]{\includegraphics[width=0.3\linewidth]{fig/disque-e2.pdf}}\hspace{0.03\linewidth}
\subfigure[$D^r(C,r)$]{\includegraphics[width=0.3\linewidth,page=2]{fig/disque-r.pdf}}
\label{fig:modeles} 
\end{figure}

\end{frame}

% ----------------------------------------------------------------------
\begin{frame}
  	\frametitle{Motivation}

\begin{block}{Comprendre la distribution spatiale des points}
\centering
\includegraphics[width=0.3\linewidth]{fig/disque-alea.pdf}\hspace{0.1\linewidth}
\includegraphics[width=0.3\linewidth,page=2]{fig/disque-r.pdf}
\end{block}

\end{frame}

% ----------------------------------------------------------------------
\begin{frame}
  	\frametitle{Remarque}

\begin{block}{Seul les points du bord nous int�resse}
\centering
\includegraphics[width=0.3\linewidth,page=1]{fig/bord-vs-interior.pdf}\hspace{0.1\linewidth}
\includegraphics[width=0.3\linewidth,page=2]{fig/bord-vs-interior.pdf}
\end{block}

\end{frame}

% ----------------------------------------------------------------------
\begin{frame}
  	\frametitle{Un outils}

\only<1>{
\begin{block}{$\alpha$-hull, intersection de disques g�n�ralis�s}
\begin{figure}[htbp]
\centering
\subfigure[n�gatif]{\includegraphics[width=0.4\linewidth,page=1]{fig/alpha-shape.pdf}}\hspace{0.03\linewidth}
\subfigure[positif]{\includegraphics[width=0.4\linewidth,page=3]{fig/alpha-shape.pdf}}
\end{figure}
\end{block}
}

\only<2>{
\begin{block}{$\alpha$-shape, graphe}
\begin{figure}[htbp]
\centering
\subfigure[n�gatif]{\includegraphics[width=0.4\linewidth,page=2]{fig/alpha-shape.pdf}}\hspace{0.03\linewidth}
\subfigure[positif]{\includegraphics[width=0.4\linewidth,page=4]{fig/alpha-shape.pdf}}
\end{figure}
\end{block}
}

\scriptsize
\begin{thebibliography}{alpha}

\bibitem{EKS83}
[EKS83] Edelsbrunner, H., Kirkpatrick, D., Seidel, R.
\newblock On the Shape of a Set of Points in the Plane
\newblock {\em IEEE Transactions on Information Theory}, 29(4):551--559, 1983.

\end{thebibliography}

\end{frame}

% ----------------------------------------------------------------------
\begin{frame}
  	\frametitle{Int�r�t des $\alpha$-shape, $\alpha \in [-2;1/r_{min}[$}

%Les $\alpha$-shape captent et ordonnent 
% les relations de proximit� entre les points du bord. 

\includegraphics[width=0.3\linewidth,page=1]{fig/ex-alpha-shape.pdf}\hspace{0.03\linewidth}
\includegraphics[width=0.3\linewidth,page=2]{fig/ex-alpha-shape.pdf}\hspace{0.03\linewidth}
\includegraphics[width=0.3\linewidth,page=3]{fig/ex-alpha-shape.pdf}


\includegraphics[width=0.3\linewidth,page=4]{fig/ex-alpha-shape.pdf}\hspace{0.03\linewidth}
\includegraphics[width=0.3\linewidth,page=5]{fig/ex-alpha-shape.pdf}\hspace{0.03\linewidth}
\includegraphics[width=0.3\linewidth,page=6]{fig/ex-alpha-shape.pdf}

\end{frame}

% ----------------------------------------------------------------------
%\begin{frame}
%  	\frametitle{Nombre de sommets en fonction de $\alpha$}
%
%\begin{tabular}{|c|c|c|}
%\centering
%\hline \hline
%$\alpha$               & $\sharp$       & observation        \\ \hline \hline
%$-2$                   &  $O(r)$        & trivial            \\ \hline
%$-\sqrt{2}$            &  $O(r)$        & trivial            \\ \hline
%$-\epsilon$            &  $O(r^{2/3})$  & [H98], non observ� \\ \hline
%$0$                    &  $O(r^{2/3})$  & [H98]              \\ \hline
%$1/(r_{min}-\epsilon)$ &                &                    \\ \hline
%\hline
%\end{tabular}
%
%\scriptsize
%\begin{thebibliography}{alpha}
%
%\bibitem{H98}
%[H98] Har-Peled, S.
%\newblock An output sensitive algorithm for discrete convex hulls
%\newblock {\em Computational Geometry}, 10(2):125--138, 1998.
%
%\end{thebibliography}
%
%\end{frame}

% ----------------------------------------------------------------------
\begin{frame}
  	\frametitle{Triangulation de Delaunay d'ordre 0 et n}

\begin{block}{Union des $\alpha$-shapes, Delaunay, Vorono\"{i} }
\only<1>{
\begin{figure}[htbp]
\centering
\subfigure[n�gatif]{\includegraphics[width=0.45\linewidth,page=8]{fig/CDT-alpha-shape.pdf}}\hspace{0.03\linewidth}
\subfigure[positif]{\includegraphics[width=0.45\linewidth,page=2]{fig/FDT-alpha-shape.pdf}}
\end{figure}
}
\only<2>{
\begin{figure}[htbp]
\centering
\subfigure[n�gatif]{\includegraphics[width=0.45\linewidth,page=9]{fig/CDT-alpha-shape.pdf}}\hspace{0.03\linewidth}
\subfigure[positif]{\includegraphics[width=0.45\linewidth,page=3]{fig/FDT-alpha-shape.pdf}}
\end{figure}
}
\only<3>{
\begin{figure}[htbp]
\centering
\subfigure[n�gatif]{\includegraphics[width=0.45\linewidth,page=10]{fig/CDT-alpha-shape.pdf}}\hspace{0.03\linewidth}
\subfigure[positif]{\includegraphics[width=0.45\linewidth,page=4]{fig/FDT-alpha-shape.pdf}}
\end{figure}
}
\end{block}

\end{frame}

% ----------------------------------------------------------------------
\begin{frame}
  	\frametitle{Prenons un objet simple: le motif (1er octant) }

\begin{block}{Discr�tisation plancher d'un segment de droite }
\centering
\includegraphics[width=0.25\linewidth]{fig/motif.pdf}\hspace{0.1\linewidth}
\includegraphics[width=0.3\linewidth]{fig/motifs.pdf}
%\end{block}
%
%\begin{block}{D�finition analytique �quivalente}
\[
\{ (x,y) \in \mathbb{Z}^2 | 0 \leq ax + by < (a+b), \: 0 \leq x+y \leq (a+b) \}
\]
\end{block}

\alert{Triangulations de Delaunay d'ordre 0 et n d'un motif ?}

\end{frame}

\section{Triangulation de Delaunay d'ordre 0}

  \begin{frame}<beamer>
    \frametitle{Plan}
    \tableofcontents[currentsection]
  \end{frame}

% ----------------------------------------------------------------------
\begin{frame}
  	\frametitle{Un slide}


\end{frame}

\section{Triangulation de Delaunay d'ordre n}

  \begin{frame}<beamer>
    \frametitle{Plan}
    \tableofcontents[currentsection]
  \end{frame}

% ----------------------------------------------------------------------
\begin{frame}
  	\frametitle{Un slide}


\end{frame}


% ----------------------------------------------------------------------
%\begin{frame}
%  	\frametitle{Conclusion}

%-CDT et FDT d'un motif

%extension � un DSS, � un DSS+2
%d�termination de la classes des segments max. qui sont arcs max. 

%partie de motifs sur un cercle

%\end{frame}

% ----------------------------------------------------------------------
\begin{frame}
  	\frametitle{Merci}
\end{frame}

\end{document}


