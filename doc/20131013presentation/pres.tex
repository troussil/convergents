\documentclass{beamer}

\usefonttheme{professionalfonts}

\usepackage{times}
\usepackage[latin1]{inputenc} %franc
\usepackage[T1]{fontenc} %ang
\usepackage{xspace}
\usepackage[english,francais]{babel}

\usepackage[]{beamerthemeliris} % paquet theme, options: nogradient,nobackground


\usepackage{graphicx} %paquet graphiques
\usepackage{epsfig}
\usepackage{subfigure}
\graphicspath{{fig/}}

\usepackage{amsmath} %math
\usepackage{amssymb}

\title[GT GEODIS 13/10/2013]
{G�om�trie des motifs de droites discr�tes}
%\subtitle{}

\author[T. Roussillon]
{Tristan Roussillon}


\date{13/10/2013}

%rappel du sommaire 
%\AtBeginSection[]
%{
%  \begin{frame}<beamer>
%    \frametitle{Outline}
%    \tableofcontents[currentsection]
%  \end{frame}
%}


\begin{document}

% ----------------------------------------------------------------------
%slide de titre
\begin{frame}%[plain]
  \titlepage
  \begin{center}
  Journ�e du GT GEODIS
  \end{center}
\end{frame}

% ----------------------------------------------------------------------
%corps de la pr�sentation
\section{Motivation}

% ----------------------------------------------------------------------
\begin{frame}
  	\frametitle{Qu'est-ce que c'est ?}

\centering
\includegraphics[width=0.3\linewidth,page=1]{fig/disque-r.pdf}

\end{frame}

% ----------------------------------------------------------------------
\begin{frame}
  	\frametitle{Disque discret}

\begin{block}{Discr�tisation de Gauss d'un disque euclidien $D(C,r)$}
\centering
\includegraphics[width=0.3\linewidth,page=2]{fig/disque-r.pdf}
\end{block}

\begin{block}{D�finition analytique �quivalente}
\[
\{ (x,y) \in \mathbb{Z}^2 | (x - C_x)^2 + (y - C_y)^2 \leq r^2 \}
\]
\end{block}

\end{frame}

% ----------------------------------------------------------------------
\begin{frame}
  	\frametitle{Trois mod�les de disques discrets}

%% \begin{itemize}
%%   \item $D^e$: centre et rayon entiers
%%   \item $D^{e^2}$: centre et carr� du rayon entiers
%%   \item $D^r$: centre et carr� du rayon rationels
%% \end{itemize}

\begin{figure}[htbp]
\centering
\subfigure[$D^e(C,r)$]{\includegraphics[width=0.3\linewidth]{fig/disque-e.pdf}}\hspace{0.03\linewidth}
\subfigure[$D^{e^2}(C,r)$]{\includegraphics[width=0.3\linewidth]{fig/disque-e2.pdf}}\hspace{0.03\linewidth}
\subfigure[$D^r(C,r)$]{\includegraphics[width=0.3\linewidth,page=2]{fig/disque-r.pdf}}
\label{fig:modeles} 
\end{figure}

\end{frame}

% ----------------------------------------------------------------------
\begin{frame}
  	\frametitle{Motivation}

\begin{block}{Comprendre la distribution spatiale des points}
\centering
\includegraphics[width=0.3\linewidth]{fig/disque-alea.pdf}\hspace{0.1\linewidth}
\includegraphics[width=0.3\linewidth,page=2]{fig/disque-r.pdf}
\end{block}

\end{frame}

% ----------------------------------------------------------------------
\begin{frame}
  	\frametitle{Remarque}

\begin{block}{Seul les points du bord nous int�resse}
\centering
\includegraphics[width=0.3\linewidth,page=1]{fig/bord-vs-interior.pdf}\hspace{0.1\linewidth}
\includegraphics[width=0.3\linewidth,page=2]{fig/bord-vs-interior.pdf}
\end{block}

\end{frame}

% ----------------------------------------------------------------------
\begin{frame}
  	\frametitle{Un outils}

\only<1>{
\begin{block}{$\alpha$-hull, intersection de disques g�n�ralis�s}
\begin{figure}[htbp]
\centering
\subfigure[n�gatif]{\includegraphics[width=0.4\linewidth,page=1]{fig/alpha-shape.pdf}}\hspace{0.03\linewidth}
\subfigure[positif]{\includegraphics[width=0.4\linewidth,page=3]{fig/alpha-shape.pdf}}
\end{figure}
\end{block}
}

\only<2>{
\begin{block}{$\alpha$-shape, graphe}
\begin{figure}[htbp]
\centering
\subfigure[n�gatif]{\includegraphics[width=0.4\linewidth,page=2]{fig/alpha-shape.pdf}}\hspace{0.03\linewidth}
\subfigure[positif]{\includegraphics[width=0.4\linewidth,page=4]{fig/alpha-shape.pdf}}
\end{figure}
\end{block}
}

\scriptsize
\begin{thebibliography}{alpha}

\bibitem{EKS83}
[EKS83] Edelsbrunner, H., Kirkpatrick, D., Seidel, R.
\newblock On the Shape of a Set of Points in the Plane
\newblock {\em IEEE Transactions on Information Theory}, 29(4):551--559, 1983.

\end{thebibliography}

\end{frame}

% ----------------------------------------------------------------------
\begin{frame}
  	\frametitle{Int�r�t des $\alpha$-shape, $\alpha \in [-2;1/r_{min}[$}

%Les $\alpha$-shape captent et ordonnent 
% les relations de proximit� entre les points du bord. 

\includegraphics[width=0.3\linewidth,page=1]{fig/ex-alpha-shape.pdf}\hspace{0.03\linewidth}
\includegraphics[width=0.3\linewidth,page=2]{fig/ex-alpha-shape.pdf}\hspace{0.03\linewidth}
\includegraphics[width=0.3\linewidth,page=3]{fig/ex-alpha-shape.pdf}


\includegraphics[width=0.3\linewidth,page=4]{fig/ex-alpha-shape.pdf}\hspace{0.03\linewidth}
\includegraphics[width=0.3\linewidth,page=5]{fig/ex-alpha-shape.pdf}\hspace{0.03\linewidth}
\includegraphics[width=0.3\linewidth,page=6]{fig/ex-alpha-shape.pdf}

%\begin{figure}[htbp]
%\centering
%\subfigure[]{\includegraphics[width=0.3\linewidth,page=1]{fig/ex-alpha-shape.pdf}}\hspace{0.03\linewidth}
%\subfigure[$-2$]{\includegraphics[width=0.3\linewidth,page=2]{fig/ex-alpha-shape.pdf}}\hspace{0.03\linewidth}
%\subfigure[$-\sqrt{2}$]{\includegraphics[width=0.3\linewidth,page=3]{fig/ex-alpha-shape.pdf}}
%\subfigure[$-\epsilon$]{\includegraphics[width=0.3\linewidth,page=4]{fig/ex-alpha-shape.pdf}}\hspace{0.03\linewidth}
%\subfigure[$0$]{\includegraphics[width=0.3\linewidth,page=5]{fig/ex-alpha-shape.pdf}}\hspace{0.03\linewidth}
%\subfigure[$1/(r_{min}-\epsilon)$]{\includegraphics[width=0.3\linewidth,page=6]{fig/ex-alpha-shape.pdf}}
%\end{figure}

\end{frame}

% ----------------------------------------------------------------------
\begin{frame}
  	\frametitle{Nombre de sommets en fonction de $\alpha$}

\begin{tabular}{|c|c|c|}
%\centering
\hline \hline
$\alpha$               & $\sharp$       & observation        \\ \hline \hline
$-2$                   &  $O(r)$        & trivial            \\ \hline
$-\sqrt{2}$            &  $O(r)$        & trivial            \\ \hline
$-\epsilon$            &  $O(r^{2/3})$  & [H98], non observ� \\ \hline
$0$                    &  $O(r^{2/3})$  & [H98]              \\ \hline
%$1/(r_{min}-\epsilon)$ &                &                    \\ \hline
\hline
\end{tabular}

\scriptsize
\begin{thebibliography}{alpha}

\bibitem{H98}
[H98] Har-Peled, S.
\newblock An output sensitive algorithm for discrete convex hulls
\newblock {\em Computational Geometry}, 10(2):125--138, 1998.

\end{thebibliography}

\end{frame}

% ----------------------------------------------------------------------
\begin{frame}
  	\frametitle{Triangulation de Delaunay d'ordre 0 et n}

\begin{block}{Union des $\alpha$-shapes, Delaunay, Vorono\"{i} }
\only<1>{
\begin{figure}[htbp]
\centering
\subfigure[$\alpha \leq 0$]{\includegraphics[width=0.45\linewidth,page=8]{fig/CDT-alpha-shape.pdf}}\hspace{0.03\linewidth}
\subfigure[$\alpha \geq 0$]{\includegraphics[width=0.45\linewidth,page=2]{fig/FDT-alpha-shape.pdf}}
\end{figure}
}
\only<2>{
\begin{figure}[htbp]
\centering
\subfigure[$\alpha \leq 0$]{\includegraphics[width=0.45\linewidth,page=9]{fig/CDT-alpha-shape.pdf}}\hspace{0.03\linewidth}
\subfigure[$\alpha \geq 0$]{\includegraphics[width=0.45\linewidth,page=2]{fig/FDT-alpha-shape.pdf}}
\end{figure}
}
\only<3>{
\begin{figure}[htbp]
\centering
\subfigure[$\alpha \leq 0$]{\includegraphics[width=0.45\linewidth,page=10]{fig/CDT-alpha-shape.pdf}}\hspace{0.03\linewidth}
\subfigure[$\alpha \geq 0$]{\includegraphics[width=0.45\linewidth,page=3]{fig/FDT-alpha-shape.pdf}}
\end{figure}
}
\end{block}

\end{frame}

% ----------------------------------------------------------------------
\begin{frame}
  	\frametitle{Prenons un objet simple: le motif (1er octant) }

\begin{block}{Discr�tisation plancher d'un segment de droite }
\centering
\includegraphics[width=0.25\linewidth]{fig/motif.pdf}\hspace{0.1\linewidth}
\includegraphics[width=0.3\linewidth]{fig/motifs.pdf}
%\end{block}
%
%\begin{block}{D�finition analytique �quivalente}
\[
\{ (x,y) \in \mathbb{Z}^2 | \mu \leq ax + by < \mu + (a+b), \: \nu \leq x+y \leq \nu + (a+b) \}
\]
\end{block}

\alert{Triangulations de Delaunay d'ordre 0 et n d'un motif ?}

\end{frame}

\section{Triangulation de Delaunay d'ordre 0 (CDT)}

  \begin{frame}<beamer>
    \frametitle{Plan}
    \tableofcontents[currentsection]
  \end{frame}

% ----------------------------------------------------------------------
\begin{frame}
  	\frametitle{R�sultat}


%exemple %sch�ma
\includegraphics[width=0.22\linewidth,page=4]{fig/CDT-motif.pdf}\hspace{0.02\linewidth}
\includegraphics[width=0.22\linewidth,page=8]{fig/CDT-motif.pdf}\hspace{0.02\linewidth}
\includegraphics[width=0.45\linewidth]{fig/CDT-tree.pdf}

~

%ref
\scriptsize
\begin{thebibliography}{alpha}

\bibitem{RL11}
[RL11] Roussillon, T., Lachaud, J-O., 
\newblock Delaunay properties of digital straight segments
\newblock {\em Discrete Geometry and Computer Imagery},  308--319, 2011.

\end{thebibliography}


\end{frame}

% ----------------------------------------------------------------------
\begin{frame}
  	\frametitle{Sch�ma de preuve}

%EMST in CDT
\begin{block}{1. L'arbre recouvrant minimal (euclidien) divise la CDT en deux parties} 
\centering
\includegraphics[width=0.22\linewidth,page=2]{fig/CDT-motif.pdf}\hspace{0.02\linewidth}
\includegraphics[width=0.22\linewidth,page=3]{fig/CDT-motif.pdf}\hspace{0.02\linewidth}
\end{block}

%triangulation unique, donc de Delaunay
\begin{block}{2. La partie sup�rieure de l'enveloppe convexe se triangule de mani�re unique} 
\only<1>{
\begin{figure}[htbp]
\centering
\includegraphics[width=0.22\linewidth,page=5]{fig/CDT-motif.pdf}\hspace{0.02\linewidth}
\end{figure}
}
\only<2>{
\begin{figure}[htbp]
\centering
\includegraphics[width=0.22\linewidth,page=6]{fig/CDT-motif.pdf}\hspace{0.02\linewidth}
\end{figure}
}
\only<3>{
\begin{figure}[htbp]
\centering
\includegraphics[width=0.22\linewidth,page=7]{fig/CDT-motif.pdf}\hspace{0.02\linewidth}
\end{figure}
}
\end{block}

\end{frame}

% ----------------------------------------------------------------------
\begin{frame}
  	\frametitle{Application directe: $\alpha$-shape, $\alpha \in [-2;0]$}

\begin{block}{Calcul incr�mental et \emph{output-sensitive} 
de l'enveloppe convexe ($\alpha = 0$)}
%
Complexit� en $O(h \log r)$, o� $h = O(r^{2/3})$ est le nombre de sommets. 
%ref Har-peled
\scriptsize
\begin{thebibliography}{alpha}
\bibitem{H98b}
[H98] Har-Peled, S.
\newblock An output sensitive algorithm for discrete convex hulls
\newblock {\em Computational Geometry}, 10(2):125--138, 1998.
\end{thebibliography}
%
\end{block}


\begin{block}{G�n�ralisation pour $\alpha \leq 0$}
%explications orales ?
 \begin{itemize}
  \item calcul r�cursif � partir des ar�tes de l'enveloppe convexe
  \item calcul incr�mental � partir des convergents des ar�tes
 \end{itemize}
Complexit� en $O(h \log r)$. 

%ref stage
\scriptsize 
Stage de master de Thomas Lafond 
\end{block}

\end{frame}

\newcommand{\vect}[1]{\ensuremath{\overrightarrow{#1}}}
\newcommand{\conj}[1]{\ensuremath{\overline{#1}}}
%
\section{Triangulation de Delaunay d'ordre n (FDT)}

  \begin{frame}<beamer>
    \frametitle{Plan}
    \tableofcontents[currentsection]
  \end{frame}

% ----------------------------------------------------------------------
\begin{frame}
  	\frametitle{R�sultat}

\centering
\includegraphics[width=0.4\linewidth,page=1]{fig/FDT-schema.pdf}\hspace{0.05\linewidth}
\includegraphics[width=0.5\linewidth]{fig/FDT-tree.pdf}
%exemple ?

\begin{block}{Topologie}
Soit $n = O(\max{(a,b)})$, la profondeur de d�veloppement 
en fractions continues de $a/b$, la pente du motif. 
\scriptsize
 \begin{itemize}
  \item $n+2$ sommets
  \item $2n+1$ ar�tes ($n+2$ au bord et $n-1$ internes)
  \item $n$ faces
 \end{itemize}
\end{block}

~

%ref
\scriptsize
\begin{thebibliography}{alpha}

\bibitem{R13}
[R13] Roussillon, T.,  
\newblock Euclidean Farthest-Point Vorono\"{i} Diagram of a Pattern
\newblock {\em soumis � Discrete Applied Mathematics}, 2013.

\end{thebibliography}

\end{frame}

% ----------------------------------------------------------------------
\begin{frame}
  	\frametitle{Exemples}

\only<1>{
\begin{figure}[htbp]
\centering
\includegraphics[width=1\linewidth]{fig/29_12.pdf}
\end{figure}
}
\only<2>{
\begin{figure}[htbp]
\centering
\includegraphics[width=0.7\linewidth,page=1]{fig/8_5x2.pdf}
\end{figure}
}


\end{frame}


% ----------------------------------------------------------------------
\begin{frame}
  	\frametitle{Retour sur l'algorithme d'Euclide}

%equations
Soit $\theta(z_n) = p_n/q_n = [u_0, u_1, \ldots, u_n]$
($q_n > p_n > 0$ sont des entiers premiers entre eux). 

\begin{equation}
\label{eq:rec-rem}
\begin{array}{l}
  u_0 = 0, \quad r_{-1} = q_n, \: r_0 = p_n, \\
  \forall 1 \leq k \leq n, \:
  u_k = \bigg\lfloor\frac{r_{k-2}}{r_{k-1}}\bigg\rfloor, \:
  r_k = r_{k-2} - u_k r_{k-1}. \\
\end{array}
\end{equation}

Les restes sont positifs et d�croissants.  
\begin{equation}
\label{eq:rem-order}
r_{-1} = q_n > p_n = r_{0} > \ldots > r_{n-1} = 1 > r_n = 0. 
\end{equation}


Les convergents $z_k$ sont d�finis tels que $\forall 0 \leq k \leq n$, 
$p_k/q_k$ $=$ $[u_0, u_1, \ldots, u_k]$. 

\begin{equation}
\label{eq:rec-conv}
\begin{array}{l}
  z_0 = 1, \: z_{-1} = i, \\
  \forall 1 \leq k \leq n, \:
  z_k = z_{k-2} + u_k z_{k-1}. \\
\end{array}
 \quad
%
\end{equation}

\end{frame}

% ----------------------------------------------------------------------
\begin{frame}
  	\frametitle{Interpr�tation g�om�trique de l'algorithme d'Euclide}

\only<1>{
\begin{figure}[htbp]
\centering
\includegraphics[width=0.5\linewidth,page=2]{interpretation-euclid.pdf}
\end{figure}
}
\only<2>{
\begin{figure}[htbp]
\centering
\includegraphics[width=0.5\linewidth,page=3]{interpretation-euclid.pdf}
\end{figure}
}
\only<3>{
\begin{figure}[htbp]
\centering
\includegraphics[width=0.5\linewidth,page=4]{interpretation-euclid.pdf}
\end{figure}
}
\only<4>{
\begin{figure}[htbp]
\centering
\includegraphics[width=0.5\linewidth,page=5]{interpretation-euclid.pdf}
\end{figure}
}
\only<5>{
\begin{figure}[htbp]
\centering
\includegraphics[width=0.5\linewidth,page=6]{interpretation-euclid.pdf}
\end{figure}
}
\only<6>{
\begin{figure}[htbp]
\centering
\includegraphics[width=0.5\linewidth,page=7]{interpretation-euclid.pdf}
\end{figure}
}
\only<7>{
\begin{figure}[htbp]
\centering
\includegraphics[width=0.5\linewidth,page=8]{interpretation-euclid.pdf}
\end{figure}
}
\only<8>{
\begin{figure}[htbp]
\centering
\includegraphics[width=0.5\linewidth,page=9]{interpretation-euclid.pdf}
\end{figure}
}
\only<9>{
\begin{figure}[htbp]
\centering
\includegraphics[width=0.5\linewidth,page=10]{interpretation-euclid.pdf}
\end{figure}
}

\begin{equation}
\label{eq:rem-conv}
\forall 0 \leq k \leq n, \:
r_k z_{k-1} + r_{k-1} z_k = z_n.
\end{equation}

\end{frame}

% ----------------------------------------------------------------------
\begin{frame}
  	\frametitle{Convergents et enveloppe convexe: observation}

%convergents + sommets et aretes enveloppes convexes
\begin{figure}[htbp]
\centering
\includegraphics[width=0.4\linewidth,page=1]{8_5_Zk2.pdf}\hspace{0.05\linewidth}
\includegraphics[width=0.4\linewidth,page=2]{8_5_Zk2.pdf}

\includegraphics[width=0.4\linewidth,page=3]{8_5_Zk2.pdf}\hspace{0.05\linewidth}
\includegraphics[width=0.4\linewidth,page=1]{8_5_Hk.pdf}
\end{figure}

\end{frame}

% ----------------------------------------------------------------------
\begin{frame}
  	\frametitle{Convergents et enveloppe convexe: id�e de la preuve}

\begin{figure}[htbp]
\centering
\includegraphics[width=0.4\linewidth,page=5]{8_5_Zk2.pdf}\hspace{0.05\linewidth}
\includegraphics[width=0.4\linewidth,page=2]{8_5_Hk.pdf}
\end{figure}

%convergents, cones vides / enveloppes convexes
%
\begin{equation}
\label{eq:det-k1}
\forall 0 \leq k \leq n \quad 
q_kp_{k-1} - p_{k-1}q_k = (-1)^k.
\end{equation}
%
\begin{equation}
\label{eq:det-k2}
\forall 1 \leq k \leq n \quad 
q_kp_{k-2} - p_{k-2}q_k = -u_k(-1)^k.
\end{equation}
%
\begin{equation}
\label{eq:det-rem}
\forall -1 \leq k \leq n \quad 
q_np_k - p_kq_n = r_k(-1)^k.
\end{equation}


\end{frame}

% ----------------------------------------------------------------------
\begin{frame}
  	\frametitle{D�finition de la triangulation}

%ordre
\scriptsize
$(H_0, O, Z_n)$, $(H_1, H_0, O)$, $\ldots$, 
$(H_k, H_{k-1}, H_{k-2})$, $\ldots$, 
$(H_{n-1}, H_{n-2}, H_{n-3})$. 
\normalsize

\begin{figure}
\centering
\includegraphics[width=0.4\linewidth,page=2]{fig/FDT-schema-bis.pdf}\hspace{0.05\linewidth}
\end{figure}

\begin{block}{Id�e de la preuve}
 \begin{itemize}
  \item I. c'est une triangulation
  \item II. le cercle circonscrit de chaque triangle contient tous les points
  \begin{enumerate}
   \item \alert{deux triangles cons�cutifs partagent une ar�te; le cercle circonscrit 
du second contient le sommet oppos� du premier}
   \item conclure par r�currence 
  \end{enumerate}
 \end{itemize}
\end{block}

\end{frame}

% ----------------------------------------------------------------------
\begin{frame}
  	\frametitle{Exemple ($k < n$, $k$ impair)}

\begin{block}{Trois �quivalences}
 \begin{enumerate}
  \item Le cercle passant par $H_k$, $H_{k-1}$, $H_{k-2}$ contient $H_{k-3}$.
  \item $(\vect{H_{k-2}H_{k}},\vect{H_{k-2}H_{k-1}}) < (\vect{H_{k-3}H_k},\vect{H_{k-3}H_{k-1}})$
  \item \scriptsize{
$x = (z_n - z_{k-1} - z_{k-2})\conj{z_{k-1}}$,
$y = \conj{(z_n - z_{k} - z_{k-3})}z_{k-2}$ }
\normalsize{ et $\theta(x) < \theta(y)$.} 
 \end{enumerate}
\end{block}

\centering
\includegraphics[width=0.3\linewidth]{angles.pdf}

\begin{block}{3 est vrai car: }
 \begin{itemize}
  \item (parties r�elles) $\Re(x) > \Re(y) > 0$, par (\ref{eq:rec-conv}). 
  \item (parties imaginaires) 
Par (\ref{eq:rec-conv}), (\ref{eq:rem-conv}), 
(\ref{eq:det-k1}), (\ref{eq:det-k2}):
  \begin{itemize}
   \item $\Im(x) = (r_{k-1} - 1)$.
   \item $\Im(y) = (r_{k-1} - 1)u_k + (r_k - 1)$. 
   \item D'o� $\Im(y) \geq \Im(x) > 0$, par (\ref{eq:rem-order}). 
  \end{itemize}
 \end{itemize}
\end{block}

\end{frame}

%avant conclusion uni-modularit�



% ----------------------------------------------------------------------
\begin{frame}
  	\frametitle{Conclusion}

%-CDT et FDT d'un motif

%extension � un DSS, � un DSS+2
%d�termination de la classes des segments max. qui sont arcs max. 

%partie de motifs sur un cercle

\end{frame}

\end{document}


