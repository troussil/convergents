%------------------------------------------------
\section{Motivations}
%------------------------------------------------

%-----------------------------------------------------------------
\subsection{Disques discret}
%-----------------------------------------------------------------

\begin{frame}
\frametitle{Construction de la problématique}
  \only<1>
  {
    \begin{block}{Définition du disque Euclidien}
      $$\mathcal{D} =  \left\{ (x,y) \in \mathbb{R}^{2} |  ax + by + c(x^2 + y^2) + d \leq 0. \right\}$$
    \end{block}
    
    \begin{figure}[h!]
      \centering
      \includegraphics[width=4cm]{fig/2-motivation/circle/mot-cercle-euc.pdf}
    \end{figure}
        
  }
  \only<2->
  {
    \begin{block}{Définition du disque discret}
      $$\mathcal{D} =  \left\{ (x,y) \in \alert{\mathbb{Z}^{2}} |  ax + by + c(x^2 + y^2) + d \leq 0. \right\}$$
    \end{block}
  }
  \only<2>
  {  
    \begin{figure}[h!]
      \centering
      \includegraphics[width=4cm]{fig/2-motivation/circle/mot-cercle-dis.pdf}
    \end{figure}
  }
  
  \begin{columns}[t]
    \begin{column}{6cm}
      \only<3>
      {
        \begin{figure}[h!]
          \centering
          \includegraphics[width=4cm]{fig/2-motivation/circle/mot-cercle-dis.pdf}
        \end{figure}
      }
      \only<4>
      {
        \begin{figure}[h!]
          \centering
          \includegraphics[width=4cm]{fig/2-motivation/circle/mot-cercle-dis-2.pdf}
        \end{figure}
      }
      \only<5,6>
      {
        \begin{figure}[h!]
          \centering
          \includegraphics[width=4cm]{fig/2-motivation/circle/mot-cercle-dis-3.pdf}
        \end{figure}
      }          
      \end{column}
      \begin{column}{5cm}
        \vspace{-0.4cm}
        
        \only<3>
        {
          \begin{block}{Remarques}
            1. On peut facilement connaître la position d'un point par rapport au disque.\\
            2. La définition n'est pas suffisante pour représenter tous les bords de disques discrets.
          \end{block}
        }
        \only<4,5>
        {
          \begin{block}{Remarques}
            Les points sont ordonnées sur une structure régulière.
          \end{block}
        }
        \only<5,6>
        {
          \begin{block}{Conséquences}
            Se concentrer sur l'étude des points du bords.
          \end{block}
        }
        \only<6>
        {
          \begin{block}{Question :}
            \begin{center}  
              \alert{Comment sont représentés les bords des disques discrets dans $\mathbb{Z}^{2}$ ?}
            \end{center}
          \end{block}
        }
      \end{column}
    \end{columns}
\end{frame}

%-----------------------------------------------------------------
\subsection{Alpha-Shape}
%-----------------------------------------------------------------

\begin{frame}
  \frametitle{Un outil, l'$\alpha$-shape}
  \begin{block}{Définition}
    \begin{itemize}
      \item $\alpha$-hull : Intersection de tous les disques généralisés de rayon $1/\alpha$ qui contiennent tous les points de l'ensemble.
      \item Point $\alpha$-extreme : Point appartenant au bord de l'$\alpha$-hull.
      \item $\alpha$-shape : Enveloppe reliant tous les $\alpha$-extremes adjacents.
    \end{itemize}
\begin{tiny}
  [EKS83] Edelsbrunner, H., Kirkpatrick, D., Seidel, R.\\
  On the Shape of a Set of Points in the Plane\\
  {\em IEEE Transactions on Information Theory}, 29(4):551--559, 1983.\\
\end{tiny}   
  \end{block}
  \vspace{-1cm}
  \only<1>
  { 
    \begin{columns}[t]
      \begin{column}{5cm}
        \begin{figure}[h!]
          \centering
          \includegraphics[width=\linewidth,page=1]{fig/2-motivation/mot-alpha-shape.pdf}
         \end{figure}
       \end{column}
       \begin{column}{5cm}
         \begin{figure}[h!]
           \centering
           \includegraphics[width=\linewidth,page=3]{fig/2-motivation/mot-alpha-shape.pdf}
         \end{figure}
       \end{column}
    \end{columns} 
  }
  \only<2>
  { 
    \begin{columns}[t]
      \begin{column}{5cm}
        \begin{figure}[h!]
          \centering
          \includegraphics[width=\linewidth,page=2]{fig/2-motivation/mot-alpha-shape.pdf}
         \end{figure}
       \end{column}
       \begin{column}{5cm}
         \begin{figure}[h!]
           \centering
           \includegraphics[width=\linewidth,page=4]{fig/2-motivation/mot-alpha-shape.pdf}
         \end{figure}
       \end{column}
    \end{columns} 
  }
\end{frame}

%-----------------------------------------------------------------

\begin{frame}
  \frametitle{Les $\alpha$-shapes de disques discrets}
 
  \begin{columns}[t]
    \begin{column}{4cm}
      \begin{exampleblock}{Cercle}
        $\mathcal{C} \left( (24,-8), (16,1), (0,0) \right)$\\
         
        \only<1>
        {
          Centre : (8.86, -13.4)\\
          R : 16.06
        }
        \only<2>
        {
          $\alpha = -\sqrt{2}$\\
           Nb Sommets : 126\\
        }
        \only<3>
        {
          $\alpha = -1$\\
          Nb Sommets : 89\\
        }
        \only<4>
        {
          $\alpha = -1/\sqrt{10}$\\
           Nb Sommets : 71\\
        }
        \only<5>
        {
          $\alpha = -1/\sqrt{20}$\\
           Nb Sommets : 64\\
        }
        \only<6>
        {
          $\alpha = -1/\sqrt{80}$\\
           Nb Sommets : 61\\
        }
        \only<7>
        {
          $\alpha = 0$\\
           Nb Sommets : 23\\
        }
        \only<8>
        {
          $\alpha = 1/\sqrt{600}$\\
           Nb Sommets : 21\\
        }
        \only<9>
        {
          $\alpha = 1/\sqrt{400}$\\
           Nb Sommets : 19\\
        }
        \only<10>
        {
          $\alpha = 1/\sqrt{300}$\\
           Nb Sommets : 11\\
        }
        \only<11>
        {
          $\alpha = 1/\sqrt{260}$\\
           Nb Sommets : 6\\
        }
      \end{exampleblock}
     
      \only<12>
      {
        \begin{block}{}
          Les $\alpha$-shapes récupèrent un beau panel de distribution de points composants le bord de disques discrets.
        \end{block}
      }
      
    \end{column}

    \begin{column}{6cm}
      \vspace{-0.8cm}     
      \only<1>
      {
        \begin{figure}[h!]
          \centering
          \includegraphics[width=5.5cm]{fig/2-motivation/alpha-shape-circle/as-0.pdf}
        \end{figure}
      }
      \only<2>
      {
        \begin{figure}[h!]
          \centering
          \includegraphics[width=5.5cm]{fig/2-motivation/alpha-shape-circle/as-1.pdf}
        \end{figure}
      }
      \only<3>
      {
        \begin{figure}[h!]
          \centering
          \includegraphics[width=5.5cm]{fig/2-motivation/alpha-shape-circle/as-2.pdf}
        \end{figure}
      }
      \only<4>
      {
        \begin{figure}[h!]
          \centering
          \includegraphics[width=5.5cm]{fig/2-motivation/alpha-shape-circle/as-3.pdf}
        \end{figure}
      }
      \only<5>
      {
        \begin{figure}[h!]
          \centering
          \includegraphics[width=5.5cm]{fig/2-motivation/alpha-shape-circle/as-4.pdf}
        \end{figure}
      }
      \only<6>
      {
        \begin{figure}[h!]
          \centering
          \includegraphics[width=5.5cm]{fig/2-motivation/alpha-shape-circle/as-5.pdf}
        \end{figure}
      }
      \only<7>
      {
        \begin{figure}[h!]
          \centering
          \includegraphics[width=5.5cm]{fig/2-motivation/alpha-shape-circle/as-6.pdf}
        \end{figure}
      }
      \only<8>
      {
        \begin{figure}[h!]
          \centering
          \includegraphics[width=5.5cm]{fig/2-motivation/alpha-shape-circle/as-7.pdf}
        \end{figure}
      }
      \only<9>
      {
        \begin{figure}[h!]
          \centering
          \includegraphics[width=5.5cm]{fig/2-motivation/alpha-shape-circle/as-8.pdf}
        \end{figure}
      }
      \only<10>
      {
        \begin{figure}[h!]
          \centering
          \includegraphics[width=5.5cm]{fig/2-motivation/alpha-shape-circle/as-9.pdf}
        \end{figure}
      }
      \only<11,12>
      {
        \begin{figure}[h!]
          \centering
          \includegraphics[width=5.5cm]{fig/2-motivation/alpha-shape-circle/as-10.pdf}
        \end{figure}
      }

    \end{column}
  \end{columns}
\end{frame}

%-----------------------------------------------------------------

\begin{frame}
  \frametitle{Remarques}
  \only<1>
  {
    \begin{columns}[t]
      \begin{column}{5cm}
        \begin{figure}[h!]
          \centering
          \includegraphics[width=\linewidth,page=8]{fig/2-motivation/circle/mot-union-neg.pdf}
         \end{figure}
       \end{column}
       \begin{column}{5cm}
         \begin{figure}[h!]
           \centering
           \includegraphics[width=\linewidth,page=2]{fig/2-motivation/circle/mot-union-pos.pdf}
         \end{figure}
       \end{column}
    \end{columns}
    
    \begin{block}{}
      La construction d'une $\alpha$-shape s'appuie sur les triangulations de Delaunay.    
    \end{block}
  }
  \only<2>
  {
    \begin{columns}[t]
      \begin{column}{5cm}
        \begin{figure}[h!]
          \centering
          \includegraphics[width=\linewidth]{fig/2-motivation/alpha-shape-circle/as-1.pdf}
         \end{figure}
       \end{column}
       \begin{column}{5cm}
         \begin{figure}[h!]
           \centering
           \includegraphics[width=\linewidth]{fig/2-motivation/alpha-shape-circle/as-2.pdf}
         \end{figure}
       \end{column}
    \end{columns}
    
    \begin{block}{Suivi de bord}
      Pour $\alpha = -\sqrt{2}$ et $\alpha = -1$ on réalise un suivi de bord 4/8-connexes.   
    \end{block}
  }
\end{frame}

