%-----------------------------------------------------------------
\section{Poursuite}
%-----------------------------------------------------------------

%-----------------------------------------------------------------
\subsection{Poursuite du projet}
%-----------------------------------------------------------------

\begin{frame}
\frametitle{Poursuite du projet}

  \begin{block}{Amélioration}
    \begin{itemize}
      \item Approche ascendante, top-down pour le calcul de l'$\alpha$ shape quand $\alpha > 0$.
    \end{itemize}
  \end{block}
  \begin{block}{Augmenter les points de comparaison}
    \begin{itemize}
      \item Approche récursive pour $\alpha <0$ basée sur une décomposition par arête.
      \item Utiliser d'autres formes discrètes : Ellipse.
    \end{itemize}
  \end{block}
  \begin{block}{Utilisation des algorithmes}
    \begin{itemize}
      \item Utiliser les algorithmes Output Sensitive pour la reconnaissance de disques discrets.
    \end{itemize}
  \end{block}


\end{frame}

%-----------------------------------------------------------------
\subsection{Poursuite personnelle}
%-----------------------------------------------------------------

\begin{frame}
\frametitle{Poursuite personnelle}
  \begin{block}{Domaine scientifique}
    \begin{itemize}
			\item Mathématiques appliquées.
			\item Informatiques.
		\end{itemize}	
  \end{block} 
  
    \begin{block}{Contexte}
    \begin{itemize}
      \item Travaux de recherche.
			\item Intérêt pour l'enseignement.
		\end{itemize}	
  \end{block} 
  
  \begin{exampleblock}{}
    Langage C++.
  \end{exampleblock} 
\end{frame}

\subsection{Thanks}
\begin{frame}
  \begin{block}{}
    \begin{center}
      Merci !
    \end{center}

  \end{block}
\end{frame}
