%-----------------------------------------------------------------
\section{Contributions}
%-----------------------------------------------------------------

%-----------------------------------------------------------------
\subsection{Généralisation de Har-Peled - $\alpha \leq 0$}
%-----------------------------------------------------------------

\begin{frame}
  \frametitle{$\alpha$-hull avec $\alpha \leq 0$}
  \begin{block}{Définition}
		L'intersection de tous les complémentaires fermés de disques de rayon -1/$\alpha$ qui contiennent tous les points à l’intérieur du disque.
	\end{block} 

  \begin{block}{Principe de l'algorithme.}
		\begin{itemize}
				\item Suivre l'algorithme de l'enveloppe convexe.
				\item Contrôler si les convergents sont des points $\alpha$-extremes.
		\end{itemize}	
	\end{block} 

\end{frame}

%-----------------------------------------------------------------

\begin{frame}
  \frametitle{Départ}
	\begin{columns}[t]
 		\begin{column}{7cm}
      \only<1>
      {
        \begin{figure}[h!]
          \centering
          \includegraphics[trim = 3cm 0cm 10cm 6cm, clip, width=7cm]{fig/as-algo/as-algo-1.pdf}
      \end{figure}
      }
      \only<2>
      {
        \begin{figure}[h!]
          \centering
          \includegraphics[trim = 3cm 0cm 10cm 6cm, clip, width=7cm]{fig/as-algo/as-algo-3.pdf}
      \end{figure}
      }
      \only<3>
      {
        \begin{figure}[h!]
          \centering
          \includegraphics[trim = 3cm 0cm 10cm 6cm, clip, width=7cm]{fig/as-algo/as-algo-4.pdf}
      \end{figure}
      }
    \end{column}
    \begin{column}{3cm}
      \begin{block}{}
        \only<1>
        {
					Départ de a, un point de l'$alpha$-shape.          
				}
        \only<2>
        {
          On tire les premiers convergents :\\
          \begin{itemize}
            \item $p_{-2} = (1, 0)$
            \item $p_{-1} = (0, 1)$
            \item $p_{0} = (1, 0)$
            \item $p_{1} = (5, 1)$
          \end{itemize}            
        }

        \only<3>
        {
          On étudie le rayon du cercle circonscrit au triangle $T(a, p_{1}, p_{1}-p_{0})$.\\
					\alert{Deux cas se présentent.}
        }
      \end{block}     
    \end{column}
  \end{columns}
\end{frame}

%-----------------------------------------------------------------

\begin{frame}
  \frametitle{Cas 1 - $R_{\alpha} > R_T$}
\begin{columns}[t]
   \begin{column}{7cm}
      \only<1>
      {
        \begin{figure}[H]
          \centering
          \includegraphics[trim = 3cm 0cm 10cm 6cm, clip, width=7cm]{fig/as-algo/as-algo-10.pdf}
      \end{figure}
      }
      \only<2,3>
      {
        \begin{figure}[H]
          \centering
          \includegraphics[trim = 4cm 0.5cm 12cm 7.5cm, clip, width=7cm]{fig/as-algo/as-algo-10.pdf}
       \end{figure}
      }
      \only<3>
      {
          \begin{block}{$b \notin$ l'$\alpha$-shape.}
          	a et $p_1$ appartiennent au bord d'un disque de rayon -1/$\alpha$ dont le complémentaire inclue b.\\
          \alert{On calcul les convergents suivants...}
        \end{block}
      }   
      \only<4>
      {
        \begin{figure}[H]
          \centering
          \includegraphics[trim = 3cm 0cm 10cm 6cm, clip, width=7cm]{fig/as-algo/as-algo-11.pdf}
      \end{figure}
      }
      \only<5,6>
      {
        \begin{figure}[H]
          \centering
          \includegraphics[trim = 3cm 0cm 10cm 6cm, clip, width=7cm]{fig/as-algo/as-algo-11a.pdf}
      \end{figure}
      }
   \end{column}
    \begin{column}{3cm}
      \begin{block}{}
        \only<1>
        {
          \alert{CAS 1}\\
          \alert{$R_{\alpha} > R_T$}
          
        }
        \only<2,3>
        {
          Soit $b = p_{1} - p_{0}$.
          \begin{itemize}
            \item $b \in \mathcal{D} \left( R_{\alpha} \right)^C$
          \end{itemize}
        }        
        \only<4>
        {
					$p_{0} + q_2 * p_1$ n'intersecte pas le disque.\\
				  $p_{1}$ est donc un sommet de l'$\alpha$-shape.			

        }
        \only<5,6>
        {
          On reprend l'algorithme.
        }
      \end{block}
      \only<6>
      {
        \begin{block}{Remarque}
          \alert{Si $\alpha$ proche de 0.}\\
          \alert{$\Rightarrow R_{\alpha} >> R_T$.}\\
          \alert{$\Rightarrow$ l'enveloppe convexe.}\\

        \end{block}
      }     
    \end{column}
  \end{columns}
\end{frame}

%-----------------------------------------------------------------

\begin{frame}
  \frametitle{Cas 2 - $R_{\alpha} < R_T$}
 \begin{columns}[t]
   \begin{column}{7cm}
      \only<1>
      {
        \begin{figure}[h!]
          \centering
          \includegraphics[trim = 3cm 0cm 10cm 6cm, clip, width=7cm]{fig/as-algo/as-algo-21.pdf}
      \end{figure}
      }
      \only<2,3>
      {
        \begin{figure}[h!]
          \centering
          \includegraphics[trim = 4cm 0.5cm 12cm 7.5cm, clip, width=7cm]{fig/as-algo/as-algo-21.pdf}
      \end{figure}
      }
      \only<3>
      {
      	\begin{block}{$b \in$ l'$\alpha$-shape.}
          On ne peut rejoindre a et $p_1$ par le bord d'un disque de rayon 1/$\alpha$ sans avant passer par au moins un autre sommets.
          \end{block}
      }         
      \only<4>
      {
        \begin{figure}[h!]
          \centering
          \includegraphics[width=7cm]{fig/as-algo/dicho.pdf}
      \end{figure}
      }
      
      \only<5>
      {
        \begin{figure}[h!]
          \centering
          \includegraphics[trim = 3cm 0cm 10cm 6cm, clip, width=7cm]{fig/as-algo/as-algo-22.pdf}
      \end{figure}
      }
      \only<6>
      {
        \begin{figure}[h!]
          \centering
          \includegraphics[trim = 3cm 0cm 10cm 6cm, clip, width=7cm]{fig/as-algo/as-algo-23.pdf}
      \end{figure}
      }
      \only<7>
      {
        \begin{figure}[h!]
          \centering
          \includegraphics[trim = 3cm 0cm 10cm 6cm, clip, width=7cm]{fig/as-algo/as-algo-24.pdf}
      \end{figure}
      }
      \only<8>
      {
        \begin{figure}[h!]
          \centering
          \includegraphics[trim = 3cm 0cm 10cm 6cm, clip, width=7cm]{fig/as-algo/as-algo-25.pdf}
      \end{figure}
      }
      \only<9>
      {
        \begin{figure}[h!]
          \centering
          \includegraphics[trim = 3cm 0cm 10cm 6cm, clip, width=7cm]{fig/as-algo/as-algo-26.pdf}
      \end{figure}
      }
    \end{column}
    \begin{column}{3cm}
      \begin{block}{}
        \only<1>
        {
          \alert{CAS 2}\\
          \alert{$R_{\alpha} < R_T$}

        }
        \only<2,3>
        {
          Soit $b = p_{1} - p_{0}$.
          \begin{itemize}
            \item $b \notin \mathcal{D} \left( R_{\alpha} \right)^C$
          \end{itemize}

        }
        \only<4>
        {
          Les rayons sont ordonnés et strictement croissant.\\
          \alert{Recherche dichotomique} du premier sommet dans l'$\alpha$-shape.\\
        }        
        \only<5>
        {
          On étudie le rayon du triangle du milieu :
          \begin{itemize}
            \item $R_{\alpha} < R_T$
          \end{itemize}
          On va chercher en aval.
        }
        \only<6>
        {
          \begin{itemize}
            \item $R_{\alpha} > R_T$
          \end{itemize}
          On va chercher en amont.
        }
        \only<7, 8>
        {
          Jusqu'à arrêter la procédure pour ajouter le premier point dans l'enveloppe...\\          
        }
        \only<8>
        {
          ainsi que tous les suivants jusqu'à $p_1$.     
        }
        \only<9>
        {
          $p_1$ qui devient le nouveau point de départ de l'algorithme.
          
        }  
      \end{block}
     
    \end{column}
  \end{columns}

\end{frame}

%-----------------------------------------------------------------

\begin{frame}
  \frametitle{Résultats}
	\begin{columns}[t]
 		\begin{column}{4cm}
  		\begin{block}{Sommets/$r^{2/3}$}
				\begin{tiny}
					\begin{tabular}{|l|c|c|}
						\hline
						Rayons & $\alpha$ & AS\\
						\hline
            32 & 1.024 & 17.7511\\
            64 & 4.096 & 17.0544\\
            128 & 16.384 & 18.5901\\
            256 & 65.536 & 19.2063\\
            512 & 262.144 & 20.3281\\
            1024 & 1048.58 & 21.0801\\
            2048 & 4194.3 & 21.9657\\
            4096 & 16777.2 & 22.192\\
            8192 & 67108.9 & 22.7541\\
            16384 & 268435 & 23.0402\\
            32768 & 1.07374e+06 & 23.2758\\
            65536 & 4.29497e+06 & 23.6162\\
            131072 & 1.71799e+07 & 23.9101\\
            262144 & 6.87195e+07 & 24.1555\\
            524288 & 2.74878e+08 & 24.4173\\
            1048576 & 1.09951e+09 & 24.5918\\
            2097152 & 4.39805e+09 & 24.7584\\
            4194304 & 1.75922e+10 & 24.9412\\
            8388608 & 7.03687e+10 & 25.0711\\
            16777216 & 2.81475e+11 & 25.2065\\
            33554432 & 1.1259e+12 & 25.309 \\
					  \hline
					\end{tabular}  
				\end{tiny}
			\end{block} 
		\end{column}
		\begin{column}{6.5cm}
				\begin{figure}[h!]
					\centering
				  \includegraphics[width=\linewidth]{fig/res/as.png}
			 	\end{figure}    
        \begin{block}{}
            FIT :  $f1(x) = 25.23*x^{2/3}$\\
            FIT :  $f10(x) = 22.2147*x^{0.674212}$ 
          \end{block} 
		\end{column}
	\end{columns} 
\end{frame}

%-----------------------------------------------------------------
\subsection{$\alpha \geq 0$}
%-----------------------------------------------------------------

\begin{frame}
  \frametitle{$\alpha$-hull avec $\alpha \geq 0$}
  \begin{block}{Définition}
		L'intersection de tous les disques fermés de rayon 1/$\alpha$ qui contiennent tous les points à l’intérieur du disque.
	\end{block} 
 
	\begin{block}{Principe de l'algorithme.}
		\begin{itemize}
			\item Récupérer tous les sommets de l'enveloppe convexe.
			\item Éliminez tous ceux qui sont en trop.
		\end{itemize}	
	\end{block} 


\end{frame}

%-----------------------------------------------------------------

\begin{frame}
  \frametitle{Départ}
\begin{columns}[t]
    \begin{column}{7cm}
      \only<1>
      {
        \begin{figure}[h!]
          \centering
          \includegraphics[trim = 0cm 0cm 0cm 3.5cm, clip, width=7cm]{fig/pas/pas-2.pdf}
        \end{figure}
      }
      \only<2>
      {
        \begin{figure}[h!]
          \centering
          \includegraphics[trim = 0cm 0cm 0cm 3.5cm, clip, width=7cm]{fig/pas/pas-3a.pdf}
        \end{figure}
      }      
    \end{column}
    \begin{column}{3cm}
      \begin{block}{}
        \only<1>
        {
			 		\begin{itemize}
						\item On part de l'enveloppe convexe.
						\item On ne conserve que les sommets.
					\end{itemize}
				}
        \only<2>
        {
          On prend trois sommets succéssifs :
			 		\begin{itemize}
						\item a, b et c.
					\end{itemize}
          On étudie le rayon du cercle circonscrit au triangle T(a, b, c).\\
					\alert{Deux cas se présentent.}
        }
      \end{block}  
    \end{column}
  \end{columns}
\end{frame}

%-----------------------------------------------------------------

\begin{frame}
  \frametitle{Cas 1 - $R_{\alpha} > R_T$}
\begin{columns}[t]
    \begin{column}{7cm}
     \only<1>
      {
        \begin{figure}[h!]
          \centering
          \includegraphics[trim = 0cm 0cm 0cm 3.5cm, clip, width=7cm]{fig/pas/pas-4.pdf}
        \end{figure}
      }
      \only<2>
      {
        \begin{figure}[h!]
          \centering
          \includegraphics[trim = 0cm 0.5cm 5cm 8.5cm, clip, width=7cm]{fig/pas/pas-4.pdf}
        \end{figure}
      
        \begin{block}{$b \notin$ l'$\alpha$-shape.}
          On ne peut pas rejoindre a et c par le bord d'un disque de rayon 1/$\alpha$ sans ommettre b.
        \end{block}
      }
      \only<3>
      {
        \begin{figure}[h!]
          \centering
          \includegraphics[trim = 0cm 0cm 0cm 3.5cm, clip, width=7cm]{fig/pas/pas-5.pdf}
        \end{figure}
      }
      \only<4,5>
      {
        \begin{figure}[h!]
          \centering
          \includegraphics[trim = 0cm 0cm 0cm 3.5cm, clip, width=7cm]{fig/pas/pas-6.pdf}
        \end{figure}
      }
    \end{column}
    \begin{column}{3cm}
      \begin{block}{}
        \only<1,2>
        {
          \alert{CAS 1}
          \alert{$R_{\alpha} >> R_T$}\\

        }
        \only<2>
        {
          \begin{itemize}
            \item $b \notin \mathcal{D} \left( R_{\alpha} \right)$
          \end{itemize}
        }
        \only<3>
        {
          b est un sommet de l'$\alpha-shape$.\\
          $[a, b]$ est une arête.
        }
        \only<4>
        {
          On poursuit l'algorithme :\\
          \begin{itemize}
            \item a devient b.
            \item b devient c.
            \item c est le sommet suivant.
          \end{itemize}
        }
				\only<5>
        { Remarques :\\
          \alert{Si $\alpha$ proche de 0.}\\
          \alert{$\Rightarrow R_{\alpha} >> R_T$.}\\
          \alert{$\Rightarrow$ l'enveloppe convexe.}\\
        }
			\end{block}

    \end{column}
  \end{columns}

\end{frame}

%-----------------------------------------------------------------

\begin{frame}
  \frametitle{Cas 2 - $R_{\alpha} < R_T$}

\begin{columns}[t]
    \begin{column}{7cm}
      \only<1>
      {
        \begin{figure}[h!]
          \centering
          \includegraphics[trim = 0cm 0cm 0cm 3.5cm, clip, width=7cm]{fig/pas/pas-10.pdf}
        \end{figure}
      }
      \only<2>
      {
        \begin{figure}[h!]
          \centering
          \includegraphics[trim = 0cm 0.5cm 5cm 8.5cm, clip, width=7cm]{fig/pas/pas-10.pdf}
        \end{figure}

				\begin{block}{$b \notin$ l'$\alpha$-shape.}
          On peut rejoindre a et c par le bord d'un disque discret de rayon 1/$\alpha$ qui inclue b.
        \end{block}

      }
      \only<3>
      {
        \begin{figure}[h!]
          \centering
          \includegraphics[trim = 0cm 0cm 0cm 3.5cm, clip, width=7cm]{fig/pas/pas-11.pdf}
        \end{figure}
      }
      \only<4>
      {
        \begin{figure}[h!]
          \centering
          \includegraphics[trim = 0cm 0cm 0cm 3.5cm, clip, width=7cm]{fig/pas/pas-12.pdf}
        \end{figure}
      }
      \only<5>
      {
        \begin{figure}[h!]
          \centering
          \includegraphics[trim = 0cm 0cm 0cm 3.5cm, clip, width=7cm]{fig/pas/pas-13.pdf}
        \end{figure}
			}
    \end{column}
    \begin{column}{3cm}
      \begin{block}{}
      \only<1,2>
        {
          \alert{CAS 2}
          \alert{$R_D < R_{\alpha} < R_T$}\\
        }
        \only<2>
        {
          \begin{itemize}
            \item $b \in \mathcal{D} \left( R_{\alpha} \right)$
          \end{itemize}
          
        }
        \only<3,4>
        {
          \begin{itemize}
            \item $b \in$ l'$\alpha$-hull.
            \item b n'est pas un $\alpha$-extrême.
            \item $b \notin$ l'$\alpha$-shape.
          \end{itemize}
        }
        \only<4,5>
        {
          On supprime b de la liste des sommets.\\
          On poursuit la procédure.\\
        }
        \only<5>
        {
          \begin{itemize}
            \item a reste a.
            \item b devient c.
            \item c est le sommet suivant.
          \end{itemize}
        }
      \end{block}
    \end{column}
  \end{columns}
\end{frame}

