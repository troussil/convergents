%----------------------------------------------------------------------------------------
%    PACKAGES AND THEMES
%----------------------------------------------------------------------------------------

\documentclass{beamer}

\usepackage{times}
\usepackage[T1]{fontenc}
\usepackage[english,francais]{babel}
\usefonttheme{professionalfonts}

\usepackage[]{beamerthemeliris} % paquet theme, options: nogradient,nobackground


\usepackage{graphicx} %paquet graphiques
\usepackage{epsfig}

\usepackage{booktabs} % Allows the use of \toprule, \midrule and \bottomrule in tables
\usepackage[utf8]{inputenc} %codage
\usepackage{lmodern}
%\usepackage[french]{babel}

%----------------------------------------------------------------------------------------
%    TITLE PAGE
%----------------------------------------------------------------------------------------

\title[Présentation M2Disco]{Structure d'un cercle discret en dimension deux} % The short title appears at the bottom of every slide, the full title is only on the title page

\author{\bsc{Lafond} Thomas} % Your name
\date{10 Juin 2013}
\institute[LIRIS] % Your institution as it will appear on the bottom of every slide, may be shorthand to save space
{
  Laboratoire d'InfoRmatique en Image et Syst\`{e}mes d'information \\ % Your institution for the title page
  \medskip
  \textit{-- Présentation d'équipe : M2Disco --}\\
  \medskip
  Encadrant : \bsc{Roussillon} Tristan
}
\date{10 Juin 2013} % Date, can be changed to a custom date

\begin{document}

% titre
\begin{frame}
  \titlepage % Print the title page as the first slide
\end{frame}

% table des matières
\begin{frame}
  \frametitle{Tables des matières} % Table of contents slide, comment this block out to remove it
  \setcounter{tocdepth}{1}
  \tableofcontents %
\end{frame}

%----------------------------------------------------------------------------------------
%    PRESENTATION SLIDES
%----------------------------------------------------------------------------------------

%------------------------------------------------
\section{Parcours}
%------------------------------------------------

\subsection{Parcours}
 
\begin{frame}
%\frametitle{Parcours}
  \begin{block}{Licence de Mathématiques et d'Informatique}
    \begin{itemize}
      \item Mathématiques : L1, L2, L3.
      \item Informatique : L1, L2.
      \item Domaine Scientifique : Science Physique, Chimie et Biologie.
    \end{itemize}
  \end{block}

  \begin{block}{Master Statistiques, Informatique et Techniques Numériques}
    \begin{columns}[t]

      \begin{column}{5.5cm}
        Trois composantes principales :
        \begin{itemize}
          \item Probabilités et Statistiques
          \item EDP et Calcul scientifique
          \item Développement Informatique
        \end{itemize}
      \end{column}

      \begin{column}{4.5cm}
        Une certaine ouverture  
        \begin{itemize}
          \item \alert{Stages}
          \item Calcul parallèle
          \item Mécanique des fluides
        \end{itemize}
      \end{column}
    \end{columns}  
  \end{block}

    \begin{block}{Stages précédents}
      \begin{itemize}
        \item Modélisation Hydrologique - Irstea - 5 Mois
        \item Décomposition de domaine - MmodD / Lyon 1 - 2 Mois
      \end{itemize}
  \end{block}

\end{frame}

%------------------------------------------------
\section{Contexte et Motivations}
%------------------------------------------------

\subsection{Géométrie Discrète}
\begin{frame}
  \frametitle{Géométrie Discrète}

  \begin{columns}[t]
    \begin{column}{6cm}
      \begin{block}{Géométrie Discrète}
        Une discipline à l'intersection de nombreux domaines scientifiques compris dans les Mathématiques et l'Informatique.
      \end{block}
      
      \begin{block}{Définition \textit{simple}}
       Étude d'objets disposés sur des structures régulières.
      \end{block}
      
      \begin{block}{Objet d'étude}
        Les \alert{cerles discrèts} sur la grille $Z^2$.
      \end{block}
    \end{column}

    \begin{column}{4cm}
      \vspace{-0.7cm}
      \begin{figure}[h!]
      \centering
         \includegraphics[width= 4cm]{fig/mot-geodiscete.png}
      \caption{Digital Geometry --D. CoeurJolly / Lattice}
    \end{figure}
    
    \vspace{-1cm}
    \begin{figure}[h!]
      \centering
        \includegraphics[width=4cm]{fig/mot-geo.pdf}
      \end{figure}
    \end{column}
  \end{columns}  
\end{frame}

\subsection{Disque discret}
\begin{frame}
  \frametitle{Disque fermé Discret}
  \begin{block}{}
    $$\mathcal{D} =  \left\{ pt : (x,y) \in \mathcal{Z}^{2} |  ax + by + c(x^2 + y^2) + d \leq 0. \right\}$$
  \end{block}

  \begin{block}{Représentation cartésienne}
    \begin{itemize}
      \item Modèle géométrique uniquement avec des entiers.
      \item Disque de centre et rayon rationnels
    \end{itemize}
  \end{block}
  \begin{exampleblock}{}
    \begin{itemize}
      \item Calcul exact permettant d'éviter tous les problèmes d'approximations et d'arrondis.
      \item Adapté à l'analyse d'images.
      \item Optimisation possible.
    \end{itemize}
  \end{exampleblock}

  \begin{exampleblock}{}
    \begin{itemize}
      \item Plusieurs représentations possibles.
      \item Peu de différences graphiques...
      \item Mais des calculs et algorithmes différents.
    \end{itemize}
  \end{exampleblock}

\end{frame}

\subsection{Cercle discret}
\begin{frame}
\frametitle{Le passage du disque discret au cercle discret}

  \begin{block}{Comment représenter un cercle ?}
    Pas d'équivalent Euclidien : 
    \vspace{-0.4cm}
    $$\mathcal{D} =  \left\{ pt : (x,y) \in \mathcal{Z}^{2} |  ax + by + c(x^2 + y^2) + d = 0 \right\} \alert{?} $$ 
  \end{block}
  
  \begin{columns}[t]
    \begin{column}{6cm}
      \only<1,4>
      {
        \begin{figure}[h!]
          \centering
          \includegraphics[width=4cm]{fig/mot-cercle.pdf}
        \end{figure}
      }
      \only<2>
      {
        \begin{figure}[h!]
          \centering
          \includegraphics[width=4cm]{fig/mot-cercle-int.pdf}
        \end{figure}
      }
      \only<3>
      {
        \begin{figure}[h!]
          \centering
          \includegraphics[width=4cm]{fig/mot-cercle-ext.pdf}
        \end{figure}
      }
    \end{column}

    \begin{column}{5cm}
      \begin{exampleblock}{Notions éxistantes}
       
        \begin{itemize}
          \item Intérieur / Extérieur.
          \item Bordure.
          \item  $\Rightarrow$ Enveloppes.
        \end{itemize}
      \end{exampleblock} 
      \only<4>
      {
        \begin{center}  
          \alert{Etude des $\alpha$-shape de cercles discrets sur $\mathcal{Z}^{2}$.}
        \end{center}
      }
    \end{column}
  \end{columns} 
  

\end{frame}

%------------------------------------------------
\section{Existant}
%------------------------------------------------

\subsection{Enveloppe convexe - Har-Peled}
\begin{frame}
\frametitle{Enveloppe convexe - Har-Peled}

  \only<1>
  {
    \begin{figure}[h!]
      \centering
      \includegraphics[width=4cm]{fig/har-1-0.pdf}
    \end{figure}
  }
  \only<2>
  {
    \begin{figure}[h!]
      \centering
      \includegraphics[width=4cm]{fig/har-1-1.pdf}
    \end{figure}
  }
  \only<3>
  {
    \begin{figure}[h!]
      \centering
      \includegraphics[width=4cm]{fig/har-1-2.pdf}
    \end{figure}
  }
  \only<4>
  {
    \begin{figure}[h!]
      \centering
      \includegraphics[width=4cm]{fig/har-1-3.pdf}
    \end{figure}
  }
  \only<5>
  {
    \begin{figure}[h!]
      \centering
      \includegraphics[width=4cm]{fig/har-1-4.pdf}
    \end{figure}
  }
  \only<6>
  {
    \begin{figure}[h!]
      \centering
      \includegraphics[width=4cm]{fig/har-1-5.pdf}
    \end{figure}
  }
  \only<7>
  {
    \begin{figure}[h!]
      \centering
      \includegraphics[width=4cm]{fig/har-1-6.pdf}
    \end{figure}
  }
  \only<8>
  {
    \begin{figure}[h!]
      \centering
      \includegraphics[width=4cm]{fig/har-1-7.pdf}
    \end{figure}
  }
  \only<9>
  {
    \begin{figure}[h!]
      \centering
      \includegraphics[width=4cm]{fig/har-1-8.pdf}
    \end{figure}
  }   

\end{frame}

\subsection{Triangulation de Delaunay}  % ?? là / pas là ??
\begin{frame}
\frametitle{Triangulation de Delaunay}
On parle un peu de triangulations de Delaunay, de leur lien au alpha-hull, au alpha shape.
\end{frame}

%------------------------------------------------
\section{Contributions}
%------------------------------------------------

\subsection{Généralisation de Har-Peled - $\alpha \leq 0$}
\begin{frame}
\frametitle{Généralisation de Har-Peled - $\alpha \leq 0$}

\end{frame}

\subsection{$\alpha \geq 0$}
\begin{frame}
\frametitle{$\alpha \geq 0$}
\end{frame}

\subsection{Protocole}
\begin{frame}
\frametitle{Protocole}
\end{frame}

%------------------------------------------------
\section{Résultats}
%------------------------------------------------

\subsection{Enveloppe convexe}
\begin{frame}
\frametitle{Enveloppe convexe}
\end{frame}

\subsection{$\alpha \leq 0$}
\begin{frame}
\frametitle{$\alpha \leq 0$}
\end{frame}

\subsection{$\alpha \geq 0$}
\begin{frame}
\frametitle{$\alpha \geq 0$}
\end{frame}

%------------------------------------------------
\section{Suite}
%------------------------------------------------

\subsection{Poursuite du projet}
\begin{frame}
\frametitle{Poursuite du projet}
\end{frame}

\subsection{Poursuite personnelle}
\begin{frame}
\frametitle{Poursuite personnelle}
\end{frame}

%----------------------------------------------------------------------------------------
\end{document} 
