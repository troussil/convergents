%----------------------------------------------------------------------------------------
%    PACKAGES AND THEMES
%----------------------------------------------------------------------------------------

\documentclass{beamer}

\usepackage{times}
\usepackage[T1]{fontenc}
\usepackage[english,francais]{babel}
\usefonttheme{professionalfonts}

\usepackage[]{beamerthemeliris} % paquet theme, options: nogradient,nobackground


\usepackage{graphicx} %paquet graphiques
\usepackage{epsfig}

\usepackage{booktabs} % Allows the use of \toprule, \midrule and \bottomrule in tables
\usepackage[utf8]{inputenc} %codage
\usepackage{lmodern}
\usepackage{amssymb}
\usepackage{amsmath}

\usepackage{bibentry}
\nobibliography*

 %----------------------------------------------------------------------------------------
%    TITLE PAGE
%----------------------------------------------------------------------------------------

\title[Présentation M2Disco]{Structure du bord d'un disque discret} % The short title appears at the bottom of every slide, the full title is only on the title page

\author{Thomas \bsc{Lafond}} % Your name
\date{10 Juin 2013}
\institute[LIRIS] % Your institution as it will appear on the bottom of every slide, may be shorthand to save space
{
  Laboratoire d'InfoRmatique en Image et Syst\`{e}mes d'information \\ % Your institution for the title page
  \medskip
  \textit{-- Présentation d'équipe : M2Disco --}\\
  \medskip
  Encadrant : Tristan \bsc{Roussillon}
}
\date{10 Juin 2013} % Date, can be changed to a custom date

\begin{document}

% titre
\begin{frame}
  \titlepage % Print the title page as the first slide
\end{frame}

%----------------------------------------------------------------------------------------
%    PRESENTATION SLIDES
%----------------------------------------------------------------------------------------

%------------------------------------------------
\section{Parcours}
%------------------------------------------------

\begin{frame}
%\frametitle{Parcours}
   \begin{block}{Licence de Mathématiques et d'Informatique}
    \begin{itemize}
      \item Mathématiques : L1, L2, L3.
      \item Informatique : L1, L2.
      \item Domaine Scientifique : Science Physique, Chimie et Biologie.
    \end{itemize}
  \end{block}

  \begin{block}{Master Statistiques, Informatique et Techniques Numériques}
    \begin{columns}[t]
      \begin{column}{5.5cm}
        Trois composantes principales :
        \begin{itemize}
          \item Probabilités et Statistiques
          \item EDP et Calcul scientifique
          \item Développement Informatique
        \end{itemize}
      \end{column}
      \begin{column}{4.5cm}
        Une certaine ouverture  
        \begin{itemize}
          \item \alert{Stages}
          \item Calcul parallèle
          \item Mécanique des fluides
        \end{itemize}
      \end{column}
    \end{columns}  
  \end{block}

  \begin{block}{Stages précédents}
    \begin{itemize}
      \item Modélisation Hydrologique - Irstea - 5 Mois
      \item Décomposition de domaine - MmodD / Lyon 1 - 2 Mois
    \end{itemize}
  \end{block}
\end{frame}


% table des matières
\AtBeginSection[]{
  \begin{frame}{Sommaire}
  \small \tableofcontents[currentsection, hideallsubsections]
  \end{frame} 
}


%------------------------------------------------
\section{Motivations}
%------------------------------------------------

%-----------------------------------------------------------------
\subsection{Du disque Euclidien au disque discret}
%-----------------------------------------------------------------


\begin{frame}
\frametitle{Du disque Euclident au disque discret}
\only<1>
{
  \begin{columns}[t]
    \begin{column}{0.5\linewidth}
      \begin{block}{Disque Euclidien : $\mathcal{D}_e$}
        $$\left\{ (x,y) \in \mathbb{R}^{2} | (x - u)^2 + (y - v)^2 \leq R^2 \right\}$$
        \begin{figure}[h!]
          \centering
          \includegraphics[width=0.6\linewidth]{fig/2-mot/circle/circle-euc-0.pdf}
        \end{figure}
      \end{block}
    
    \end{column}
    \begin{column}{0.5\linewidth}
      \begin{block}{Disque Discret : $\mathcal{D}$}
        $$\left\{ (x,y) \in \alert{\mathbb{Z}^{2}} | (x - u)^2 + (y - v)^2 \leq R^2 \right\}$$

        \begin{figure}[h!]
          \centering
          \includegraphics[width=0.6\linewidth]{fig/2-mot/circle/circle-dis-0.pdf}
        \end{figure}
      \end{block}  
    \end{column}
  \end{columns} 
}
        
\only<2>
{
  \begin{columns}[t]
    \begin{column}{0.5\linewidth}
      \begin{block}{Cercle Euclidien : $\mathcal{C}_e$}
        $$\left\{ (x,y) \in \mathbb{R}^{2} | (x - u)^2 + (y - v)^2 \alert{=} R^2 \right\}$$
        \begin{figure}[h!]
          \centering
          \includegraphics[width=0.6\linewidth]{fig/2-mot/circle/circle-euc-0.pdf}
        \end{figure}
      \end{block}
    
    \end{column}
    \begin{column}{0.5\linewidth}
      \begin{block}{Cercle Discret : $\mathcal{C}$}
        $$\left\{ (x,y) \in \alert{\mathbb{Z}^{2}} | (x - u)^2 + (y - v)^2 \alert{\stackrel{?}{=}} R^2 \right\}$$
        \vspace{-0.24cm}
        \begin{figure}[h!]
          \centering
          \includegraphics[width=0.6\linewidth]{fig/2-mot/circle/circle-dis-0.pdf}
        \end{figure}
      \end{block}  
    \end{column}
  \end{columns} 
}

\begin{exampleblock}{}
  avec $O(u,v) \in \mathbb{Q}^{2}$ les coordonnées du centre et $R^2 \in \mathbb{Q}$ le rayon.\\
\end{exampleblock}
\end{frame}

%-----------------------------------------------------------------

\begin{frame}
\frametitle{Le voisinage et la connexité}

\begin{block}{}
  \begin{columns}[t]
    \begin{column}{0.5\linewidth}
      4-voisinage d'un point (u,v) : $\mathcal{V}_4(u,v)$
      $$ \left\{ (x,y) \in \mathbb{Z}^{2} |  |x-u|+|y-u| = 1 \right\}$$
    \end{column}
    \hspace{-1cm}
    \begin{column}{0.5\linewidth}
      8-voisinage d'un point (u,v) : $\mathcal{V}_8(u,v)$
      $$ \left\{ (x,y) \in \mathbb{Z}^{2} |  max(|x-u|,|y-u|) = 1 \right\}$$  
    \end{column}
  \end{columns} 

  \begin{figure}[H]
    \centering
    \includegraphics[width=.3\linewidth]{fig/2-mot/connexe/connexite.pdf}
  \end{figure}
\end{block}

\begin{block}{Cercle Discret 8-connexes et 4-connexes}
  \begin{columns}[t]
    \begin{column}{0.5\linewidth}
      \begin{figure}[H]
        \centering
        \includegraphics[width=.5\linewidth]{fig/2-mot/circle/circle-dis-1a.pdf}

      \end{figure}
      
    \end{column}
    \begin{column}{0.5\linewidth}
      \begin{figure}[H]
        \centering
        \includegraphics[width=.5\linewidth]{fig/2-mot/circle/circle-dis-1b.pdf}
      \end{figure}
    \end{column}
  \end{columns} 

  $$ \mathcal{C}_{*} =  \left\{ (x,y) \in \mathcal{D} | \left( \mathcal{V}_{*}(x,y) \cap \mathcal{D} \right) \neq \mathcal{V}_{*}(x,y) \right\}$$
\end{block}
\end{frame}

%-----------------------------------------------------------------

\begin{frame}
\frametitle{Énoncé de la problématique}
\begin{block}{Le disque discret est l’union de deux ensembles disjoints}
  \begin{columns}[t]
    \begin{column}{0.65\linewidth}
      \begin{itemize}      
        \item Son intérieur strict : (bleu clair)
        $\stackrel{\ \circ}{\mathcal{D}}_{*} =  \left\{ (x,y) \in \mathcal{D} | \mathcal{V}_{*}(x,y) \cap \mathcal{D} = \mathcal{V}_{*}(x,y) \right\}$\\
        Les points possèdent tous leurs quatre plus proches voisins à l’intérieur du disque.
        \item Son bord : (bleu foncé)
        $\mathcal{C}_{*} =  \left\{ (x,y) \in \mathcal{D} | \left( \mathcal{V}_{*}(x,y) \cap \mathcal{D} \right) \neq \mathcal{V}_{*}(x,y) \right\}$\\
        Organisation spatiale des points moins triviale.
      \end{itemize}
    \end{column}
 
    \begin{column}{0.35\linewidth}
      \begin{figure}[H]
        \centering
        \includegraphics[width=.7\linewidth]{fig/2-mot/circle/circle-dis-2.pdf}
       \end{figure}
    \end{column}
  \end{columns}
\end{block}


\only<1>
{
  \begin{exampleblock}{Remarques}
    \begin{itemize}
      \item La structure de l’intérieur d'un disque discret est évidente.
      \item Seule l’étude du bord, nous intéresse pour comprendre l’organisation des points des disques discrets.
    \end{itemize}
  \end{exampleblock} 
}
\only<2>
{
  \begin{alertblock}{Problématique}
    Comprendre comment sont organisés spatialement les points du bord d’un disque discret et comment cette structure est déterminée par les paramètres du disque (position et taille) par rapport à la grille sous-jacente.
  \end{alertblock}

}
\end{frame}

%-----------------------------------------------------------------
\subsection{$\alpha$-hulls et $\alpha$-shapes}
%-----------------------------------------------------------------

\begin{frame}
\frametitle{Un outil particulièrement opportun}

\begin{block}{$\alpha$-hulls et $\alpha$-shapes}
  Les $\alpha$-hulls et les $\alpha$-shapes ont été définies pour la première fois par Edelsbrunner \emph{et. al.} [EKS83] et font appel aux disques généralisés.\\
\end{block}

\begin{block}{Disques généralisés }
  Ils nous permettent de définir des disques avec des rayons positifs et négatifs en faisant appel au complémentaire :

  \begin{itemize}
    \item $\mathcal{D}_{\alpha}$ est le disque fermé de rayon $1/\alpha$ pour $\alpha > 0$.
    \item $\mathcal{D}_{\alpha}$ est le complémentaire fermé du disque de rayon $- 1/\alpha$ pour $\alpha < 0$. 
  \end{itemize}
\end{block}

%----- Begin biblio -----
\scriptsize
\begin{thebibliography}{alpha}
  \bibitem{EKS83}
  [EKS83] Edelsbrunner, H., Kirkpatrick, D., Seidel, R.
  \newblock On the Shape of a Set of Points in the Plane
  \newblock {\em IEEE Transactions on Information Theory}, 29(4):551--559, 1983.
\end{thebibliography}
%----- End biblio   -----
\end{frame}

%-----------------------------------------------------------------

\begin{frame}
\frametitle{Définitions}
\only<1>
{ Soit $\mathcal{S}$ un ensemble fini de points.
  \begin{block}{$\alpha$-hull de $\mathcal{S}$}
    Intersection de tous les disques généralisés de rayon $1/\alpha$ contiennant tous les points de l'ensemble.
    $$ \alpha_h(\mathcal{S}) = \cap \left\{ \mathcal{D}_{\alpha} | \mathcal{S} \subseteq \mathcal{D}_{\alpha} \right\}$$
    \begin{figure}[H]
      \centering
      \includegraphics[width=0.3\linewidth,page=1]{fig/2-mot/as/mot-alpha-shape.pdf}
      \includegraphics[width=0.3\linewidth,page=3]{fig/2-mot/as/mot-alpha-shape.pdf}
    \end{figure} 
  \end{block}

  \begin{block}{Sommets}
    \begin{itemize}
      \item Les sommets de $\alpha$-hull sont appelés points $\alpha$-extrêmes.
      \item S'ils sont reliés par un arc de cercle de rayon $\pm 1/ \alpha$ qui ne contient aucun autre point que ses extrémités et qui se trouve sur le bord d'un disque généralisé contenant l'ensemble des points, on dit qu'ils sont $\alpha$-adjacents.
    \end{itemize}
  \end{block}
}
\only<2>
{
  \begin{block}{}
    \vspace{-0.2cm}
    \begin{figure}[H]
      \centering
      \includegraphics[width=0.2\linewidth,page=1]{fig/2-mot/as/mot-alpha-shape.pdf}
      \includegraphics[width=0.2\linewidth,page=3]{fig/2-mot/as/mot-alpha-shape.pdf}
    \end{figure} 
    \vspace{-1cm}
    \begin{itemize}
      \item $\alpha$-hull :
      \item Les sommets de $\alpha$-hull sont appelés points $\alpha$-extrêmes.
      \item S'ils sont reliés par un arc de cercle de rayon $\pm 1/ \alpha$ qui ne contient aucun autre point que ses extrémités et qui se trouve sur le bord d'un disque généralisé contenant l'ensemble des points, on dit qu'ils sont $\alpha$-adjacents.
    \end{itemize}
  \end{block}


  \begin{block}{$\alpha$-shape de $\mathcal{S}$}
        Graphe plongé dans le plan reliant tous les points $\alpha$-extrêmes adjacents par des segments de droite.
    \begin{figure}[H]
      \centering
      \includegraphics[width=0.3\linewidth,page=2]{fig/2-mot/as/mot-alpha-shape.pdf}
      \includegraphics[width=0.3\linewidth,page=4]{fig/2-mot/as/mot-alpha-shape.pdf}
    \end{figure}
  \end{block}
}
\end{frame}

%-----------------------------------------------------------------

\begin{frame}
\frametitle{Propriétés}
\begin{columns}[t]
  \begin{column}{0.5\linewidth}
    \only<1>
    {
      \begin{figure}[H]
        \centering
        \includegraphics[width=\linewidth]{fig/2-mot/as/mot-as-3.pdf}
        \caption{Enveloppe convexe : $\alpha = 0$}
      \end{figure}
    }
    \only<2>
    {
      \begin{figure}[H]
        \centering
        \includegraphics[width=\linewidth]{fig/2-mot/as/mot-as-1.pdf}
        \caption{Bord 4-connexe : $\alpha = -2$}
      \end{figure}
    }
    \only<3>
    {
      \begin{figure}[H]
        \centering
        \includegraphics[width=\linewidth, page=8]{fig/2-mot/as/mot-as-5.pdf}
        \caption{Union d'$\alpha$-shape, $\alpha < 0$}
      \end{figure}
    }  
  
  \end{column}
  \begin{column}{0.5\linewidth}
    \begin{block}{}
      \begin{itemize}
        \item<1-> Cas $\alpha = 0$.\\
        Intersection de disques généralisés de rayon infini.\\
        Interpréter comme l'enveloppe convexe.
        \item<2-> Cas $\alpha = -2$ et $\alpha = -\sqrt{2}$.\\
        Bords définis au moyen du 8 et 4-voisinage.
        \item<3-> Union d'$\alpha$-shape [EKS83].\\
        Sous-ensembles des triangulations d’ordre 0 ($\alpha < 0$) et d’ordre n ($\alpha > 0$) de Delaunay.
      \end{itemize}
    \end{block} 
  \end{column}
\end{columns}


\end{frame}

%-----------------------------------------------------------------
\subsection{Triangulation de Delaunay}
%-----------------------------------------------------------------

\begin{frame}
\frametitle{Les triangulations de Delaunay}

Soit $\mathcal{S}$ un ensemble fini de points

\begin{block}{Triangulation de Delaunay d'ordre 0 de $\mathcal{S}$}
  Triangulation où chaque disque circonscrit au triangle ne contient aucun autre point que les sommets du triangle.
\end{block}
   
\begin{figure}[H]
  \centering
  \includegraphics[width=0.3\linewidth]{fig/2-mot/tri/mot-tri-a.pdf}
  \includegraphics[width=0.3\linewidth]{fig/2-mot/tri/mot-tri-b.pdf}
  \caption{Triangulations de Delaunay d'ordre 0 et n}
\end{figure}

\begin{block}{La triangulation de Delaunay d'ordre n de $\mathcal{S}$}
  Triangulation de l'enveloppe convexe de $\mathcal{S}$ où chaque disque circonscrit au triangle contient tous les points de l'ensemble.
\end{block}

\end{frame}


%-----------------------------------------------------------------
\section{Existant}
%-----------------------------------------------------------------

%-----------------------------------------------------------------
\subsection{Suivi de bord}
%-----------------------------------------------------------------
\begin{frame}
  \frametitle{Suivi de bord}
      \only<1>
    { 
      \begin{block}{Deux cas}
        \begin{itemize}
          \item 4-Connexes.
          \item 8-connexes.
        \end{itemize}
      \end{block} 
    } 
    \only<2,3>
    {
      \begin{block}{Suivi de bord d'un disque discret.}
        \begin{itemize}
          \item On étudie la position des voisins par rapport au disque.
          \item<3> On en déduit la direction à suivre en fonction du sens de rotation.
        \end{itemize}
      \end{block} 
    }
    
    %pics
    \only<1>
    {   
      \begin{figure}[h!]
        \centering
        \includegraphics[width=0.8\linewidth]{fig/3-existant/connexe/ex-connexe-0.pdf}
      \end{figure}    
    }
    \only<2>
    {     
      \begin{figure}[h!]
        \centering
        \includegraphics[width=0.8\linewidth]{fig/3-existant/connexe/ex-connexe-1.pdf}
      \end{figure}    
    }
    \only<3>
    {     
      \begin{figure}[h!]
        \centering
        \includegraphics[width=0.8\linewidth]{fig/3-existant/connexe/ex-connexe-2.pdf}
      \end{figure}    
    }
\end{frame}

%-----------------------------------------------------------------
\subsection{Enveloppe Convexe}
%-----------------------------------------------------------------

\begin{frame}
  \frametitle{Enveloppe convexe}

  \begin{block}{}
    On obtient une liste de sommets adjacents.\\
    On recherche le meilleurs candidat pour construire la prochaine arête.
  \end{block} 

  \begin{columns}[t]
 	 \begin{column}{3cm}
			\begin{figure}[h!]
	      \centering
	      \includegraphics[width=\linewidth]{fig/3-existant/convexe/ch-1.pdf}
     	\end{figure}    
    \end{column}

    \begin{column}{3cm}
      \begin{figure}[h!]
		      \centering
		      \includegraphics[width=\linewidth]{fig/3-existant/convexe/ch-2.pdf}
     	\end{figure}    
    \end{column}
			
		\begin{column}{3cm}
      \begin{figure}[h!]
		      \centering
		      \includegraphics[width=\linewidth]{fig/3-existant/convexe/ch-3.pdf}
     	\end{figure}    
    \end{column}
        
	\end{columns}  

\end{frame}

%-----------------------------------------------------------------

\begin{frame}
  \frametitle{Har-Peled - Convergents}

	\begin{block}{}
	\begin{tiny}
    [H98] Har-Peled, S.\\
    An output sensitive algorithm for discrete convex hulls\\
    {\em Computational Geometry}, 10(2):125--138, 1998.\\
    \end{tiny}
	\end{block} 


	\begin{columns}[t]
 		\begin{column}{5cm}
  		
			\begin{block}{}
	      Calcul des convergents :
	      \begin{itemize}
	        \item $a = (0,0), b = (y, x)$
	        \item $p_{-2} = (1,0), p_{-1} = (0,1)$
					\item<2-> \alert{$ p_{k} = p_{k-2} + q_{k}*p_{k-1}$}\\
		      avec le plus grand $q_{k} \in \mathbb{Z}$ tq $p_{k}$ et $p_{k-2}$ soient du même côté.
	      \end{itemize}
	    \end{block}
		\end{column}

		\begin{column}{5cm}
			\only<1>
			{
				\begin{figure}[h!]
					\centering
				  \includegraphics[width=0.6\linewidth]{fig/3-existant/har/har2-1.pdf}
			 	\end{figure}    
			}
			\only<2>
			{
				\begin{figure}[h!]
					\centering
				  \includegraphics[width=0.6\linewidth]{fig/3-existant/har/har2-2.pdf}
			 	\end{figure}    
			}
			\only<3>
			{
				\begin{figure}[h!]
					\centering
				  \includegraphics[width=0.6\linewidth]{fig/3-existant/har/har2-3.pdf}
			 	\end{figure}    
			}
	  \end{column}
	\end{columns}  

\end{frame}

%-----------------------------------------------------------------

\begin{frame}
  \frametitle{Har-Peled - Passage à l'enveloppe convexe du cercle}
  		
			\begin{block}{Calcul des convergents}
					On initialise avec les deux premiers convergents : $p_{-2}$ et $p_{1}$.
				\begin{itemize}
	        \item Tous les convergents impairs sont à l'intérieur du disque.
					\item Tous les convergents pairs sont à l'extérieur du disque.
	      \end{itemize}
		
	    \end{block}

				\begin{figure}[h!]
					\centering
				  \includegraphics[width=0.2\linewidth]{fig/3-existant/circle/cer-1.pdf}
				  \includegraphics[width=0.2\linewidth]{fig/3-existant/circle/cer-2.pdf}
					\includegraphics[width=0.2\linewidth]{fig/3-existant/circle/cer-3.pdf}
				  \includegraphics[width=0.2\linewidth]{fig/3-existant/circle/cer-4.pdf}
			 	\end{figure}    
      \only<2>
      {
			\begin{block}{}
					\alert{On repart du plus grand convergent impair.}
	    \end{block}
      }

\end{frame}

%-----------------------------------------------------------------

\begin{frame}
  \frametitle{Har-Peled - Résultat}
	\begin{columns}[t]
 		\begin{column}{3.8cm}
  		\begin{block}{Sommets/$r^{2/3}$}
				\begin{tiny}
					\begin{tabular}{|l|c|r|}
						\hline
						Rayons & CH-E0 & CH-E1 \\
						\hline
						32 & 3.50816 & 10.1246 \\
						64 & 3.48625 & 10.635 \\
						128 & 3.45612 & 11.1696 \\
						256 & 3.46525 & 11.5349 \\
						512 & 3.46984 & 11.8908 \\
						1024 & 3.46203 & 12.2068 \\
						2048 & 3.46115 & 12.4457 \\
						4096 & 3.45258 & 12.6752 \\
						8192 & 3.45479 & 12.9217 \\
						16384 & 3.45495 & 13.0297 \\
						32768 & 3.45479 & 13.1866 \\
						65536 & 3.45564 & 13.2766 \\
						131072 & 3.45433 & 13.4516 \\
						262144 & 3.45375 & 13.5341 \\
						524288 & 3.45423 & 13.6428 \\
						1048576 & 3.45374 & 13.7542 \\
						2097152 & 3.45316 & 13.8098 \\
						4194304 & 3.45429 & 13.8796 \\
						8388608 & 3.45336 & 13.9427 \\
						16777216 & 3.45343 & 14.0154 \\
						33554432 & 3.45357 & 14.0745 \\
						67108864 & 3.45092 & 14.0992 \\
						134217728 & 3.42459 & 14.2689 \\
						268435456 & 3.2448 & 15.9038 \\
					  \hline
					\end{tabular}  
				\end{tiny}
			\end{block} 
		\end{column}
		\begin{column}{7cm}
		    \only<1>
		    {
				  \begin{figure}[h!]
					  \centering
				    \includegraphics[width=\linewidth]{fig/res/ch-e1-100vertices-byradius.png}
			   	\end{figure}    
        }
        \only<2>
		    {
				  \begin{figure}[h!]
					  \centering
				    \includegraphics[width=\linewidth]{fig/res/time.png}
			   	\end{figure}    
        }
		\end{column}
	\end{columns} 

\end{frame}


%-----------------------------------------------------------------
\subsection{Triangulation de Delaunay}
%-----------------------------------------------------------------

\begin{frame}
  \frametitle{Triangulation de Delaunay}

	\begin{columns}[t]
 		\begin{column}{7cm}

			\begin{block}{Triangulation des arêtes}
\scriptsize
\begin{thebibliography}{alpha}

\bibitem{EdeKirSei83}
[EKS83] Edelsbrunner, H., Kirkpatrick, D., Seidel, R.
\newblock On the Shape of a Set of Points in the Plane
\newblock {\em IEEE Transactions on Information Theory}, 29(4):551--559, 1983.

\end{thebibliography}
				\begin{itemize}

					\item Étude des motifs des arêtes pour construire leurs triangulations de Delaunay.
					\item Les $\alpha$-shapes sont des sous-graphes de la triangulation. 
	      \end{itemize}	
			\end{block} 

			\only<2>
			{
				\begin{block}{}
					\alert{Construire les alpha-shapes directement en s'appuyant sur les convergents.}
				\end{block}
			}
 		\end{column}
    \begin{column}{3cm}
			\begin{figure}[h!]
				\centering
			  \includegraphics[width=\linewidth]{fig/3-existant/tri/delaunay.pdf}
			\end{figure}
	  \end{column}
	\end{columns}
\end{frame}


%-----------------------------------------------------------------
\section{Contributions}
%-----------------------------------------------------------------

%-----------------------------------------------------------------
\subsection{Généralisation de Har-Peled - $\alpha \leq 0$}
%-----------------------------------------------------------------

\begin{frame}
  \frametitle{$\alpha$-hull avec $\alpha \leq 0$}
  \begin{block}{Définition}
		L'intersection de tous les complémentaires fermés de disques de rayon -1/$\alpha$ qui contiennent tous les points à l’intérieur du disque.
	\end{block} 

  \begin{block}{Principe de l'algorithme.}
		\begin{itemize}
				\item Suivre l'algorithme de l'enveloppe convexe.
				\item Contrôler si les convergents sont des points $\alpha$-extremes.
		\end{itemize}	
	\end{block} 

\end{frame}

%-----------------------------------------------------------------

\begin{frame}
  \frametitle{Départ}
	\begin{columns}[t]
 		\begin{column}{7cm}
      \only<1>
      {
        \begin{figure}[h!]
          \centering
          \includegraphics[trim = 3cm 0cm 10cm 6cm, clip, width=7cm]{fig/4-contribution/negatif/as-algo-1.pdf}
      \end{figure}
      }
      \only<2>
      {
        \begin{figure}[h!]
          \centering
          \includegraphics[trim = 3cm 0cm 10cm 6cm, clip, width=7cm]{fig/4-contribution/negatif/as-algo-3.pdf}
      \end{figure}
      }
      \only<3>
      {
        \begin{figure}[h!]
          \centering
          \includegraphics[trim = 3cm 0cm 10cm 6cm, clip, width=7cm]{fig/4-contribution/negatif/as-algo-4.pdf}
      \end{figure}
      }
    \end{column}
    \begin{column}{3cm}
      \begin{block}{}
        \only<1>
        {
					Départ de a, un point de l'$alpha$-shape.          
				}
        \only<2>
        {
          On tire les premiers convergents :\\
          \begin{itemize}
            \item $p_{-2} = (1, 0)$
            \item $p_{-1} = (0, 1)$
            \item $p_{0} = (1, 0)$
            \item $p_{1} = (5, 1)$
          \end{itemize}            
        }

        \only<3>
        {
          On étudie le rayon du cercle circonscrit au triangle $T(a, p_{1}, p_{1}-p_{0})$.\\
					\alert{Deux cas se présentent.}
        }
      \end{block}     
    \end{column}
  \end{columns}
\end{frame}

%-----------------------------------------------------------------

\begin{frame}
  \frametitle{Cas 1 - $R_{\alpha} > R_T$}
\begin{columns}[t]
   \begin{column}{7cm}
      \only<1>
      {
        \begin{figure}[H]
          \centering
          \includegraphics[trim = 3cm 0cm 10cm 6cm, clip, width=7cm]{fig/4-contribution/negatif/as-algo-10.pdf}
      \end{figure}
      }
      \only<2,3>
      {
        \begin{figure}[H]
          \centering
          \includegraphics[trim = 4cm 0.5cm 12cm 7.5cm, clip, width=7cm]{fig/4-contribution/negatif/as-algo-10.pdf}
       \end{figure}
      }
      \only<3>
      {
          \begin{block}{$b \notin$ l'$\alpha$-shape.}
          	a et $p_1$ appartiennent au bord d'un disque de rayon -1/$\alpha$ dont le complémentaire inclue b.\\
          \alert{On calcul les convergents suivants...}
        \end{block}
      }   
      \only<4>
      {
        \begin{figure}[H]
          \centering
          \includegraphics[trim = 3cm 0cm 10cm 6cm, clip, width=7cm]{fig/4-contribution/negatif/as-algo-11.pdf}
      \end{figure}
      }
      \only<5,6>
      {
        \begin{figure}[H]
          \centering
          \includegraphics[trim = 3cm 0cm 10cm 6cm, clip, width=7cm]{fig/4-contribution/negatif/as-algo-11a.pdf}
      \end{figure}
      }
   \end{column}
    \begin{column}{3cm}
      \begin{block}{}
        \only<1>
        {
          \alert{CAS 1}\\
          \alert{$R_{\alpha} > R_T$}
          
        }
        \only<2,3>
        {
          Soit $b = p_{1} - p_{0}$.
          \begin{itemize}
            \item $b \in \mathcal{D} \left( R_{\alpha} \right)^C$
          \end{itemize}
        }        
        \only<4>
        {
					$p_{0} + q_2 * p_1$ n'intersecte pas le disque.\\
				  $p_{1}$ est donc un sommet de l'$\alpha$-shape.			

        }
        \only<5,6>
        {
          On reprend l'algorithme.
        }
      \end{block}
      \only<6>
      {
        \begin{block}{Remarque}
          \alert{Si $\alpha$ proche de 0.}\\
          \alert{$\Rightarrow R_{\alpha} >> R_T$.}\\
          \alert{$\Rightarrow$ l'enveloppe convexe.}\\

        \end{block}
      }     
    \end{column}
  \end{columns}
\end{frame}

%-----------------------------------------------------------------

\begin{frame}
  \frametitle{Cas 2 - $R_{\alpha} < R_T$}
 \begin{columns}[t]
   \begin{column}{7cm}
      \only<1>
      {
        \begin{figure}[h!]
          \centering
          \includegraphics[trim = 3cm 0cm 10cm 6cm, clip, width=7cm]{fig/4-contribution/negatif/as-algo-21.pdf}
      \end{figure}
      }
      \only<2,3>
      {
        \begin{figure}[h!]
          \centering
          \includegraphics[trim = 4cm 0.5cm 12cm 7.5cm, clip, width=7cm]{fig/4-contribution/negatif/as-algo-21.pdf}
      \end{figure}
      }
      \only<3>
      {
      	\begin{block}{$b \in$ l'$\alpha$-shape.}
          On ne peut rejoindre a et $p_1$ par le bord d'un disque de rayon 1/$\alpha$ sans avant passer par au moins un autre sommets.
          \end{block}
      }         
      \only<4>
      {
        \begin{figure}[h!]
          \centering
          \includegraphics[width=7cm]{fig/4-contribution/negatif/dicho.pdf}
      \end{figure}
      }
      
      \only<5>
      {
        \begin{figure}[h!]
          \centering
          \includegraphics[trim = 3cm 0cm 10cm 6cm, clip, width=7cm]{fig/4-contribution/negatif/as-algo-22.pdf}
      \end{figure}
      }
      \only<6>
      {
        \begin{figure}[h!]
          \centering
          \includegraphics[trim = 3cm 0cm 10cm 6cm, clip, width=7cm]{fig/4-contribution/negatif/as-algo-23.pdf}
      \end{figure}
      }
      \only<7>
      {
        \begin{figure}[h!]
          \centering
          \includegraphics[trim = 3cm 0cm 10cm 6cm, clip, width=7cm]{fig/4-contribution/negatif/as-algo-24.pdf}
      \end{figure}
      }
      \only<8>
      {
        \begin{figure}[h!]
          \centering
          \includegraphics[trim = 3cm 0cm 10cm 6cm, clip, width=7cm]{fig/4-contribution/negatif/as-algo-25.pdf}
      \end{figure}
      }
      \only<9>
      {
        \begin{figure}[h!]
          \centering
          \includegraphics[trim = 3cm 0cm 10cm 6cm, clip, width=7cm]{fig/4-contribution/negatif/as-algo-26.pdf}
      \end{figure}
      }
    \end{column}
    \begin{column}{3cm}
      \begin{block}{}
        \only<1>
        {
          \alert{CAS 2}\\
          \alert{$R_{\alpha} < R_T$}

        }
        \only<2,3>
        {
          Soit $b = p_{1} - p_{0}$.
          \begin{itemize}
            \item $b \notin \mathcal{D} \left( R_{\alpha} \right)^C$
          \end{itemize}

        }
        \only<4>
        {
          Les rayons sont ordonnés et strictement croissant.\\
          \alert{Recherche dichotomique} du premier sommet dans l'$\alpha$-shape.\\
        }        
        \only<5>
        {
          On étudie le rayon du triangle du milieu :
          \begin{itemize}
            \item $R_{\alpha} < R_T$
          \end{itemize}
          On va chercher en aval.
        }
        \only<6>
        {
          \begin{itemize}
            \item $R_{\alpha} > R_T$
          \end{itemize}
          On va chercher en amont.
        }
        \only<7, 8>
        {
          Jusqu'à arrêter la procédure pour ajouter le premier point dans l'enveloppe...\\          
        }
        \only<8>
        {
          ainsi que tous les suivants jusqu'à $p_1$.     
        }
        \only<9>
        {
          $p_1$ qui devient le nouveau point de départ de l'algorithme.
          
        }  
      \end{block}
     
    \end{column}
  \end{columns}

\end{frame}

%-----------------------------------------------------------------

\begin{frame}
  \frametitle{Résultats}
	\begin{columns}[t]
 		\begin{column}{4cm}
  		\begin{block}{Sommets/$r^{2/3}$}
				\begin{tiny}
					\begin{tabular}{|l|c|c|}
						\hline
						Rayons & $\alpha$ & AS\\
						\hline
            32 & 1.024 & 17.7511\\
            64 & 4.096 & 17.0544\\
            128 & 16.384 & 18.5901\\
            256 & 65.536 & 19.2063\\
            512 & 262.144 & 20.3281\\
            1024 & 1048.58 & 21.0801\\
            2048 & 4194.3 & 21.9657\\
            4096 & 16777.2 & 22.192\\
            8192 & 67108.9 & 22.7541\\
            16384 & 268435 & 23.0402\\
            32768 & 1.07374e+06 & 23.2758\\
            65536 & 4.29497e+06 & 23.6162\\
            131072 & 1.71799e+07 & 23.9101\\
            262144 & 6.87195e+07 & 24.1555\\
            524288 & 2.74878e+08 & 24.4173\\
            1048576 & 1.09951e+09 & 24.5918\\
            2097152 & 4.39805e+09 & 24.7584\\
            4194304 & 1.75922e+10 & 24.9412\\
            8388608 & 7.03687e+10 & 25.0711\\
            16777216 & 2.81475e+11 & 25.2065\\
            33554432 & 1.1259e+12 & 25.309 \\
					  \hline
					\end{tabular}  
				\end{tiny}
			\end{block} 
		\end{column}
		\begin{column}{6.5cm}
				\begin{figure}[h!]
					\centering
				  \includegraphics[width=\linewidth]{fig/res/as.png}
			 	\end{figure}    
        \begin{block}{}
            FIT :  $f1(x) = 25.23*x^{2/3}$\\
            FIT :  $f10(x) = 22.2147*x^{0.674212}$ 
          \end{block} 
		\end{column}
	\end{columns} 
\end{frame}

%-----------------------------------------------------------------
\subsection{$\alpha \geq 0$}
%-----------------------------------------------------------------

\begin{frame}
  \frametitle{$\alpha$-hull avec $\alpha \geq 0$}
  \begin{block}{Définition}
		L'intersection de tous les disques fermés de rayon 1/$\alpha$ qui contiennent tous les points à l’intérieur du disque.
	\end{block} 
 
	\begin{block}{Principe de l'algorithme.}
		\begin{itemize}
			\item Récupérer tous les sommets de l'enveloppe convexe.
			\item Éliminez tous ceux qui sont en trop.
		\end{itemize}	
	\end{block} 


\end{frame}

%-----------------------------------------------------------------

\begin{frame}
  \frametitle{Départ}
\begin{columns}[t]
    \begin{column}{7cm}
      \only<1>
      {
        \begin{figure}[h!]
          \centering
          \includegraphics[trim = 0cm 0cm 0cm 3.5cm, clip, width=7cm]{fig/4-contribution/positif/pas-2.pdf}
        \end{figure}
      }
      \only<2>
      {
        \begin{figure}[h!]
          \centering
          \includegraphics[trim = 0cm 0cm 0cm 3.5cm, clip, width=7cm]{fig/4-contribution/positif/pas-3a.pdf}
        \end{figure}
      }      
    \end{column}
    \begin{column}{3cm}
      \begin{block}{}
        \only<1>
        {
			 		\begin{itemize}
						\item On part de l'enveloppe convexe.
						\item On ne conserve que les sommets.
					\end{itemize}
				}
        \only<2>
        {
          On prend trois sommets succéssifs :
			 		\begin{itemize}
						\item a, b et c.
					\end{itemize}
          On étudie le rayon du cercle circonscrit au triangle T(a, b, c).\\
					\alert{Deux cas se présentent.}
        }
      \end{block}  
    \end{column}
  \end{columns}
\end{frame}

%-----------------------------------------------------------------

\begin{frame}
  \frametitle{Cas 1 - $R_{\alpha} > R_T$}
\begin{columns}[t]
    \begin{column}{7cm}
     \only<1>
      {
        \begin{figure}[h!]
          \centering
          \includegraphics[trim = 0cm 0cm 0cm 3.5cm, clip, width=7cm]{fig/4-contribution/positif/pas-4.pdf}
        \end{figure}
      }
      \only<2>
      {
        \begin{figure}[h!]
          \centering
          \includegraphics[trim = 0cm 0.5cm 5cm 8.5cm, clip, width=7cm]{fig/4-contribution/positif/pas-4.pdf}
        \end{figure}
      
        \begin{block}{$b \notin$ l'$\alpha$-shape.}
          On ne peut pas rejoindre a et c par le bord d'un disque de rayon 1/$\alpha$ sans ommettre b.
        \end{block}
      }
      \only<3>
      {
        \begin{figure}[h!]
          \centering
          \includegraphics[trim = 0cm 0cm 0cm 3.5cm, clip, width=7cm]{fig/4-contribution/positif/pas-5.pdf}
        \end{figure}
      }
      \only<4,5>
      {
        \begin{figure}[h!]
          \centering
          \includegraphics[trim = 0cm 0cm 0cm 3.5cm, clip, width=7cm]{fig/4-contribution/positif/pas-6.pdf}
        \end{figure}
      }
    \end{column}
    \begin{column}{3cm}
      \begin{block}{}
        \only<1,2>
        {
          \alert{CAS 1}
          \alert{$R_{\alpha} >> R_T$}\\

        }
        \only<2>
        {
          \begin{itemize}
            \item $b \notin \mathcal{D} \left( R_{\alpha} \right)$
          \end{itemize}
        }
        \only<3>
        {
          b est un sommet de l'$\alpha-shape$.\\
          $[a, b]$ est une arête.
        }
        \only<4>
        {
          On poursuit l'algorithme :\\
          \begin{itemize}
            \item a devient b.
            \item b devient c.
            \item c est le sommet suivant.
          \end{itemize}
        }
				\only<5>
        { Remarques :\\
          \alert{Si $\alpha$ proche de 0.}\\
          \alert{$\Rightarrow R_{\alpha} >> R_T$.}\\
          \alert{$\Rightarrow$ l'enveloppe convexe.}\\
        }
			\end{block}

    \end{column}
  \end{columns}

\end{frame}

%-----------------------------------------------------------------

\begin{frame}
  \frametitle{Cas 2 - $R_{\alpha} < R_T$}

\begin{columns}[t]
    \begin{column}{7cm}
      \only<1>
      {
        \begin{figure}[h!]
          \centering
          \includegraphics[trim = 0cm 0cm 0cm 3.5cm, clip, width=7cm]{fig/4-contribution/positif/pas-10.pdf}
        \end{figure}
      }
      \only<2>
      {
        \begin{figure}[h!]
          \centering
          \includegraphics[trim = 0cm 0.5cm 5cm 8.5cm, clip, width=7cm]{fig/4-contribution/positif/pas-10.pdf}
        \end{figure}

				\begin{block}{$b \notin$ l'$\alpha$-shape.}
          On peut rejoindre a et c par le bord d'un disque discret de rayon 1/$\alpha$ qui inclue b.
        \end{block}

      }
      \only<3>
      {
        \begin{figure}[h!]
          \centering
          \includegraphics[trim = 0cm 0cm 0cm 3.5cm, clip, width=7cm]{fig/4-contribution/positif/pas-11.pdf}
        \end{figure}
      }
      \only<4>
      {
        \begin{figure}[h!]
          \centering
          \includegraphics[trim = 0cm 0cm 0cm 3.5cm, clip, width=7cm]{fig/4-contribution/positif/pas-12.pdf}
        \end{figure}
      }
      \only<5>
      {
        \begin{figure}[h!]
          \centering
          \includegraphics[trim = 0cm 0cm 0cm 3.5cm, clip, width=7cm]{fig/4-contribution/positif/pas-13.pdf}
        \end{figure}
			}
    \end{column}
    \begin{column}{3cm}
      \begin{block}{}
      \only<1,2>
        {
          \alert{CAS 2}
          \alert{$R_D < R_{\alpha} < R_T$}\\
        }
        \only<2>
        {
          \begin{itemize}
            \item $b \in \mathcal{D} \left( R_{\alpha} \right)$
          \end{itemize}
          
        }
        \only<3,4>
        {
          \begin{itemize}
            \item $b \in$ l'$\alpha$-hull.
            \item b n'est pas un $\alpha$-extrême.
            \item $b \notin$ l'$\alpha$-shape.
          \end{itemize}
        }
        \only<4,5>
        {
          On supprime b de la liste des sommets.\\
          On poursuit la procédure.\\
        }
        \only<5>
        {
          \begin{itemize}
            \item a reste a.
            \item b devient c.
            \item c est le sommet suivant.
          \end{itemize}
        }
      \end{block}
    \end{column}
  \end{columns}
\end{frame}



%-----------------------------------------------------------------
\section{Conclusions}
%-----------------------------------------------------------------

\begin{frame}
\frametitle{Conclusions}
\begin{block}{Conclusions}
  L'algorithme incrémental et output-sensitive de calcul de l'$\alpha$-shape pour $\alpha \leq 0$ donne à partir d'un certain sommet, le suivant. 
  L'organisation spatiale des points est déterminée par les convergents de la pente de l'arête.
\end{block}

\begin{block}{Poursuites}
  Les relations de distance entre les sommets de l'enveloppe convexe, données par l'$\alpha$-shape pour $\alpha > 0$, reste encore à étudier, au moyen d'une approche ``Top-down''.
  
\end{block}

\end{frame}



%----------------------------------------------------------------------------------------
\end{document} 
