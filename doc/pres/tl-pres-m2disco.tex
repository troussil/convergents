%----------------------------------------------------------------------------------------
%	PACKAGES AND THEMES
%----------------------------------------------------------------------------------------

\documentclass{beamer}

\usepackage{times}
\usepackage[T1]{fontenc}
\usepackage[english,francais]{babel}
\usefonttheme{professionalfonts}

\usepackage[]{beamerthemeliris} % paquet theme, options: nogradient,nobackground


\usepackage{graphicx} %paquet graphiques
\usepackage{epsfig}

\usepackage{booktabs} % Allows the use of \toprule, \midrule and \bottomrule in tables
\usepackage[utf8]{inputenc} %codage
\usepackage{lmodern}
%\usepackage[french]{babel}

%----------------------------------------------------------------------------------------
%	TITLE PAGE
%----------------------------------------------------------------------------------------

\title[Présentation M2Disco]{Structure d'un cercle discret en dimension deux} % The short title appears at the bottom of every slide, the full title is only on the title page

\author{\bsc{Lafond} Thomas} % Your name
\institute[LIRIS] % Your institution as it will appear on the bottom of every slide, may be shorthand to save space
{
Laboratoire d'InfoRmatique en Image et Syst\`{e}mes d'information \\ % Your institution for the title page
\medskip
\textit{-- Présentation d'équipe : M2Disco --}\\
\medskip
Encadrant : \bsc{Roussillon} Tristan 
}
\date{10 Juin 2013} % Date, can be changed to a custom date

\begin{document}

\begin{frame}
\titlepage % Print the title page as the first slide
\end{frame}

\begin{frame}
\frametitle{Tables des matières} % Table of contents slide, comment this block out to remove it
\setcounter{tocdepth}{1}
\tableofcontents % Throughout your presentation, if you choose to use \section{} and \subsection{} commands, these will automatically be printed on this slide as an overview of your presentation
\end{frame}

%----------------------------------------------------------------------------------------
%	PRESENTATION SLIDES
%----------------------------------------------------------------------------------------

%------------------------------------------------
\section{Parcours} 
%------------------------------------------------

\subsection{Parcours} 
\begin{frame}
%\frametitle{Parcours}
\begin{block}{Licence de Mathématiques et d'Informatique}
\begin{itemize}
\item Mathématiques : L1, L2, L3.
\item Informatique : L1, L2.
\item Domaine Scientifique : Science Physique, Chimie et Biologie.
\end{itemize}
\end{block}

\begin{block}{Master Statistiques, Informatique et Techniques Numériques}
\begin{columns}[t]

\begin{column}{5.5cm}
Trois composantes principales
\begin{itemize}
\item Probalités et Statistiques 
\item EDP et Calcul scientifique
\item Developpement Informatique
\end{itemize}
\end{column}

\begin{column}{4.5cm}
Une certaine ouverture  
\begin{itemize}
\item \alert{Stages}
\item Calcul parallèle
\item Mécanique des fluides
\end{itemize}
\end{column}
\end{columns}  
\end{block}

\begin{block}{Stages précédents}
\begin{itemize}
\item Modélisation Hydrologique - Irstea - 5 Mois
\item Décomposition de domaine - MmodD / Lyon 1 - 2 Mois
\end{itemize}
\end{block}

\end{frame}

%------------------------------------------------
\section{Motivation} 
%------------------------------------------------

\subsection{Géométrie Discrète} 
\begin{frame}
\frametitle{Géométrie Discrète}
Une discipline à la croisée de bien de domaines scientifiques entre les mathématiques et l'informatique.\\
Il s'agit principalement d'étudier la géométrie et la topologie d'objets sur une grille régulière. Dans notre cas, on s'intéresse à $Z^2$.
\end{frame}

\subsection{Approche Constructive}
\begin{frame}
\frametitle{Approche Constructive}
Exemple : Disque Discret : L'ensemble des points aux coordonnées entières tel que $Ax^2+By^2 + Cx + Dy + E \leq 0$ soit à l'intérieur et $Ax^2+By^2 + Cx + Dy + E \geq 0$ soit à l'extérieur.\\
Construit un modèle géométrique uniquement avec des entiers.\\
+ Permet le calcul exact en évitant tous les problèmes d'approximations et d'arrondis.\\
+ Adapté à l'analyse d'images.\\
+ Optimisation possible.\\
- Plusieurs représentations sont possibles pour construire un même objet. Si les différences graphiques seront faibles, les calculs seront eux bien différents et souvent icompatibles.
\end{frame}

\subsection{Modélisation de contours/ bords discrets} 
\begin{frame}
\frametitle{Modélisation de contours/ bords discrets}
\end{frame}

%------------------------------------------------
\section{Existant} 
%------------------------------------------------

\subsection{Enveloppe convexe - Har-Peled} 
\begin{frame}
\frametitle{Enveloppe convexe - Har-Peled}
Notion de convergents, de parité...
\end{frame}

\subsection{Triangulation de Delaunay}  % ?? là / pas là ??
\begin{frame}
\frametitle{Triangulation de Delaunay}
On parle un peu de triangulationsde delaunay, de leur lien au alpha-hull, au alpha shape.
\end{frame}

%------------------------------------------------
\section{Contributions} 
%------------------------------------------------

\subsection{Généralisation de Har-Peled - $\alpha \leq 0$} 
\begin{frame}
\frametitle{Généralisation de Har-Peled - $\alpha \leq 0$}

\end{frame}

\subsection{$\alpha \geq 0$} 
\begin{frame}
\frametitle{$\alpha \geq 0$}
\end{frame}

\subsection{Protocole} 
\begin{frame}
\frametitle{Protocole}
\end{frame}

%------------------------------------------------
\section{Résultats} 
%------------------------------------------------

\subsection{Enveloppe convexe} 
\begin{frame}
\frametitle{Enveloppe convexe}
\end{frame}

\subsection{$\alpha \leq 0$} 
\begin{frame}
\frametitle{$\alpha \leq 0$}
\end{frame}

\subsection{$\alpha \geq 0$} 
\begin{frame}
\frametitle{$\alpha \geq 0$}
\end{frame}

%------------------------------------------------
\section{Suite} 
%------------------------------------------------

\subsection{Poursuite du projet} 
\begin{frame}
\frametitle{Poursuite du projet}
\end{frame}

\subsection{Poursuite personnelle} 
\begin{frame}
\frametitle{Poursuite personnelle}
\end{frame}

%----------------------------------------------------------------------------------------
\end{document} 
